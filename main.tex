\documentclass[11pt,dvipsnames]{memoir}
\usepackage[dvipsnames,table]{xcolor}
\usepackage[utf8]{inputenc}
\usepackage[T1]{fontenc}
\usepackage{babel}
\usepackage{amsmath, amssymb, amsthm}
\usepackage{todonotes}
\usepackage{graphicx}
\usepackage{tikz}
\usepackage[colorlinks=true, linkcolor=blue]{hyperref}
\usepackage{hhline}
\usepackage{physics}
\usepackage{mathtools}
\usepackage{tikz}
\usepackage{bm}
\usepackage[linesnumbered,ruled]{algorithm2e}
\usepackage{listings}
\usepackage{float}
\usepackage[bibstyle=nature,maxbibnames=10,backend=bibtex,sorting=none]{biblatex}
\usepackage{subcaption}
\usepackage[font={small}]{caption}
\usepackage{xfrac}
\usepackage{multirow}
\usepackage{makecell}
\usepackage[chapter]{minted}

% ================================================================================
% Configuration of subfigures
% ================================================================================
\DeclareCaptionLabelFormat{subfigure}{\textbf{#2.}}
\captionsetup[subfigure]{
  font={bf}, skip=1pt, singlelinecheck=false, labelformat=subfigure, subrefformat=parens
}
\captionsetup{subrefformat=subfigure}

% ================================================================================
% Colors for table headers
% ================================================================================
\colorlet{theader}{ProcessBlue!30}
\colorlet{tsubheader}{ProcessBlue!20}

\addbibresource{my_papers_phd.bib}
\addbibresource{ref.bib}


% ================================================================================
% Theorem environments
% ================================================================================
\theoremstyle{definition}
\newtheorem{example}{Example}[chapter]
\newtheorem{lemma}{Lemma}
\newtheorem{theorem}{Theorem}


% ================================================================================
% Symbols to be used in figures' captions, they correspond to the markers
% ================================================================================
\DeclareRobustCommand\tikzdot{\tikz[overlay,yshift=0.5ex] \fill[blue!90] (0.ex,0.ex) circle (0.3ex);}
\DeclareRobustCommand\tikzcircle{\tikz[overlay,yshift=0.5ex] \draw[red,thick] (0.ex,0.ex) circle (0.4ex);}
\DeclareRobustCommand\tikzquad{\tikz[overlay,yshift=0.ex,xshift=0.1ex] \draw[green,thick] (0.ex,0.ex) rectangle (1ex,1ex);}

% ================================================================================
% Title page info
% ================================================================================
\title{Validation and benchmarking of quantum annealing technology}
\author{Konrad Jałowiecki}
\date{January 2022}


% ================================================================================
% Styles and definitions
% ================================================================================
\def\NN{\mathbb{N}}
\def\RR{\mathbb{R}}
\def\CC{\mathbb{C}}

\DeclareMathOperator*{\argmax}{arg\,max}
\DeclareMathOperator*{\argmin}{arg\,min}

\def\clockop{\altmathcal{C}}
\def\coefmatrix{\altmathcal{A}}
\def\JJ{\mathcal{J}}
\def\tout{t_\text{out}}
\def\tin{t_\text{in}}
\def\ttout{t_\text{out}^\text{timetable}}
\def\ttin{t_\text{in}^\text{timetable}}
\def\pt{p^\text{timetable}}
\def\pmin{p_\text{min}}
\def\tauu{\tau_{(1)}}
\def\tauuu{\tau_{(2)}}

\linespread{1.15}
% minted style
\usemintedstyle{default}
\colorlet{lstbg}{black!80}

\begin{document}

\begin{titlingpage}
  \begin{center}
    \includegraphics[width=0.4\textwidth]{figures/iitis_logo}\\
    \vspace{0.5em}
    \textsc{\large Institute of Theoretical and Applied Informatics, Polish Academy of Sciences}
    \vspace*{1in}
    \hrule
    \vspace*{0.5em}
    \textsc{\huge Validation and benchmarking of quantum annealing technology}
    \vspace*{0.5em}
    \hrule
    \vspace*{1em}
    \textsc{\large Doctoral dissertation}
    \par
    \vspace{1.5in}
    {\large mgr Konrad \textsc{Jałowiecki}}\\
    \vspace{0.25in}
    Supervisor:\\ dr hab. Bartłomiej Gardas\\
    \vspace{0.25in}
    Co-supervisor:\\ dr hab. inż. Łukasz Pawela\\
    \vfill
    {Gliwice, \today}
  \end{center}
\end{titlingpage}

\frontmatter

\tableofcontents*
\newpage
\chapter{Acknowledgements}

I am deeply grateful to my supervisor, dr hab. Bartłomiej Gardas, whose unwavering support and insightful guidance have been instrumental in shaping this doctoral thesis. His expertise, encouragement, and mentorship have been invaluable, and I am truly fortunate to have had the opportunity to work under his supervision.

I would also like to express my sincere appreciation to my co-supervisor, dr hab. inż. Łukasz Pawela, for his constructive feedback, especially in the field of software engineering. His expertise and willingness to share knowledge have significantly enriched the quality of this thesis.

I would also like to express my gratitude to all my colleagues from the Institute who contributed to this thesis via many fruitful conversations I had with them. In particular, I would like to thank Krzysztof Domino for sharing his knowledge and expertise in the railway dispatching field.

Additionally, I extend my deepest gratitude to my friends, Alexander Juda and Michał Stęchły, for taking the time to read parts of this thesis. Their valuable feedback contributed greatly to improving the readability and overall quality of this work.

This project was partially supported by the National Science Center (NCN), Poland, under Projects: Sonata Bis 10, No. 2020/38/E/ST3/00269
and the National Centre for Research and Development (NCBR), Poland, under Project No. POIR.01.01.01-00-0061/2. I would also like to thank The Quantum Data Center Corporation for providing me with access to several GPUs used for benchmarks presented in this thesis.

%%% Local Variables:
%%% mode: latex
%%% TeX-master: "main"
%%% End:

%\addtocontents{toc}{\setlength{\cftchapterindent}{\cftchapternumwidth}}
\chapter{Published work}

\begin{refsection}
\nocite{Jaowiecki2020,Rams2021,JALOWIECKI2021107728,e25020191,omnisolver}
\printbibliography[heading=none]
\end{refsection}

%%% Local Variables:
%%% mode: latex
%%% TeX-master: "main"
%%% End:

%\addtocontents{toc}{\setlength{\cftchapterindent}{\cftchapternumwidth}}
\chapter*{Abstract}

In this thesis, we focus on the problem of validating and benchmarking quantum annealers in a practical context. To this end, we propose two algorithms solving real--world problems and test how well they perform on the current generation of quantum annealers. The first algorithm allows for solving dynamics of arbitrary (complex or real) dynamical systems. The second of the proposed algorithms is suitable for solving a particular family of railway dispatching problems, delay and conflict management on single--track railway lines.

%%% Local Variables:
%%% mode: latex
%%% TeX-master: "../main"
%%% End:

\chapter{Streszczenie}

W niniejszej pracy skupiamy się na problemie walidowania i benchmarkowania wyżaraczy kwantowych w
praktycznym kontekście. W tym celu, przedstawiamy dwa algorytmy służące do rozwiązywania rzeczywistych
problemów, oraz sprawdzamy, jak dobrze sprawdzają się na obecnej generacji wyżaraczy kwantowych.
Pierwszy z algorytmów pozwala na rozwiązywanie dynamiki kwantowych układów (lub, w gruncie rzeczy,
dowolnych układów dynamicznych). Drugi z przedstawianych algorytmów może skolei zostać użyty do
rozwiązywania pewnego podzbioru problemów kolejowych\todo{Ask KD how to translate ``railway dispatching problem''}:
zarządania opóźnieniami i konfliktami w sieciach kolejowych o jednej linii.
Oceny działania obu w.w. algorytmów na bieżącej generacji wyżaraczy D-Wave dokonujemy z pomocą dwóch,
nowatorskich, klasycznych strategii rozwiązywania szkieł spinowych Isinga, które również prezentujemy
w pracy. Piewrszym z nich, jest opierający się na sieciach tensorowych heurystyczny algorytm
stworzony specjalnie do rozwiązywania szkieł spinowych zdefiniowanych na grafach przypominających
topologię Chimera, co sprawia, że idealnie nadaje się do wyznaczania referencyjnych rozwiązań,
do których można porównać wyniki z fizycznych wyżarzaczy. Drugim z prezentowanych podejść jest masywnie
równoległa implementacja wyczerpującego przeszukiwania całej przestrzeni rozwiązań, tzw. brute-force.
Mimo, że użycie algorytmu brute-force jest ograniczone do instancji o niewielkich rozmiarach,
posiada on tę zaletę, że może wyznaczać niskoenergetyczne spektrum, oraz certyfikować rozwiązaniea.
W związku z tym, algorytm ten może slużyć do uzyskania dodatkowego wglądu w strukturę przestrzeni rozwiązań.
Wyniki otrzymane w naszych eksperymentach sugerują, że już współczesne wyżarzacze są w stanie
uchwycić dynamikę prostych, dwupoziomowych układów kwantowych w specyficznym reżimie parametrów,
oraz mogą znaleźć dobrej jakości rozwiązania instancji kolejowych problemów. Wreszcie, nasze eksperymenty
pokazują jasno, że obecna generacja wyżaraczy D-Wave nie jest idealna. Wymieniamy instancje problemów,
dla których wyżarzanie nie potrafily znaleźść wysokojakościowych, lub nawet dopuszczalnych rozwiązań.
Tam gdzie to możliwe, omawiamy również możliwe wyjaśnienie dlaczego niektóre z prezentowanych instancji
mogą być dla wyżaraczy wymagające.

%%% Local Variables:
%%% mode: latex
%%% TeX-master: "../main"
%%% End:


\mainmatter

\chapter*{Introduction}
The previous century has witnessed what is now called the digital revolution. Introduction of computers dramatically altered multiple aspects of our lives. In particular, almost every area of science benefitted tremendously from the increasingly available computing power. Physics was no exception and numerical simulations assisting experiments are now a commonplace.

However, simulating dynamics of quantum systems -- a holy grail for modern computational physics -- is still a highly nontrivial task. A natural question to ask is whether advances in technology can bring us closer to this goal.


Moore's law, that so far well predicted rate of growth of computational power of classical computers, is expected to slow down in the years to come. Even more importantly, using any Turing-machine compatible model of computations, simulating quantum systems must result in an exponential slowdown. For those two reasons alone one should expect that efficient simulations of many-body quantum systems, if at all possible, can be done only if one reaches beyond classical architectures of computing devices.

In 1980s Richard Feynmann suggested that quantum devices can be used to carry simulations of quantum systems. This realization opened new avenues of research and led to several different (albeit equivalent) computing paradigms, including quantum annealing and quantum gate computing. Even long before experimental devices implementing those paradigms were constructed, several notable algorithms that could utilize them were developed, which further fueled interest in quantum computing. In recent years we observed unprecedented development of hardware that brings us closer to the quantum revolution. In particular, several implementations of gate-based quantum computers and quantum annealers were developed and made publicly available, allowing scientist to benchmark them and further research their possible applications.

However promising, near-term quantum computers are far from perfect. One could ask whether already these noisy devices can be used for simulating dynamics of quantum systems. In this thesis, we are aiming to show that this is indeed the case, and present-day quantum annealers can be used in hybrid, parallel-in-time algorithms simulating simple quantum systems. While our algorithm is only a proof of concept, it exemplifies possible directions of future research.

Despite the fact that it is known that efficient simulations of sufficiently large quantum systems are outside the reach of the classical computers, it is still unclear where the boundary of quantum supremacy lies. Thus, it is still reasonable to seek new algorithms to push the limits of classical architectures further. This motivated us for developing exhaustive search (a.k.a. brute-force) solver for Ising spinglasses utilizing massively parallel Nvidia CUDA architecture. The solver is capable of finding low-energy spectrum of the spinglass and may be therefore utilized as a tool for benchmarking other heuristic solvers, especially ones that cannot certify their solution.

The structure of this work is as follows. In chapter \ref{chapter:near-term}, we provide a short yet self-contained introduction to the near term technologies used in the chapters that follow. At first, we begin with the description of Nvidia CUDA technology, an example of massively parallel architecture that gained popularity due to multiple applications in numerical computations and machine learning. The rest of the chapter is devoted to quantum annealing and the description of one of its implementations, namely the D-Wave quantum annealers.

The next chapter describes how CUDA can be harnessed for finding low energy spectrum of the Ising spin glass. Here we describe our novel algorithm for efficiently performing an exhaustive search (also known as a bruteforce search) over possible system's configuration and discuss results of extensive benchmarks we subjected our algorithm to. We also show an example application of our bruteforce solver to benchmarking the recently introduced algorithm based on MPS ansatz.

Parallel computations on classical computers are subjected to the Amdahl's law which states that the speedup resulting from scaling execution units horizontally is limited by the serial part of the algorithm. Quantum computers, on the other hand, operate inherently in parallel. While quantum supremacy is still to be demonstrated and currently available hardware is far from perfect, one observes steady improvement in the available technology. This motivated us to develop a proof of concept algorithm for simulating dynamics of a (in principle arbitrary) dynamical system using quantum annealer which we describe in chapter \ref{chapter:simulating}. Here we also discuss experimental results obtained from D-Wave annealers available at the moment of writing and show that even those near-term devices can capture the dynamics of a simple two-qubit system.
\chapter{Ising model and QUBO problem}

Quantum annealers are fundamentally different from classical computers. For
one, they don't execute programs written as a sequence of instructions in their
memory. Instead, they are single--purpose devices capable (in principle) of
solving a specific optimization problem. Namely, annealers are designed to find
the lowest energy configuration (called \emph{ground state}) of the Ising
spin--glass model, which we introduce in this chapter.

The potential usefulness of quantum annealers stems from the fact that the
optimization problem they are supposed to solve is hard for classical
computers. But what does it formally mean for a problem to be hard? To answer
this question, we will need a brief recap of complexity theory, which is a
second point of this chapter.

Finding a ground state of the Ising spin--glass model may be hard for classical
computers, but there exists a plethora of heuristic, classical algorithms
capable of finding solutions that are at least ``good enough''. As the next
point in this chapter, we provide a brief overview of the most popular ones.
These algorithms will serve as a baseline for comparison with quantum annealing
and a recent tensor network--based approach discussed later in the thesis.

As the last point in the chapter, we define the Quadratic Unconstrained Binary
Optimization (QUBO) problem, which is equivalent to the problem of finding the
ground state of the Ising model. We will use the QUBO formulation on several
occasions in the thesis, as it oftentimes results in a more natural phrasing of
the problem, or leads to a surprising performance improvement when implementing
software solvers.

\section{Ising model}

The Ising spin--glass model was introduced in 1920 by Wilhelm Lenz \cite{lenz}
as a description of ferromagnetism in solids but is named after his student
Ernst Ising, who studied and solved it in the one-dimensional case
\cite{ising}. For purposes of this thesis, however, we will forget about the
physical interpretation of the model, treating it merely as a description of a
particular optimization problem.

Consider a simple\footnote{That is, one that does not contain duplicate edges
  or loops.}, undirected graph $G = (V, E)$ with $N$ nodes labeled by consecutive
natural numbers. With each node, $i \in V$ we associate a spin variable $s_i
  \in \{-1, 1\}$. To each edge $\{i, j\} \in E$, we assign an interaction
strength $J_{ij}$ and to each node $i \in V$ we assign a local magnetic field
$h_i$. Here, all $J_{ij}$ and $h_i$ are real numbers. For such a system, one
can define the following energy function (Hamiltonian):
\begin{equation}
  \label{eq:ising-hamiltonian}
  H(\mathbf{s}) = \sum_{\langle i, j \rangle} J_{ij} s_i s_j +  \sum_{i=1}^N h_i s_i,
\end{equation}
where $\mathbf{s} = (s_i, \ldots, s_N)$ and the first sum runs over all edges exactly once\footnote{
  In the
  literature, the Ising Hamiltonian \eqref{eq:ising-hamiltonian} is often negated. However, the
  definition provided here is consistent with the one used by D-Wave, and thus more suitable for use
  in this thesis.}.

For fixed model coefficients, one is typically interested in finding its
\emph{ground state}, a configuration $\mathbf{s}$ that minimizes $H$. More
generally, it might be desirable to search for $k \ll 2^N$ configurations with
the lowest energy, a so-called \emph{low--energy spectrum of size $k$}.

\begin{figure}[H]
  \centering
  \includegraphics{figures/spins.pdf}
  \caption{Symbolic representation of Ising spin--glass defined on the graph with $N=16$ nodes. Here, $h_i$ is a real number associated with $i$-th node, and $J_{ij}$ denotes coupling strength associated with an edge between $i$-th and $j$-th node. The configuration of each spin is marked by a red arrow pointing upwards (+1) or a blue arrow pointing downwards (-1).}
  \label{fig:my_label}
\end{figure}

\begin{example}
  Consider an Ising model instance with 3 spins given by the Hamiltonian $H$:
  \begin{equation}
    \label{eq:isingexample}
    H(s_1, s_2, s_3) = s_1 - s_2 +2s_3 - 2s_2s_3 + 3s_1s_2
  \end{equation}
  This instance has 8 possible states:

  \begin{table}[h]
    \begin{center}
      \begin{tabular}{|c|c||c|c|}
        \hline
        $\mathbf{s}=(s_1, s_2, s_3)$ &
        $H(\mathbf{s})$              &
        $\mathbf{s}=(s_1, s_2, s_3)$ &
        $H(\mathbf{s})$                                      \\\hline
        (-1, -1, -1)                 & -1 & (1, -1, -1) & -5 \\ \hline
        (-1, -1, 1)                  & 7  & (1, -1, 1)  & 3  \\ \hline
        (-1, 1, -1)                  & -5 & (1, 1, -1)  & 3  \\ \hline
        (-1, 1, 1)                   & -5 & (1, 1, 1)   & 3  \\ \hline
      \end{tabular}
    \end{center}
  \end{table}
  Observe that the lowest attainable energy is -5 and there are 3 states with exactly this energy. Hence, all the configurations $(-1, 1, -1)$, $(-1, 1, 1)$, $(1,
    -1, -1)$ are ground states. This situation, i.e. when two or more states share the same energy, is called \emph{degeneracy} and the states in question are called \emph{degenerate}. For this instance, a low energy spectrum of size $k=5$ comprises all
  ground states, the $(-1, -1, -1)$ state with $H(-1, -1, -1) = -1$ and any of the states with
  $H(\mathbf{s})=3$.
\end{example}

Despite the simple formulation, the problem of finding a ground state of Ising
spin--glass is computationally hard \cite{barahoma}. Before expanding on this
idea, let us first introduce the hierarchy of complexity classes.

\section{Algorithms and complexity}

Solving the computational problem requires a suitable \emph{algorithm}, a
description of steps to be performed by a computer to obtain a solution. It is
hardly surprising that some problems might be solved in more than one way, i.e.
there might exist different algorithms performing essentially the same task.
Different algorithms solving the same problems might vastly differ in their
demand on various resources, like memory or time needed to execute them. In
practice, execution time (and usage of other resources) of a given algorithm
might also vary between its implementations, depending on factors like
programming language or libraries used and the hardware it is executed on.
Moreover, measuring execution time can only tell us how the given algorithm's
implementation performs on a specific problem. But if we increase the problem
size tenfold, will the execution time be 10 times slower? Or maybe 100 times
slower? Or maybe it will remain unchanged? Clearly, measuring execution times
is useful, but cannot be used for comparing algorithms (instead of their
implementations). Instead, it is more useful to characterize algorithms based
on how their execution time scales (asymptotically) with increasing problem
size \cite{arora}. For instance, given an algorithm with execution time roughly
proportional to the input size $N$, one might suspect that for problem
instances large enough, it will perform better than the one with execution time
proportional to $N^{2}$. This characteristic, known as computational
complexity\footnote{Note that here we focus only on \emph{time complexity}, but
  other notions like memory complexity can be defined similarly}, can be
formalized by a big-$O$ notation (see appendix for a more detailed
description). Using this notation, the algorithms from the above example would
be classified as $O(N)$ and, $O(N^{2})$ respectively.

\section{Complexity classes}
Although there might exist multiple algorithms for solving a given
computational problem, one might consider the minimal time complexity required
to do so. More generally, one might group computational problems based on their
demand on resources. In this view, sets of similar problems are called
\emph{complexity classes} \cite{arora}. The definition of some complexity
classes might also be restricted to specific types of problems. For instance,
one might consider only decision problems, i.e. problems to which the answer is
yes or no \cite{arora}.

One of the fundamental complexity classes is \textbf{P}, a class of decision
problems solvable in polynomial time on a deterministic Turing Machine
\cite{arora}. Another class, \textbf{NP}, comprises all decision problems whose
solution can be verified in polynomial time using a deterministic Turing
Machine \cite{arora}. One can immediately see that \textbf{P} $\subset$
\textbf{NP}. Indeed, if a problem is solvable in polynomial time, then it is
also trivially verifiable in polynomial time. However, it is not immediately
obvious if the inclusion is strict, and whether \textbf{P} $\ne$ \textbf{NP} is
one of the most important, yet unsolved problems in theoretical computer
science \cite{fortnow}. The class of \textbf{NP--hard} problems comprises all
the problems that are at least as hard as every problem in \textbf{NP}. More
formally, given decision problem $S$ is \textbf{NP--hard} if and only if
solving every problem in \textbf{NP} can be reduced to solving $S$ a polynomial
number of times \cite{arora}. A particular subclass of \textbf{NP--hard}
problems, \textbf{NP--complete}, is an intersection of \textbf{NP} and
\textbf{NP--hard} \cite{arora}. Figure \ref{fig:complexity} shows the
relationship between the discussed complexity classes, both under assumptions
\textbf{P} = \textbf{NP} and \textbf{P} $\ne$ \textbf{NP}.

Problems in the complexity class \textbf{P} are often considered tractable, or
efficiently solvable, whereas problems not in \textbf{P} are perceived as hard
and computationally demanding, a statement known as the Cobham's thesis
\cite{cobham, arora}. At first, one might find it strange and unintuitive.
After all, a decision problem for which the best known algorithm runs in
$O(N^{10^5})$ time is definitely in \textbf{P}, but can hardly be called
efficiently solvable. However, such large polynomial complexities are rarely
encountered in practice. Furthermore, even in such cases, it is not uncommon
that a better algorithm is found shortly after the original one is discovered
\cite{arora}.

\begin{figure}
  \includegraphics[width=\textwidth]{figures/complexity_new.pdf}
  \caption{Hierarchy of basic complexity classes. Under the assumption of $\textbf{P} \ne \textbf{NP}$ (left), the hierarchy is richer and there exist problems in \textbf{NP}  that are not \textbf{NP}--complete. Under the opposite assumption (right), the hierarchy collapses. Notice that in both cases there exist \textbf{NP}--hard problems that are not in \textbf{NP}
  }
  \label{fig:complexity}
\end{figure}

\section{Ising model and complexity}

Thus far, we only discussed classes of decision problems. How do they relate to
the problem of finding a ground state of the Ising model? Suppose we are given
an Ising model instance with hamiltonian $H$ and let $x \in \RR$ be some fixed
number. Consider the problem of deciding whether there exists $\mathbf{s}$ such
that $H(\mathbf{s}) \le x$. We will call this problem a \emph{decision version
  of the Ising problem}.

If we can minimize $H$, we can also solve the decision problem by simply
finding a ground state and checking if its energy exceeds threshold $x$. On the
other hand, the sole capability of solving the decision version of a problem
does not give us an algorithm for solving an original optimization problem.
Therefore, one can see that the optimization problem is at least as hard as the
corresponding decision problem. Of course, the same reasoning applies for other
optimization problems. Hence, if the decision version of an optimization
problem is \textbf{NP--hard}, the optimization problem is sometimes also called
\textbf{NP--hard}, even if it slightly abuses the terminology. For simplifying
the vocabulary, in what follows we will use this slightly imprecise but more
concise convention.

It was shown that finding a ground state of the Ising spin--glass in the case
of three-dimensional lattices, as well as for some planar graphs, is
\textbf{NP--hard} \cite{barahoma}. The decision version of the problem is
\textbf{NP--complete}. Multiple known \textbf{NP--hard} problems, such as
Travelling Salesman Problem or Hamiltonian Cycles Problem, are reducible to
finding the ground state of Ising spin-glass \cite{lucas}.

As a side note, one might be tempted to think that the \textbf{NP--hard}ness of
finding Ising model's ground state is trivial, because its enormous state space
comprises $2^{N}$ states. However, it is important to remember that the size of
the solution space itself is not enough to reason about the problem's hardness.
For instance, the number of possible spanning trees in the complete graph of
$N$ vertices is $N^{(N-2)}$, yet the minimum spanning tree problem is solvable
in polynomial time via several algorithms \cite{clrs}.

\section{Algorithms for solving Ising model}

\begin{figure}
  \centering
  \includegraphics[width=\textwidth]{figures/pt_and_sa.pdf}
  \caption{Schematic representation of simulated annealing \textbf{(a)} and parallel tempering \textbf{(b)} algorithms. In simulated annealing, a single copy of the system is simulated. The temperature of the system is decreasing with each epoch, thus reducing movement through the state space. In parallel tempering, several copies (replicas) of the system are simulated, each with a fixed temperature. Hotter replicas move through the state space rapidly and less predictably, while colder replicas move conservatively. Between epochs, replicas can exchange states, which helps avoid being stuck at local minima. Exchanging replicas can also be viewed as reseeding of the colder replicas by randomized solutions provided by hotter replicas.}
  \label{fig:sa}
\end{figure}

As is the case with many \textbf{NP--hard} optimization problems, there are
many heuristic approaches for solving the Ising model. One family of such
algorithms relies on the Metropolis-Hastings \cite{beichl} algorithm for
sampling from the underlying Boltzmann distribution. In \emph{simulated
  annealing} \cite{cook, isakov}, one lowers the temperature over time. Thus, the
chance of accepting a locally worse solution is greater at the start of the
algorithm and decreases with each iteration, which helps avoid getting stuck in
a local minimum. In another approach from the same family, \emph{parallel
  tempering}, one simulates several replicas of the system, each of them in a
different temperature. Neighboring replicas are allowed to exchange states,
with exchange probability depending on their energy and temperature difference
\cite{swendsen}. Replicas with higher temperatures explore state space rapidly
(thus reseeding the algorithm), while ones with lower temperatures refine the
best solutions found so far. Various modifications of the aforementioned
algorithms exist. For instance, one could employ isoenergetic cluster moves
\cite{zhu} or adaptive choosing the number of sweeps performed between replica
exchanges \cite{bittner}. Population annealing is another Monte Carlo method,
sharing similarities with simulated annealing and parallel tempering
\cite{wang}. Other approaches for solving Ising spin--glasses include methods
involving branch--and--bound framework \cite{rendl}, its chordal extensions
\cite{baccari} or methods based on simulating dynamical systems \cite{sheldon}.

\section{Quadratic Unconstrained Binary Optimization}

Let us now introduce the Quadratic Unconstrained Binary Optimization (QUBO)
problem, which is essentially the same as the problem of finding the ground
state of Ising spin--glass, except that in QUBO we use binary variables $q_{i}
  \in \{0, 1\}$ instead of $\pm 1$ spin variables. To distinguish between the two
problems, we will use symbols $a_{ij}$ and $b_{i}$ to denote respectively
quadratic and linear coefficients in QUBO, so the energy function to be
minimized can be written as:
\begin{equation}
  \label{eq:qubo}
  F(q_1, \ldots, q_N) =  \sum_{\langle i, j \rangle} a_{ij} q_i q_j + \sum_{i=1}^N b_iq_i,
\end{equation}
where, as in the Ising model, the first sum runs through all the edges of the graph on which
the problem is defined.

The QUBO and Ising formulations are essentially equivalent. Indeed, it is
always possible to transform the Ising Hamiltonian into the QUBO cost function
by a linear substitution of variables $s_i \mapsto 2q_i-1$. Then, one obtains
the function $F$ like in the equation \eqref{eq:qubo}, with the following
values for $a_{ij}$ and $b_{i}$:
\begin{equation}
  \label{eq:toQUBO}
  a_{ij}= 4J_{ij},
  \quad
  b_i= 2h_i - 2 \sum_{\langle i, j \rangle} J_{ij},
\end{equation}
where the last sum runs over all neighbors of node $i$. The obtained function $F$ differs from the original $H$ by the constant offset
\begin{equation}
  F(\mathbf{q}) - H(\mathbf{s}) =\sum_{i=1}^N h_i - \sum_{\langle i, j \rangle} J_{ij},
\end{equation}
which is irrelevant to the optimization process.

\begin{example}
  Let us go back to the previous example and convert the Ising Hamiltonian from the equation \eqref{eq:isingexample}
  to an equivalent QUBO. We compute the coefficients using formulas from the equation \eqref{eq:toQUBO} to obtain:
  \begin{equation}
    \begin{alignedat}{6}
      b_{1} &= 2h_{1} - 2 J_{12} = -4            &\quad & a_{12} &= 4J_{12}=12\\
      b_{2} &= 2h_{2} - 2(J_{12} + J_{23}) = -4 &\quad & a_{23} &= 4J_{23}=-8\\
      b_{3} &= 2h_{3} - 2J_{23} = 8.             & & &
    \end{alignedat}
  \end{equation}
  This gives the following energy function:
  \begin{equation}
    F(q_{1}, q_{2}, q_{3}) = -4q_{1} -4q_{2}+8q_{3}+12q_{1}q_{2}-8q_{2}q_{3}.
  \end{equation}
  The possible system configurations and their energies are listed in the table below.
  Observe that all configurations differ from the ones by $1$, which is exactly what we get
  if we computed the offset explicitly:
  \begin{equation}
    \mbox{offset} = h_{1} + h_{2} + h_{3} - J_{12} - J_{23} = 1.
  \end{equation}

  \begin{table}[ht!]
    \begin{center}
      \begin{tabular}{|c|c||c|c|}
        \hline
        $\mathbf{q}=(q_1, q_2, q_3)$ &
        $F(\mathbf{q})$              &
        $\mathbf{q}=(q_1, q_2, q_3)$ &
        $F(\mathbf{q})$                                    \\\hline
        (0, 0, 0)                    & 0  & (1, 0, 0) & -4 \\ \hline
        (0, 0, 1)                    & 8  & (1, 0, 1) & 4  \\ \hline
        (0, 1, 0)                    & -4 & (1, 1, 0) & 4  \\ \hline
        (0, 1, 1)                    & -4 & (1, 1, 1) & 4  \\ \hline
      \end{tabular}
    \end{center}
  \end{table}

\end{example}

We will conclude this chapter by discussing alternative notation for QUBO
problems when the problem is defined on a complete graph. In such a case, the
first sum in the equation \eqref{eq:qubo} runs through all the possible pairs
${i, j}$, and thus $F$ can be written as:
\begin{equation}
  \label{eq:qubocomplete}
  F(q_1, \ldots, q_N) =  \sum_{i=1}^N \sum_{j=i+1}^N a_{ij} q_i q_j + \sum_{i=1}^N b_iq_i,
\end{equation}
One can now define a $n \times n$ real symmetric matrix $Q$ with coefficients
\begin{equation}
  Q_{ij} = \begin{cases}
    b_{i}  & i = j   \\
    a_{ij} & i \ne j
  \end{cases}
\end{equation}
Having in mind that squaring a binary variable does not change its value, we can again
rewrite $F$ as
\begin{equation}
  \label{eq:qubomatrix}
  F(q_{i}, \ldots, q_{N}) = \sum_{i \le j} Q_{ij} q_{i}q_{j} = \sum_{j \le i} Q_{ij} q_{i}q_{j}.
\end{equation}
Moreover, since it is always possible to view any given QUBO as a one defined on a complete graph
(by introducing artificial edges with weights 0), the equation \eqref{eq:qubomatrix} provides
a one--to--one correspondence between real symmetric matrices and QUBO problems. We will see
the benefits of representing QUBO models as symmetric matrices when discussing the bruteforce
algorithm in chapter \ref{chapter:bruteforce}

%%% Local Variables:
%%% mode: latex
%%% TeX-master: "../main"
%%% End:

\chapter{Technologies involved}
\label{chapter:near-term}

 In this chapter we discuss technologies used for performing research presented in this work.
 First, we discuss quantum annealing, a quantum computation paradigm witnessing rapid development in recent years. The second discussed technology, Nvidia CUDA, is present on the market for well over a decade, yet each and every iteration of GPUs brings to the table new functionalities and performance improvements. 


\section{Adiabatic quantum computation and quantum annealing}
One of the possible alternatives to the standard gate model is adiabatic quantum computation which relies on adiabatic quantum theorem \cite{born}. Informally, the theorem states that if the physical system starting from its instantenous ground state is driven slowly enough, the evolution follows hamiltonian and the system stays in the instantenous ground state. Adiabatic quantum computing might be realized in a form of quantum annealing. In this metaheuristics, a goal is to minimize value of some target function $f$. In order to do so, one first has to find some Ising hamiltonian $H$ whose ground state encodes the optimal solution to the problem. Then, the system is prepared in a ground state of some initial, simpler hamiltonian $H_0$. After that. the system is slowly perturbed in such a way that its hamiltonian changes from $H_0$ to $H$, i.e. the evolution of the system is governed by the following hamiltonian $H_\text{total}$
\begin{equation}
    \label{eq:aqc}
    {H}_\text{total}(s) = a(s) {H}_0 + b(s){H}, \quad s \in [0, \tau]
\end{equation}
where $a(\tau) = b(0) = 0, \; a(0) = b(\tau) = 1$, and $a$ and $b$ are decreasing and increasing functions respectively.
By adiabatic quantum theorem, if the evolution described by the equation \eqref{eq:aqc} is slow enough, the system will stay in ground state of ${H}_\text{total}(s)$ for every $s \in [0, \tau]$. In particular, after reaching $s=\tau$, i.e. when $H_\text{total} = H$, the state of the system should encode an optimal solution to target optimization problem.


\subsection{D-Wave quantum annealers}

The first commercially available quantum annealer was introduced by D-Wave company in 2011. At the time of writing, the newest series of D-Wave annealers is The Advantage System utilizing chip with at least 5000 qubits. These novel devices deomnstrate a significant improvement over previous 2000Q generation, which used at most 2048 qubit chip with sparser connectivity. Since most research utilizing quantum annealers in this work was conducted using 2000Q devices, in what follows we mostly limit our attention to this specific devices.

\begin{figure}[h]
    \centering
    \includegraphics[width=\textwidth]{figures/chimera.pdf}
    \caption{\textbf{a.} Example of Chimera topology. Graph presented here consists of $2 \times 2$ grid of unit cells (hence $C_2$ name). Note the sparse connectivity of the presented graph as compared to the full graph with the same number of nodes. \textbf{b.} Full $K_9$ graph embedded in Chimera graph from \textbf{a.}. The nodes are labelled with numbers corresponding to groups of logical qubits they were constructed from.}
    \label{fig:chimera}
\end{figure}


\subsection{Annealer topologies}

The Ising model described in previous chapter allows arbitrary interactions between spins. However, this is often not the case with physical devices like quantum annealers. For instance, in D-Wave devices spins are arranged into a specific \emph{topology} with limited connectivity. Thus, the possible graph on which the optimization problems can be defined is much sparser than the complete $K_{N}$ graph. Spins in 2000Q devices are arranged in the so called \emph{Chimera} topology depicted in figure \ref{fig:chimera}. The Chimera graph can be thought of as a $n \times m$ lattice of unit cells, where each cell is a complete bipartite $K_{4,4}$. Qubits are also connected to two qubits in neighbouring cells via external couplers (unless they reside on the border or a corner of the lattice), resulting in at most total 6 connections per qubit. Straightforward calculation shows that Chimera graph has total of $16mn + 4(m-1)n + 4(n-1)m$ edges.

The new Advantage system uses \emph{Pegasus} topology, which increases maximum degree of vertex in the graph to 15. For the details of this topology we refer interested reader to \cite{boothby}.

\subsection{Minor embeddings}
Not every Ising problem one might wish to solve is compatible with topology of physical quantum annealer. This problem can sometimes be circumvented using procedure called \emph{minor embedding}. In this approach one considers groups of several physical spins connected in a chain as a single logical one. Vertices corresponding to spins in each such chain are contracted, resulting in the new graph, having higher connectivity but smaller number of vertices than the original one. With the proper choice of chains this new graph can contain a subgraph isomorphic to the graph of the problem to solve. The minor embedding process is illustrated in figure \ref{fig:chimera}.

Grouping qubits into chains and contracting them is not enough to run the embedded problem. One first has to ensure that all spins in a chain align in the same direction, so that they can indeed represent a single logical variable. This is accomplished by adding a penalty term which prohibitively increases cost of solutions in which value of spins in chains differ. However, due to the probabilitsic nature of quantum annealer, even after adding penalty one might obtain solutions in which values of spins in some chains disagree. Such disagreeing chains are called \emph{broken}. Since solutions containing broken chains cannot be interpreted directly as a solution to the original problem, one has to either discard them, or manually correct them (e.g. by choosing most common value among the spins in chain).

% To define minor embedding more formally, let us first define the notion of vertex contraction. Let $(G, E)$ be a graph defining topology of the device and let $v_{1}, \ldots, v_{{n}} \in E$ be a chain of vertices. Contraction of vertices $v_{1}, \ldots, v_{n}$ is a new graph $G'=(V', E')$ such that
% \begin{itemize}
%   \item $V' = V \setminus \{v_{1}, \ldots, v_{{n}}\} \cup \{w\}$, where $w$ is some new vertex.
%   \item $E' = \{e \in E\colon v_{i} \notin E\} \cup \{\{v, w\}\colon \exists_{i} \{v, v_{i}\} \in E\}$
% \end{itemize}
% Intuitively, contracting vertices replaces them with a new one connected with their every neighbour.

\section{Nvidia CUDA}
History of specialized hardware for manipulating graphics ranges as far as to 1970s. Initially these devices, that later became known as Graphics Processing Units (GPU), offered limited functionalities. Increasing demand for performance in gaming industry and professional graphics processing drove the evolution of GPUs, which quickly became highly sophisticated devices supporting advanced 2D and 3D image manipulation. Performing such arithmetically intensive operations requires enormous computational power, and it was only the matter of time until it was realised that the power of these devices can be harnessed for general purpose computations (so called GPGPU - General Purpose computing on GPU). 

Early efforts in development of GPGPU required framing of computational problem in terms of operations performed on graphical primitives, as this was the only way for using specialized API of GPUs. This changed with the development of devices and toolkits that supported operations needed for general purpose computations out of the box. Notably, in 2007 Nvidia introduced its massively parallel CUDA architecture.

\subsection{Differences between CPU and GPU}
Principles behind operation of CUDA-enabled devices are fundamentally different than the ones governing execution of program on traditional CPU-only architecture. In current x86 based computers, CPU runs given sequence of instructions (so called thread of execution) using one of its cores. Such processor is the ''brain'' of a computer, and it can perform a wide variety of tasks ranging from arithmetic operations, through accessing system's RAM, to performing IO operations and controlling other components of the system. Typical CPUs are optimized for sequential execution, and as such are usually equipped with moderate (as compared to the GPUs) number of high-performance cores. 

On the other hand, GPUs are more specialized. They are well suited for performing a large number of arithmetic operations and accessing memory in parallel. They typically have more cores than a traditional CPU (with even modern commodity GPUs boasting thousands of them). Although those cores are less performant than their CPU counterparts and support much narrower set of operations, their large number combined with fast memory access gives modern GPUs advantage over CPUs in multiple areas.

\subsection{Processing flow on CUDA}
Considering the architectural differences between CPUs and GPUs, it is hardly surprising that both of these types of devices are programmed quite differently. The first major difference is that GPUs cannot operate on their own and are themselves controlled by CPU. This is why CUDA is a type of \emph{heterogenous} architectures (as opposed to CPU-only \emph{homogenous} architectures). The processing flow on CUDA is summarized in Figure \ref{fig:cuda_flow}.

\begin{figure}[ht]
    \centering
    \includegraphics[width=0.7\textwidth]{figures/CUDA_processing_flow_(En).png}
    \caption{{\protect\todo[inline]{Important: this figure is only here temporarily, I will prepare my own graph once the text is done.}} Processing flow on CUDA. The CPU sends input data to GPU memory, and launches computational kernel. The kernel's code is executed, in parallel, using GPU. Once the execution is done, results are copied form GPU memory to system's RAM.}
    \label{fig:cuda_flow}
\end{figure}

Programs run on GPU are organized in \emph{kernels}. For the most part kernels might be viewed as functions or subroutines (which is indeed how they are implemented) that don't have a return value. On a CPU, such function would be executed by some core as a part of a thread. In CUDA however, the very same kernel is executed by multiple threads. Executing a kernel requires specifying a \emph{grid} that will be used for running it. A grid can be 1, 2 or 3 dimensional and is itself divided into blocks. Each block is in turn also organized in 1, 2, or 3 dimensional structure of threads (the same for every block in the grid). Schematic view of a two-dimensional grid is presented on Figure \ref{fig:cuda_grid}.

\begin{figure}[ht]
    \centering
    \includegraphics[width=0.8\textwidth]{figures/grid-of-thread-blocks.png}
    \caption{{\protect \todo[inline]{This is also a temporary figure...}} A schematic view of an example two-dimensional CUDA grid. Presented here is a 2 $\times$ 3 grid of 3 $\times$ 4 blocks.}
    \label{fig:cuda_grid}
\end{figure}

As already mentioned, each thread in the grid executes \emph{precisely the same} kernel. It might therefore seem surprising that nevertheless they are able to access different parts of memory. This is possible, because each thread is identified by its indices in both the grid and the block. Those indices can be used e.g. for computing offsets in arrays that are being processed. A more sophisticated use of thread and block indices will be exemplified in chapter \ref{chapter:bruteforce}.

\subsection{Programming environment}
CUDA devices can be programmed directly using either C/C++ or Fortran. The C/C++ code can be compiled using Nvidia's nvcc compiler, shipped out of the box with CUDA toolkit, while using Fortran requires installing its third-party distribution (e.g. PG Fortran developed by Portland Group). Giving a comprehensive walkthrough of using either C/C++ or Fortran with CUDA is well beyond the scope of this thesis, but for the sake of completeness we present a short example of CUDA Fortran code in listing \ref{lst:cuda_fortran} \todo{Note to self: remember to either replace it with my own example or cite the source (CUDA Fortran documentatoin)}. 

\begin{lstlisting}[
    language=Fortran,
    label=lst:cuda_fortran,
    captionpos=b,
    caption={Example code in CUDA Fortran. Presented here are an example kernel performing saxpy operation together with a host subroutine that uses it.},
    frame=tlrb,
    aboveskip=2em,
    belowskip=2em
]{fortran}
! Kernel definition
attributes(global) subroutine ksaxpy( n, a, x, y )
   real, dimension(*) :: x,y
   real, value :: a
   integer, value :: n, i
   i = (blockidx%x-1) * blockdim%x + threadidx%x
   if( i <= n ) y(i) = a * x(i) + y(i)
end subroutine

! Host subroutine
subroutine solve( n, a, x, y )
   real, device, dimension(*) :: x, y
   real :: a
   integer :: n
   ! call the kernel
   call ksaxpy<<<n/64, 64>>>( n, a, x, y )
end subroutine
\end{lstlisting}

Along the nvcc compiler, the CUDA toolkit contains several, more specialized libraries. Among others, those include:
\begin{itemize}
    \item cuBLAS -- CUDA Basic Linear Algebra Subroutines library,
    \item cuFFT -- CUDA Fast Fourier Transform library,
    \item cuRAND -- CUDA Random Number Generation library,
    \item cuSPARSE -- CUDA library for manipulating sparse matrices.
\end{itemize}
For many high-level languages, there exist third-party wrappers enabling use of CUDA (e.g. PyCuda for Python).



%%% Local Variables:
%%% mode: latex
%%% TeX-master: "../main"
%%% End:

\chapter{Simulating dynamics of quantum systems using quantum annealing}
\chaptermark{Simulating dynamics}
\label{chapter:simulating}

One of the leading motivations behind the development of quantum computing
devices is simulating quantum systems intractable by classical computers. But
how far are we from this goal? To answer this question, one might design an
algorithm for conducting such a simulation of a physical system and then test
how it performs on the current generation of quantum computers. In this
chapter, we follow this idea and present a possible approach for simulating
quantum systems (or any time-dependent dynamical system) that can be used with
annealing devices such as D-Wave quantum annealers and similar devices. To
illustrate the working of our algorithm, we simulate the simplest single-qubit
system and demonstrate that already near-term annealing devices are capable of
capturing its dynamics in a narrow regime of parameters. Furthermore, the class
of physics-inspired problem instances proposed in this chapter
can be valuable in benchmarking other (not necessarily quantum) solvers.

\section{Parallel in time simulation of dynamical systems}
Optimization problems that can be solved using quantum annealers exhibit no
time--dependence. Therefore, simulating any time-dependent phenomena using
those devices requires reformulating the problem as one that is static in
nature. In our case, it is possible by enlarging the Hilbert space of the
system under consideration, so that the states of this larger space encode also
temporal information \cite{feynmanclock}.

Let us start by precisely defining the problem we want to address.
Consider an $L$--dimensional real or complex system, whose state at time $t$ is
described by the vector $\ket{\psi(t)}$ evolving according to a differential
equation of the form:
\begin{equation}
  \label{eq:dynamical-system}
  \frac{\partial \ket{\psi(t)}}{\partial t} = K(t) \ket{\psi(t)}.
\end{equation}

Here, $K$ is the so-called Kamiltonian \cite{goldstein2002classical} and can be
any linear operator acting on $\mathbb{R}^L$ or resp. $\mathbb{C}^L$. Observe
that any isolated quantum system can be described by equation
\eqref{eq:dynamical-system}, as putting $K=-\frac{i}{\hbar}H$, where $H$ is its
Hamiltonian, transforms the equation \eqref{eq:dynamical-system} into
Schr\"{o}dinger equation.

Given an initial state, $\ket{\psi(t_0)}$ the equation
\eqref{eq:dynamical-system} admits a unique solution:
\begin{equation}
  \ket{\psi(t)} \coloneqq U(t, t_0) \ket{\psi(t_0)},
\end{equation}
where operator $U(t, t_0)$ is a propagator transforming the initial state of
the system into its state at time $t$ and is given by:
\begin{equation}
  \label{eq:propagator}
  U(t, t_0) = \mathcal{T} \exp \left( \int_{t_0}^t K(\tau)d\tau \right).
\end{equation}
Here, $\mathcal{T}$ denotes the time-ordering operation \cite{chronological}.
Note that in the case when $K(t)$ commutes with $K(t')$ for every $t' \ne t$,
the time-ordering can be omitted. In particular, this is the case if $K$ is
time-independent.

Given the initial state, we are interested in finding the state of the system
at some time $t > t_0$. Numerical methods for solving this problem usually
start by partitioning the interval $[t_0, t]$ into $N$ distinct time points
$t_0 < t_1 < \ldots < t_{N-1} = t$. Then, the desired state $\ket{\psi(t)}$ can
be computed as:
%
\begin{equation}
  \ket{\psi(t)} = U_{N-1} \cdots U_1 \ket{\psi(t_0)},
\end{equation}
%
where $U_i$ is a shorthand notation for $U(t_i, t_{i-1})$. This is purely a
rearrangement of computations, which by itself gives no benefit over applying
$U(t, t_0)$ directly. However, shortening the interval allows for a more
efficient approximation of propagators, which can be done using a variety of
methods, including Suzuki-Trotter approximation \cite{suzuki}, commutator-free
expansion \cite{commutatorfree} or tensor-networks based approaches
\cite{dmrg}.

This procedure, common to many sequential methods, gives a starting point for a
class of the so-called parallel in--time methods based on the Feynman clock
operator. In these approaches, one starts by suitably enlarging the state space
so that it can encode the temporal data \cite{feynmanclock}. This can be done
by considering a tensor product of a state space with the new Hilbert space
spanned by the orthonormal basis $\{\ket{0}, \ket{1}, \ldots, \ket{N-1}\}$.
Then, the following superposition encodes states of the system in all $N$
moments of time:
%
\begin{equation}
  \ket{\Psi} = \sum_{n=0}^{N-1} \ket{n} \otimes \ket{\psi(t_n)}.
\end{equation}
Consider now the following \emph{clock operator} $\clockop$:
\begin{equation}
  \label{eq:clock2}
  \begin{split}
    \clockop
    =
    \sum_{n=0}^{N-2}
    &\ketbra{n+1}{n+1} \otimes I - \ketbra{n+1}{n} \otimes U_n + \\
    &\ketbra{n}{n} \otimes I - \ketbra{n}{n+1}\otimes U_{n}^{\dagger}.
  \end{split}
\end{equation}
One can see that $\ket{\Psi}$ is a solution (although not unique) to the
eigenequation:
\begin{equation}
  \label{eq:clock-eigenequation}
  \clockop \ket{\mathbf{x}} = 0.
\end{equation}
The non--uniqueness of the solution of \eqref{eq:clock-eigenequation} follows
from the fact that the definition of the clock operator $\clockop$ does not
depend on the initial state. We can fix this problem by adding a \emph{penalty}
term $\clockop_0 = \ketbra{0}{0}\otimes(I-\ketbra{\psi_0}{\psi_0})$ to the
left-hand side. The equation to solve becomes then:
\begin{equation}
  \label{eq:clock-eigenequation2}
  (\clockop + \clockop_0) \ket{\mathbf{x}} = 0.
\end{equation}
If one puts $\coefmatrix=\clockop + \ketbra{0}{0} \otimes I$ and $\ket{\Phi} =
  \ket{0}\otimes \ket{\psi_0}$, the equation \eqref{eq:clock-eigenequation2}
becomes:
\begin{equation}
  \label{eq:gsys}
  \coefmatrix \mathbf{x} = \ket{\Phi}.
\end{equation}
Thus, using an approximation of evolution operators, we constructed a system of
linear equations encoding the solution to the equation
\eqref{eq:dynamical-system} under the given initial condition. At this point,
however, it is not possible to solve it using a quantum annealer yet. To do so,
one first needs to convert this system into an optimization problem with
dichotomous variables, which will be the topic of the next section.
\section{Solving systems of linear equations as an optimization problem}
There is a straightforward way of converting equation \eqref{eq:gsys} into an
optimization problem. One can observe that the solution minimizes the norm
$\left\Vert \coefmatrix\ket{\mathbf{x}} - \ket{\Phi}\right\Vert$. Since the
norm is non-negative, it follows that solving equation \eqref{eq:gsys} is
equivalent to the following optimization problem:
\begin{equation}
  \label{eq:optimize_1}
  \ket{\Psi} = \argmin_{\mathbf{x}} f(\mathbf{x}), \quad f({\mathbf x}) = \left\Vert \coefmatrix\ket{\mathbf{x}} - \ket{\Phi}\right\Vert^2.
\end{equation}
However, $f$ in the equation \eqref{eq:optimize_1} is not the only choice of a
target function. If $\coefmatrix$ is positive-definite, one can consider the
following function $h$ instead:
\begin{equation}
  \label{eq:optimize_2}
  h(\mathbf{x})=\frac{1}{2}\bra{\mathbf{x}}\coefmatrix\ket{\mathbf{x}} -
  \braket{\mathbf{x}}{\Phi}.
\end{equation}
Indeed, one can verify that solution to \eqref{eq:gsys} also minimizes $h$ by
computing its gradient and Hessian:
\begin{equation}
  \nabla h(\mathbf{x}) = \coefmatrix\ket{\mathbf{x}}-\ket{\Phi}, \quad \nabla^2 h({\mathbf
    x})=\coefmatrix>0
\end{equation}
Since Hessian is positive and $\ket{\Psi}$ is the only vector at which $\nabla
  h$ vanishes, it follows that $\ket{\Psi}$ is indeed a global minimum of $h$.

\section{Discretizing variables}
Thus far, we have been working with continuous variables. The next necessary
step before solving optimization problems \eqref{eq:optimize_1} and
\eqref{eq:optimize_2} using annealer is converting them in such a way that all
unknowns are dichotomous. To this end, we will follow a strategy presented in
\cite{fixedpoint,chang}. The idea is to express each of the unknown
coefficients of $\ket{\mathbf{x}} = $[$x_1, \ldots, x_{LN}]^T$ in fixed-point
approximation. While this strategy was originally described for real matrices,
it works for complex matrices as well, since one can employ the natural embedding of $\CC$
into $\RR^{2 \times 2}$, $a+bi \mapsto a\hat{I} + ib\ssigma_{y}$. Henceforth, we assume
that the considered systems are real.

If one assumes (binary) order of magnitude of coefficients of
$\mathbf x$ to be $D$ (i.e. $x_i \in [-2^D, 2^D]$ for each $i$), then it can be
approximated up to $R$ bits of precision using the formula:
\begin{equation}
  \label{eq:fixed}
  x_i \approx 2^D \left(2 \sum_{\alpha=0}^{R-1}2^{-\alpha}q_i^{\alpha} -1\right).
\end{equation}
Here variables $q_i^\alpha$ are consecutive bits of the fixed-points expansion
of $x_i$. Note that approximation of $x_i$ in \eqref{eq:fixed} is a linear
combination of its bits, therefore plugging it into optimization problems
\eqref{eq:optimize_1} and \eqref{eq:optimize_2} yields quadratic unconstrained
optimization problems of the form:
\begin{align}
  \label{eq:qubo_f}
  \argmin_{\mathbf{q}} f(\mathbf{q}) = \argmin_{\mathbf{q}} \sum_{i,\alpha} c_i^{\alpha} q_i^r + \sum_{i,j,\alpha,\beta} d_{ij}^{\alpha\beta} q_i^{\alpha} q_j^{\beta} + f_0, \\
  \label{eq:qubo_h}
  \argmin_{\mathbf{q}} h(\mathbf{q}) = \argmin_{\mathbf{q}} \sum_{i,\alpha} a_i^{\alpha} q_i^r + \sum_{i,j,\alpha,\beta} b_{ij}^{\alpha\beta} q_i^{\alpha} q_j^{\beta} + h_0.
\end{align}
Coefficients in equations \eqref{eq:qubo_f} and \eqref{eq:qubo_h} can be
straightforwardly computed by appropriate substitutions into equations
\eqref{eq:optimize_1} and \eqref{eq:optimize_2}. For brevity, here we present
only the formulas for the equation \eqref{eq:qubo_h}, which reads:
\begin{eqnarray}
  \begin{split}
    b_{ij}^{\alpha\beta} &= \coefmatrix_{ij} 2^{1-\alpha-\beta+2D} \\
    a_i^\alpha &= \left( 2^{-\alpha+D}\coefmatrix_{ii} - 2^D\sum_{j}\coefmatrix_{ij}- \Phi_i\right)2^{1-\alpha+D},
    \\
    h_0 &= 2^D\left( 2^{D-1}\sum_{ij}\coefmatrix_{ij}+\sum_i \Phi_i\right).
  \end{split}
  \label{eq:coeff}
\end{eqnarray}

Our approach requires the order of magnitude $D$ and precision $R$ in equation
\eqref{eq:fixed} to be chosen beforehand. Choosing the right $D$ requires
knowledge of the range in which coefficients lie. If its value is too small,
the approximations will fail to capture the most significant bits of the real
solution. On the other hand, choosing $D$ that is too large will result in
wasting variables for encoding insignificant zeros. Fortunately, for many
systems, a suitable $D$ can be determined. For instance, for qubit and
multi-qubit systems, each $x_i$ is bounded by $\pm 1$ which makes $D=0$ the
optimal choice for this case.

QUBOs in the equations \eqref{eq:qubo_f} and \eqref{eq:qubo_h} are defined on
the graph of size $N \cdot R \cdot L$. The number of edges (i.e. non-zero
quadratic terms) depends on the number of non-zero off-diagonal elements of the
matrix $\coefmatrix$. It is interesting to note that the overall density of the
graph is an increasing function of $R$ (bigger precision requires a denser
graph) while, on the other hand, it tends to decrease with increasing $L$.

We converted the original problem of finding the dynamics of the system into a
binary optimization problem suitable for input to the quantum annealer. In the
next section, we will discuss experiments that we performed using D-Wave
2000Q$_{2.1}$ and D-Wave 2000Q$_5$ machines to test the approach we described.
The results we discuss here were originally reported in \cite{parallelintime}.

\section{Parallel-in-time simulations with quantum annealer}
\sectionmark{Parallel-in-time simulations}
\label{sec:parallel-in-time}
Before discussing the results of our experiments, let us focus first on its
design. To exemplify our approach, we chose to simulate the dynamics of a
two-level system with an initial state $\ket{0}$ and a Hamiltonian $\mathcal{H}$:
\begin{equation}
  \mathcal{H} = \frac{\pi}{2}\ssigma_y,
\end{equation}
where $\ssigma_y$ is a Pauli spin operator in the $y$-direction. This particular
choice of Hamiltonian and initial state makes the system suitable for
implementation on present-day quantum annealers for several reasons. One can
easily see that the evolution of the system is real (as opposed to complex),
which halves the number of needed variables. Secondly, for integral time points
$t_0=0, t_1=1, \ldots$ coefficients of the wave function can be expressed
\emph{exactly} using only two bits of precision per coefficient, which further
reduces the number of variables.

\begin{figure}[!h]
  \centering
  \includegraphics{figures/parallel-dynamics-result.pdf}
  \caption{Results of simulating dynamics of two-level system on D-Wave 2000Q$_{2.1}$
    (left) and low-noise D-Wave 2000Q$_{5}$ (right). \textbf{a.}--\textbf{b.}
    Energy distribution of samples obtained from D-Wave annealers for different
    annealing times $\tau$. Notice a slight shift of distributions towards the
    ground state for the 2000Q$_{5}$ device. \textbf{c.}--\textbf{d.} Rabi
    oscillations of the simulated system. The obtained samples were normed before
    plotting. As can be seen in panel \textbf{d.}, the low-noise device was able to
    faithfully capture oscillations for $\tau=200, 2000$. The annealing time is
    color-coded: $\tau=$ \tikzquad\,\,\, -- 20\textmu{}s, \,\tikzcircle\,\,\,--
    200\textmu{}s, \,\tikzdot\,\,-- 2000\textmu{}s. } \label{fig:energy-hist}
\end{figure}

We simulated the above system using values of $R=2, 3$ and for several numbers
of time points $N$. We used annealing time $\tau$ spanning several orders of
magnitude, namely $\tau=20\mu s$, $200 \mu s$ and $2000 \mu s$. Since the
resulting graphs were dense, we decided to use standard embedding of the
complete graph $K_n$ on Chimera \cite{chimeraclique}. To assess the quality of
solutions obtained from the annealer, we sampled each problem $10^4$ times on
DW-2000Q$_{2.1}$ device as well as its low-noise version, DW-2000Q$_{5}$.
Energy distributions of samples obtained for $N=6$ are depicted in Fig.
\ref{fig:energy-hist}. The same figure also illustrates the dynamics of the
expected value of $\ssigma_z$ for the lowest energy sample obtained for a given
annealing time. Note that to preserve the physical meaning of the decoded
solution, the state vector was normed before plotting.

To put these results into context, we also compare them to the ones obtained
using two purely classical methods: CPLEX optimizer and recently developed
tensor network-based algorithm (which we describe later in Chapter
\ref{chapter:tn}. The results of this comparison are depicted in Fig.
\ref{fig:cplex_tn_dwave}.

\begin{figure}
  \centering
  \includegraphics[width=\textwidth]{figures/fig34_merge.pdf}
  \caption{ \textbf{a.}--\textbf{b.} Performance of the two state-of-the-art heuristic
    algorithms: the CPLEX optimizer (CP) and a tensor networks-based (TN) solver
    (see Chapter \ref{chapter:tn}) in comparison to the D-Wave $2000$Q quantum annealer (DW),
    cf. Fig~\ref{fig:energy-hist}. The graphs on which the problems were defined
    had respectively $|V|=360$ (\textbf{a.}) and $|V|=624$ (\textbf{b.}) vertices. The annealing time was set
    to $\tau=200$\textmu{}s. The numerical precision of the solution vector is
    denoted as $R$. %\\
    \textbf{c.}--\textbf{h.} Degradation of the solution quality resulting from perturbing the
    problem by truncating its coefficients to a given
    numerical precision denoted as $r$. The reference ground state obtained
    with tensor networks (TN) is compared to the  experimental data from the
    D-Wave $2000$Q quantum annealer (DW). This effect, expected to be predominant
    in the current quantum annealing technology, is already visible on
    Fig.~\ref{fig:energy-hist}\textbf{a.}--\textbf{d.} and Fig.~\ref{fig:cplex_tn_dwave}\textbf{a.}--\textbf{b.}.
  }
  \label{fig:cplex_tn_dwave}
\end{figure}

Results depicted in figures \ref{fig:energy-hist} and \ref{fig:cplex_tn_dwave}
show that the DW-2000Q$_{5}$ was able to faithfully capture dynamics of qubit
if the state of the system was encoded using $R=2$ bits of precision per
coefficient when the annealing time $\tau=200$ was used. For larger values of
$N$ and $R$ one can observe that the quality of solution degrades. Both CPLEX
and tensor networks-based solvers outperformed D-Wave annealers in terms of the
quality of solutions. The differences were especially noticeable for problem
instances with larger graph sizes, i.e. ones with higher precision ($R \ge 3,
  N=6)$), or with extra time points ($N > 6, R = 2$). The observed degradation of
the solution quality is consistent with the results obtained in other works,
especially for the problems requiring complete graphs, see e.g.
\cite{Hamerly2019}.

\subsection{Discussion of error sources}

The poor performance of D-Wave annealer is something certainly to be expected
from such early-stage devices. Annealers are prone to errors stemming from
multiple sources \cite{dwavedocs}, and it is hard to judge which of those
sources contributed most to the lackluster performance of a particular problem
instance. One of the possible sources of errors is DAC quantization, which
essentially limits the precision of both the quadratic and linear coefficients
passed to the annealer. As a result, the problem that the annealer physically
solves is slightly different than the problem the programmer intended to solve.

One can see that such quantization errors would mostly affect problems with
coefficients lying in close proximity to one another. Indeed, suppose that DAC
quantization errors limit the precision of the linear coefficients to $d$
decimal digits. Then any two coefficients, say $h_{i}, h_{{j}}$ lying closer to
each other than $d$ digits, i.e. $|h_{i} - h_{j}| < 10^{-d}$, become physically
undistinguishable to the annealer. The issue also affects coefficients that are
further apart, by possibly diminishing their relative differences.

While it is hard to pinpoint which source contributed the most to the errors in
the case of the optimization problems discussed in this chapter, we argue that
in our case the poor performance of the annealer can be largely explained by
DAC quantization. Indeed, looking at the \eqref{eq:coeff} one can immediately
see that the optimization problem can contain coefficients arbitrarily close to
each other as long as a large enough $R$ is chosen. To justify this reasoning,
we studied how the tensor network solver performs when the coefficients of the
problem are perturbed by truncating their coefficients to a predefined
number of digits $r$. The results of this experiment are presented in Fig.
\ref{fig:cplex_tn_dwave}\textbf{c.--h.}. One can immediately observe that the
error patterns resemble the ones obtained from D-Wave, which might suggest that
DAC quantization might indeed be a significant source of errors in our case.
However, we would like to point out, that our analysis is by no means
conclusive, and further analysis of error patterns is still needed.

%%% Local Variables:
%%% mode: latex
%%% TeX-master: "../main"
%%% End:

\chapter{Solving spin-glass problems using tensor networks}
\label{chapter:tn}

Benchmarking quantum annealers requires adequate algorithms for providing
baselines for the obtainable solutions. While there exists a plethora of
general-purpose optimization algorithms, one might hope to achieve better
results by exploiting the topology of the problem's underlying graph and thus
locality therein. In this chapter, we describe a recent, tensor network-based
algorithm \cite{tn} for finding the low-energy spectrum of Ising spin-glasses,
designed for problems defined on Chimera-like quasi-two-dimensional graphs. The
algorithm exploits the sparsity and locality of the Chimera graph by
representing the Boltzmann distribution of spin-glass as a tensor network,
whose approximate contraction can be used for computing marginal probability
distributions. This procedure can then be combined with the well-known branch
and bound algorithm to iteratively select the most promising partial solutions,
finally producing an approximation of the low-energy spectrum.

\section{Exploring the probability space}

In the algorithm we are going to present in this chapter, we perform the search
in the probability space rather than in the energy space. This physics-inspired
approach is closely tied to the quantum computing paradigm. To explain why, let
us begin by replacing classical Ising Hamiltonian $H(s)$ with its quantum
counterpart $\mathcal{H} = H(\boldsymbol{\sigma}^{z})$ (i.e. replacing each
variable $s_{i}$ with a Pauli operator $\sigma_{a}$ acting on the $i$-th spin.
Naturally, there exists a one-to-one correspondence between the eigenstates of
$\mathcal{H}$ and the possible classical configurations. If one now wishes to
find the low-energy spectrum of size $k \ll 2^{N}$, the task is equivalent to
finding the $k$ most probable states according to the Gibbs distribution $\rho
  \sim \exp(-\beta \mathcal{H})$. One way to achieve this is to prepare the
system in a Gibbs state
\begin{equation}
  \ket{\rho} \sim \sum_{\mathbf{s}} \exp(-\beta \mathcal{H}/2)\ket{\mathbf{s}}
\end{equation}
and then perform a measurement. If repeated multiple times, this procedure
would yield the desired low-energy spectrum with high probability.

While the above procedure is useful conceptually, it clearly cannot be directly
used on a classical computer, as it would require preparing a dense vector of
$2^{N}$ elements. Instead, in our algorithm we represent the Gibbs distribution
approximately via a suitable tensor--network. Then, instead of performing a
quantum measurement, we extract the needed information by traversing the
probability tree using the branch-and-bound method. In what follows, we
describe the whole procedure in detail, starting with the branch-and-bound
part.
%\todo[inline,color=SkyBlue]{Mention the other Chimera-specific algorithm}

\section{Branch and bound}
Let us first consider an Ising spin-glass problem defined on a square lattice,
as depicted in Fig. \ref{fig:lattice-and-border}. The state space of such a
system can be viewed as a tree, in which $k$-th level contains all partial
configurations $(s_1, \ldots, s_k)$. This representation allows one to explore
the state space incrementally in search for low energy states, and possibly
prune the less promising branches. In the approach described here, we use
marginal probability $p(s_1, s_2, \ldots, s_k)$ as a criterion for deciding
which partial configurations are most promising. More precisely, we explore the
solution tree in a top-down manner, keeping at most $M$ states at $k$-th level
and branching them into $2M$ new partial configurations at level $k+1$ . The
new marginal probability distributions can be computed as

\begin{equation}
  \label{eq:conditional-prob}
  p(s_1, s_2, \ldots, s_k, s_{k+1}) = p(s_1, s_2, \ldots, s_k)p(s_{k+1}|, s_1, \ldots, s_k)
\end{equation}
We can exploit locality of the problem by observing that conditional
probability in \eqref{eq:conditional-prob} of configuration in the region $X =
  (1, 2, \ldots, k)$ depends only on configuration on the border $\partial X$ comprising those spins that directly interact with the region
$\overline{X} = (k+1, 2, \ldots N)$. To see that, denote by $H_X$ the usual
Hamiltonian $H$ restricted to the graph induced by vertices in $X$. Further,
let $H_{X, \overline{X}} = H - H_X - H_{\overline{X}}$. Notice that $H_{X,
    \overline{X}}$ contains only quadratic terms $J_{ij} s_i s_j$ such that $i \in
  X$ and $j \in \overline{X}$. Slightly abusing the notation, one may thus write
\begin{equation}
  H(s_1, \ldots, s_N) = H_X(s_1, \ldots, s_k) + H_{\overline{X}}(s_{k+1}, \ldots, s_N) + H_{X, \overline{X}}(s_1, \ldots, s_N)
\end{equation}
Using definition of conditional probability applied to Boltzmann distribution,
one thus gets
\begin{align}
  p(s_{k+1}|s_1, \ldots, s_k) & = \frac{\sum\limits_{(z_{k+2}, \ldots, z_N)}e^{-\beta H(s_1, \ldots, s_{k+1}, z_{k+2},\ldots,z_N)}}{\sum\limits_{(z_{k+1}, \ldots, z_N)}e^{-\beta H(s_1, \ldots, s_k, z_{k+1},\ldots,z_N)}}                                                                                                                                                     \\
                              & = \frac{\sum\limits_{(z_{k+2}, \ldots, z_N)}e^{-\beta (H_X(s_1, \ldots, s_k) + H_{\overline{X}}(s_{k+1}, z_{k+2},\ldots,z_N) + H_{X, \overline{X}}(s_1, \ldots, z_N))}}{\sum\limits_{(z_{k+1}, \ldots, z_N)}e^{-\beta (H_X(s_1, \ldots, s_k) + H_{\overline{X}}(z_{k+1}, \ldots,z_N) + H_{X, \overline{X}}(s_1, \ldots, z_N))}}                 \\
                              & = \frac{e^{-\beta H_X(s_1, \ldots, s_k)}\sum\limits_{(z_{k+2}, \ldots, z_N)} e^{-\beta(H_{\overline{X}}(s_{k+1}, z_{k+2},\ldots,z_N) + H_{X, \overline{X}}(s_1, \ldots, z_N))}}{e^{-\beta H_X(s_1, \ldots, s_k)}\sum\limits_{(z_{k+1}, \ldots, z_N)}e^{ -\beta(H_{\overline{X}}(z_{k+1}, \ldots,z_N) + H_{X, \overline{X}}(s_1, \ldots, z_N))}} \\
                              & = \frac{\sum\limits_{(z_{k+2}, \ldots, z_N)} e^{-\beta(H_{\overline{X}}(s_{k+1}, z_{k+2},\ldots,z_N) + H_{X, \overline{X}}(s_1, \ldots, z_N))}}{\sum\limits_{(z_{k+1}, \ldots, z_N)}e^{ -\beta(H_{\overline{X}}(z_{k+1}, \ldots,z_N) + H_{X, \overline{X}}(s_1, \ldots, z_N))}}
\end{align}
Note, in both numerator and denominator spins with indices from $X$ appear
non-trivially only in $H_{X, \overline{X}}$ , i.e. the whole expression depends
only on those spins in $X$ that directly interact with spins in $\overline{X}$,
which was to be demonstrated.

\begin{figure}
  \centering
  \includegraphics[width=\textwidth]{figures/squarelattice.pdf}
  \caption{\textbf{a.} An example Ising spin--glass  of 16 spins on a square lattice. The conditional probability for spins in the region $\overline{X}$ conditioned with a given configuration of spins in the region $X$ depends only on part of the configuration on the border $\partial X$. \textbf{b.} A fragment of state space tree. States kept at each level
    are marked with green, and the pruned branches are marked with red.}
  \label{fig:lattice-and-border}
\end{figure}

Before discussing how probabilities in \eqref{eq:conditional-prob} can be
computed, let us first extend the above approach to the more general case of a
quasi-two-dimensional graph, i.e. one in which nodes can be grouped into
\emph{clusters} forming a two-dimensional square lattice (see Fig.
\ref{fig:clustering}). One can easily see, that again we can construct a
tree-like structure representing state space, this time considering joint
configurations of spins in a single cluster. Therefore, for the most of the
time, we might "forget" the underlying spin-glass structure and consider square
lattices in which spin clusters act like higher-dimensional systems.

\begin{figure}[b]
  \includegraphics[width=\textwidth]{figures/clustering}
  \caption{Grouping spins into clusters in a quasi-two-dimensional graph. Here, spins in
    the original graphs are grouped together to form a square lattice. Each site in
    the new lattice then effectively serves as a higher-dimensional system.}
  \label{fig:clustering}
\end{figure}

% A central object considered in our algorithm is the probability distribution
% $\exp[-\beta H(\mathbf{s})]$, which we approximate using a PEPS-equivalent
% tensor network, whose detail will be described shortly. Contracting such a
% network can give the probability distribution of any full configuration, as
% well as the marginal probabilities.

% \begin{equation}
%   p(s_1, s_2, \ldots, s_k) \sim \tr[\mathcal{P}_{(s_1, \ldots, s_k)} e^{-\beta H(\mathbf{s})}]
% \end{equation}

\section{PEPS network construction}
We begin the construction of a PEPS network for a quasi-two-dimensional graph
by considering two spins at sites $i$ and $j$ connected by an edge $J_{ij}$.
This edge can be decomposed as

\begin{equation}
  e^{-\beta J_{ij}s_i s_j} = \sum_{\gamma = \pm 1} B^{s_{i\phantom{j}}}_\gamma C^{s_j}_\gamma
\end{equation}
where
\begin{equation}
  \label{eq:decomposition}
  B^{S_i}_\gamma = \delta_{\gamma s_i} \quad C^{s_j}_\gamma = e^{-\beta \gamma J_{ij} s_j}
\end{equation}
Note that decomposition \eqref{eq:decomposition}, although not unique, has the
advantage of comprising only non-negative coefficients, which positively
affects numerical stability. Next, with each cluster we associate a PEPS tensor
\begin{equation}
  \label{eq:peps}
  A^{\mathbf{s_c}}_{\mathbf{lrud}} = e^{-\beta H(\mathbf{s_c})} B^{\mathbf{s_c}^l}_\mathbf{l}C^{\mathbf{s_c}^r}_\mathbf{r}B^{\mathbf{s_c}^u}_\mathbf{u}C^{\mathbf{s_c}^d}_\mathbf{d}
\end{equation}
Here, $\mathbf{s_c}$ collects all spins in a given cluster, and
$\mathbf{s_c}^l$, $\mathbf{s_c}^r$, $\mathbf{s_c}^u$, $\mathbf{s_c}^d$ collect
spins interacting with it from the left, right, up and down respectively. Each
such tensor has five legs: the physical one $\mathbf{s_c}$ of dimension $2^m$,
where $m$ denotes the number of spins in the cluster, and the virtual ones $l,
  r, u, d$ with dimensions depending on the number of inter-cluster edges. Note
that $H$ in \eqref{eq:peps} is restricted to the graph induced by spins
belonging to the considered cluster. The construction is depicted in Fig.
\ref{fig:tensors}. Combining the tensors gives an exact representation of the
Gibbs distribution as
\begin{equation}
  \exp(-\beta H(\mathbf{s})) \sim \sum_{\mathbf{k}}\prod_{c^{i}}A_{\mathbf{k}^{i}}^{\mathbf{s}_{c^{i}}}
\end{equation}
Despite our tensor network representation of the Gibbs distribution being
exact, contracting the network to obtain the information is still a difficult
task. In principle, one could use some approximation schemes \cite{lewenstein}.
However, in our approach, we decided to use another procedure exploiting the
locality of the problem graph. Namely, we employ a matrix product state (MPS)
-- matrix product operator (MPO) based approach \cite{murg} approach. One starts by
considering the first row of the lattice as a vector in high dimensional space
having a natural decomposition in the form of MPS. Then, we add another row,
viewed as MPO, which enlarges the MPS representation. Adding subsequent rows
would require an exponential growth of the bond dimension $\chi$. To prevent
this, a sequence of truncation is performed, which results in a series of
boundary MPS. The new MPS are found by minimizing their distance from the
enlarged previous ones. The MPS-MPO construction is depicted in Fig.
\ref{fig:tensors}(e)--(f). In the end, the network can be contracted exactly
resulting in the desired conditional probability.
\begin{figure}
  \includegraphics[width=\textwidth]{figures/peps.pdf}
  \caption{Construction of PEPS network} \label{fig:tensors}
\end{figure}

\section{Benchmarks}

To fully investigate the performance of our algorithm, we performed several
benchmarks, testing various metrics quantifying both execution time, as well as
the quality of the found solutions. We tested our algorithm for sets of
\emph{droplet} instances specifically designed to be hard for classical
heuristic solvers, especially ones relying on local updates. We benchmarked our
algorithm against classical solvers based on Parallel-Tempering, and D-Wave
Quantum annealer DW-2000Q$_6$. As it is hard to directly compare samples
obtained from the D-Wave annealer with the output of our deterministic
algorithm, we decided to use time-to-solution as a metric. The time to solution
$\mbox{TTS}$ is defined as
\begin{equation}
  \label{eq:tts}
  \mbox{TTS} = t \frac{\log(1 - p_{target})}{\log(1 - p_{succ})},
\end{equation}
where $p_{target}$ is the desired probability of obtaining solution, $p_{succ}$
is the empirical probability of obtaining the solution and $t$ is the running
time of the solver. In addition, for D-Wave annealers, we multiply $\mbox{TTS}$
by the ratio $N/\mbox{num\_qubits}$, to account for the possibility of fitting
multiple instances of the problem on the device at the same time. Naturally,
one might consider $\mbox{TTS}$ metric not only for finding a ground state, but
also for finding a solution approximating a ground state with a given
approximation ratio  (i.e. solution lying in the desired lowest
fraction of the full energy spectrum). The results of these benchmarks are
presented in Table \ref{tab:tnvspt}. For all instances, our algorithm was able
to find the ground state, which was not the case for other solvers. However, if
one is not necessarily interested in finding the ground state, both D-Wave
annealers and classical parallel tempering solver might be a better choice, as
they were able to find a satisfying solution in a shorter time.
\begin{table}[b]
  \centering
  \begin{tabular}{|l|c|ccc|}
    \hline
    \rowcolor{theader}  Method & approx. ratio & $N=512$   & $N=1152$  & $N=2048$  \\
    \hline
    TN                         & g.s.          & 30s       & 150s      & 450s      \\
    \hline
    \hline
    PT (adaptive)              & g.s.          & 800s      & ---       & ---       \\
    \hline
    PT (geometric)             & $0.01$        & 0.53s     & 4.16s     & ---       \\
    PT (geometric)             & $0.005$       & 2.51s     & 56.4s     & ---       \\
    PT (geometric)             & $0.001$       & 158.4s    & timed-out & ---       \\
    PT (geometric)             & $0.0001$      & 897.6s    & timed-out & ---       \\
    \hline
    \hline
    DWave 2000Q$_6$            & $0.01$        & $0.003$s  & $0.006$s  & $0.02$s   \\
    DWave 2000Q$_6$            & $0.005$       & $0.2$s    & timed-out & timed-out \\
    DWave 2000Q$_6$            & $0.001$       & timed-out & timed-out & timed-out \\
    \hline
    \hline
  \end{tabular}
  \caption{Comparison of time-to-solution metric for our tensor network-based algorithm,
    in-house Parallel Tempering implementation and D-Wave 2000Q$_{6}$. The
    \emph{adaptive} and \emph{geometric} terms refer to the distribution of inverse
    temperature $\beta$ in Parallel Tempering replicas. We bounded the running time of
    our solver to 30 minutes with bond dimension $\chi = 16$, $\beta=3$ and
    probability cutoff $\delta_{p} = 10^{-3}$. For PT, the $t$ in the equation
    \ref{eq:tts} is inferred from the running time and number of performed MC
    sweeps:  a single MC sweep took 0.00005s for N=512 and 0.00011s
    for $N=1024$. For the adaptive PT, we used 12 replicas. For geometric PT, we
    used 25 replicas with geometrically distributed $\beta$, with
    $\beta_{\min}=0.0001$ and $\beta_{\max}=10$. For all probabilistic samplers, we
    used target probability $p_{\mbox{target}}=0.99$. In the case of D-Wave
    annealers, we modified instances by dropping inactive qubits. To obtain the
    reference ground state, we once again used our algorithm. We optimized time to
    solution over annealing times of $5\mu s$, $20\mu s$ and $200\mu s$. For each instance and each
    annealing time, we gathered 1000 samples for $N=512$ and 2500 for other values
    of $N$. Also, we used $t=\tau$ for the D-Wave annealers, i.e. we considered only
    annealing time and disregarded other factors contributing to overall solution time.
    This choice is justified by the fact that the other contributions are minuscule.
    The ``timed-out'' string indicates that the given algorithm could not
    find a solution within the given approximation ratio (i.e. $p_{\mbox{succ}}=0$). }
  \label{tab:tnvspt}
\end{table}

\begin{figure}
  \includegraphics[width=\textwidth]{figures/tn-single-state.pdf}
  \caption{Example result of running our algorithm on a droplet instance with $N=2048$.
    \textbf{a.} Low energy spectrum found by a single run of our algorithm. Observe
    consistency between different values of $\beta$. \textbf{b.} Hamming distance
    of solutions presented in \textbf{a.} from the ground state. \textbf{c.}
    Probabilities of each configuration found for least numerically stable values
    of $\beta$. In the depicted example, we can see full consistency between the
    probabilities obtained from contracting PEPS network $p_{n}$ and the Boltzmann
    weights calculated from the configuration's energy. \textbf{d.} Comparison of
    largest discarded probability $p_{d}$ and the ground state probability $p_{1}$.
    With increasing $\beta$ we were able to achieve $p_{d} < p_{1}$, indicating
    that the algorithm indeed reached the ground state. }
  \label{fig:tn-single-state}
\end{figure}

\begin{figure}
  \includegraphics[width=\textwidth]{figures/tn-ground-degeneracy.pdf}
  \caption{Histogram of ground state degeneracy found by our algorithm fore test instances
    constructed by drawing couplings $J_{ij}$ uniformly from a set $\{\pm 1, \pm 2,
        \pm 4\}$ and setting all local fields $h_{i} = 0$.}
  \label{fig:ground-degeneracy}
\end{figure}
%%% Local Variables:
%%% mode: latex
%%% TeX-master: "../main"
%%% End:

%\chaptermark{}ter{Finding low--energy spectrum of spin glass using CUDA}
\chapter[Brute--forcing spin--glass problems]{Brute--forcing spin--glass problems with CUDA}
\label{chapter:bruteforce}
In chapter \ref{chapter:tn} we presented a tensor network--based heuristic algorithm specifically
tailored for Ising spin--glass problems defined on Chimera graphs. In stark contrast, in this
chapter we will shift our attention to a deterministic algorithm capable of solving problems defined
on arbitrary (but relatively small) graphs.

Conceptually, the simplest approach for solving any optimization problem is a
brute force approach, i.e. an exhaustive search through the set of all possible
solutions. For the QUBO or Ising spin--glass with $N$ variables, this would
require iterating over $2^{N}$ possible states and computing energy for each of
them, resulting in a superexponential algorithm. Although the approach is
clearly infeasible for large problems, it presents several advantages. The
algorithm is deterministic and can certify\footnote{i.e., prove that the found
  solution is in fact optimal} the solution. Moreover, it can be used to compute
a low energy spectrum of arbitrary size $k$ (provided that it can fit into
memory). Lastly, it is trivially parallelizable and hence can be efficiently
accelerated using virtually any parallel computing paradigm, thus significantly
increasing attainable problem sizes.

In this chapter, we discuss such a brute--force algorithm using massively
parallel CUDA architecture. We start by outlining the basic version of the
algorithm and then discuss its recent optimizations for cases when the goal is
to find only the ground state (as opposed to finding a low energy spectrum).
Our implementation is capable of finding the ground state of instances of size
$N=50$ in an hour using a commodity GPU and achieving the same task in less
than 5 minutes on a server--grade NVIDIA DGX H100. Lastly, we present a
possible application of our algorithm, which is validating a recent MPS--based
algorithm for solving Ising spin--glasses.

% Optimization problems play an important role in modern society, especially in current volatile times. Instances of such problems can be found in numerous areas of industry and applied sciences. One could mention logistic issues, such as vehicle routing problem together with its variants, and the famous protein folding problem or job shop scheduling to name just a few. It is often the case where many of the aforementioned problems fall into the so called NP--hard complexity class. This fact alone renders hard to solve, making the entire operational research a challenging endeavour requiring enormous amount of computational resources.

% One way to tackle optimization problems is an exhaustive search over all possible solutions. Unfortunately, such brute--force approach quickly becomes impractical, as the number of possible solutions increases exponentially with the problem size. Nevertheless, despite its simplicity and obvious limitations, the brute--force algorithm is often the only approach capable of solving and certifying~\footnote{(i.e., proving that the found solution is in fact optimal)} \textit{arbitrary} problem instances within a given complexity class. For this very reason, efficient brute--force solvers are still considered to be an irreplaceable tool for testing and benchmarking other, often way more sophisticated, algorithms.

\section{Finding low--energy spectrum with CUDA}
\subsection{Outline of the algorithm}
An idealized brute force algorithm for solving QUBO problems running on a
hardware with infinite storage and an infinite number of execution units can be
summarized as follows:
\begin{enumerate}
  \item Launch number of threads equal to the total number of possible states.
  \item Let each thread compute the energy of one of the states.
  \item Extract (e.g. by sorting) the desired number of low--energy states.
\end{enumerate}
Naturally, an attempt to implement such an algorithm on real hardware is doomed
to fail. To exemplify this, consider a problem with $N=40$ variables. Assuming
we use 32-bit floating point numbers, one would need an enormous amount of
$2^{40}\cdot 4B = 4398046511104\mbox{B}$, or $4\mbox{TB}$ of working memory to
store the computed energies. For $N=50$, this number grows to $4096\mbox{TB}$.
Clearly, such an amount of needed memory is prohibitively large, and that is
even before we consider some form of storage for system states. Moreover, no
current hardware can execute $2^{40}$ threads in parallel. Fortunately, we can
adapt our algorithm to take into account limited memory and parallelism. To do
so, we introduce the following assumptions:
\begin{enumerate}
  \item We will process the solution space in \emph{chunks} that can fit into GPU
    memory.
  \item Number of states in a chunk can be larger than the total number of threads.
    Should this be the case, the threads will process the chunk using a
    grid--stride loop pattern.
\end{enumerate}
As an added benefit of our assumptions, we decouple the grid size from the
problem size. The number of thread blocks and block size become parameters of
our algorithm, which facilitates further fine-tuning of kernel execution
parameters.

The algorithm will keep track of $k$ lowest--energy states computed so far.

This information will be updated after each new chunk is processed. The
downside of this approach is that the size of the low--energy spectrum we can
compute is limited by the chunk size. However, this limitation is not as severe
as it seems, because in a typical scenario, we have $k \ll 2^{N}$.

In the next section, we discuss another important aspect of our algorithm,
which is efficient storage and representation of system states.

\subsection{Storage and representation of system states}
Implementing efficient algorithms involves choosing the right storage strategy
for the data the algorithm operates on. This is especially the case for
present-day GPUs, which are equipped with fairly limited memory, as compared to
the operating memory available to the traditional CPU. Moreover, memory
transfers between host and GPU induce additional overhead that should be
avoided whenever possible. For this reasons one often aims for designing the
storage strategy such that it reuses information already available on the GPU
as much as possible, thus optimizing resource usage and minimizing the number
of memory transfers.

In principle, each configuration of a $N$--variable QUBO can be represented by
$N$ integers. However, since each variable can be assigned only one of two
possible values, this wastes a lot of available memory, as out of each machine
word only a single bit is used. Instead, one can pack the whole state of the
system into a single integer by identifying each bit of the underlying machine
word with a single spin. In our implementation, we decided to use 64-bit
integers. This particular implementation choice limits attainable problem sizes
to $N=64$. However, considering that solving larger problems using the brute
force approach is not likely to be possible in the near future (as demonstrated
by our benchmarks presented further in this chapter), this is not a significant
limitation. Furthermore, should the need arise, one could extend the
implementation to use multiple 64-bit integers for storing a single
configuration.

Identifying states with integers greatly simplifies their enumeration, as it
boils down to iterating over an appropriate range of natural numbers. More
importantly, it allows GPU threads to identify the system state they have to
process using their index and additional offset designating the chunk. In our
implementation, we restrict ourselves to chunk sizes being power of two, i.e.
chunk size $=2^{M}$ for some $M < N$. We conceptually split each configuration
into two parts:
\begin{enumerate}
  \item A \emph{local} part comprising least significant $M$ bits. This is part is
    \emph{different} for each state in the chunk.
  \item A \emph{suffix} comprising the most significant $N-M$ bits. This part is
    \emph{the same} for each state in the chunk.
\end{enumerate}
Now, since there are $2^{N-M}$ chunks, we can identify each chunk with a $N-M$
bit number. Finally we arrive for a formula for an integer representation
$\mathbf{q}_{i}^{j}$ of an $i$-th configuration in $j$-th chunk:
\begin{equation}
  \mathbf{q}_{i}^{j} = i + 2^{M}\cdot j,\qquad i=0,\ldots,2^{M}-1, \; j=0,\ldots,2^{N-M}-1
\end{equation}
The following example demonstrates the representation described above in
detail.
\begin{example}[Processing solution space in chunks]
  Consider QUBO with $N=8$ variables. We decide to use $M=5$. Hence, there are
  $2^{M}=32$ states in each chunk and a total of $2^{N-M}=8$ chunks. The
  \emph{local} part of the first state in each chunk is $0$, or $(00000000)_{2}$
  in binary. The local part of the last state in each chunk is $31$, or
  $(00011111)_{2}$. The table \ref{tab:chunks} below enumerates ranges of
  combined integer representation of states in each chunk.
  \begin{table}[ht!]
    \centering
    \begin{tabular}{|c|c|c|c|c|c|}
      \hline
      \rowcolor{theader}
      \multicolumn{2}{|c|}{Chunk}       &
      \multicolumn{2}{c|}{First state} &
      \multicolumn{2}{c|}{Last state}                                                                         \\
      \hline
      \rowcolor{tsubheader} Index                            &  Binary    & Decimal & Binary           & Decimal & Binary           \\
      \hline
      0                                & $(000)_2$ & 0       & $(00000000)_{2}$ & 31      & $(00011111)_{2}$ \\
      1                                & $(001)_2$ & 32      & $(00100000)_{2}$ & 63      & $(00111111)_{2}$ \\
      2                                & $(010)_2$ & 64      & $(01000000)_{2}$ & 95      & $(01011111)_{2}$ \\
      3                                & $(011)_2$ & 96      & $(01100000)_{2}$ & 127     & $(01111111)_{2}$ \\
      4                                & $(100)_2$ & 128     & $(10000000)_{2}$ & 159     & $(10011111)_{2}$ \\
      5                                & $(101)_2$ & 160     & $(10100000)_{2}$ & 191     & $(10111111)_{2}$ \\
      6                                & $(110)_2$ & 192     & $(11000000)_{2}$ & 223     & $(11011111)_{2}$ \\
      7                                & $(111)_2$ & 224     & $(11100000)_{2}$ & 255     & $(11111111)_{2}$ \\
      \hline
    \end{tabular}
    \caption{An example enumeration of chunks iterated over by brute force algorithm.}
    \label{tab:chunks}
  \end{table}

  \begin{table}{}

  \end{table}
\end{example}

\subsection{Performance benchmarks}

\subsection{Implementation details}

In our approach we decided to store states and their corresponding energies in
arrays of size $k+2^{M}$, where $k$ is the desired size of the low energy
spectrum and $2^{M}$ is the chunk size. The arrays are always synchronized,
i.e. at all times $i$-th state corresponds to $i$-th energy. The first $k$
elements store the lowest energies and corresponding configurations found so
far. When a new chunk is being processed, the second part of the arrays is
populated with new states and energies by the energy--computing kernel. Next,
the best $k$ states from the current chunk are selected and moved into indices
$k, k+1,\ldots,2k-1$. In this way, the global best solutions from previous
chunks and the lowest energy states from the current chunk in a continuous
space in memory, which facilitates updating the best configurations.

One could use a parallel sorting procedure for extracting the $k$
lowest--energy states at each step. However, for improved performance, we
decided to use a combination of the \texttt{bucketSelect} \cite{bucketselect}
algorithm in tandem with \texttt{thrust::partition\_by\_key} \cite{thrust}. The
\texttt{bucketSelect} algorithm is used to find the pivot configuration that
would reside at $k$--th position in the sorted array. Then,
\texttt{thrust::partition\_by\_key} is used to reorder both arrays such that
the configurations with energies lower than the one of the pivot are moved to
the beginning. The same procedure is used both for extracting the $k$-lowest
energy states in the given chunk, as well as to update the global solution by
extracting $k$-lowest energy configurations from the first $2k$ configurations.
The whole procedure is depicted in Fig. \ref{fig:bruteforce}.

Lastly, we would like to note that the algorithm we just described can also be
implemented on homogenous, CPU--only architectures using any of the available
parallelization approaches. In our implementation, we used the
OpenMP\cite{openmp} for a CPU--only version.

\begin{figure}
  \includegraphics[width=\textwidth]{figures/bruteforce}
  \caption{
    Detailed representation of brute force algorithm for finding $k$-lowest energy
    states of a QUBO. The algorithm iterates over the set of all possbile states in
    chunks of size $2^{M}$, where $M$ is a user--defined parameter. Throughout the
    algorithm execution, we maintain arrays of states and corresponding energies.
    The first part of those arrays stores the $k$ best configurations encountered
    so far, and the second part stores configurations belonging to the currently
    processed chunk. In the first phase of the iteration, an energy--computing
    kernel is launched. Then, the $k$--lowest energy configurations from the given
    chunk are selected and moved towards the part of the array with the current
    best solutions. Finally, the best $k$ states are selected from the first $2k$
    configurations and the algorithm proceeds to the next chunk, or terminates if
    all the chunks have been iterated over. } \label{fig:bruteforce}
\end{figure}

\subsection{Performance benchmarks}
In order to test the performance of our algorithm, we run extensive benchmarks
using the following hardware:
%
\begin{itemize}
  \item CPU:
    \href{https://ark.intel.com/products/94456/Intel-Core-i7-6950X-Processor-Extreme-Edition-25M-Cache-up-to-3-50-GHz-}{$10$
      Cores {\rmfamily Intel\textregistered} Core \textsuperscript{TM}i7-6950X};
    %
  \item GPU(1):
    \href{https://www.nvidia.com/en-us/geforce/products/10series/geforce-gtx-1080}{Nvidia
      GeForce GTX $1080$, $8$GB GDDR$5$ global memory, $2560$ CUDA Cores};
    %
  \item  GPU(2): \href{https://www.nvidia.com/en-us/titan/titan-v/}{Nvidia Titan V,
      $12$GB HBM$2$ global memory, $5120$ CUDA Cores}.
\end{itemize}

For conducting our benchmarks we generated $100$ spinglass instances for each
$N=24, 26, \ldots, 30, 32$. Additionally we generated $100$ instances of size
$N=40$ and single instances of sizes $N=48, 50$ that were feasible to solve
with Titan V GPU. Coefficients of each spinglass were drawn randomly from
uniform distributions on the intervals $[-2, 2]$ and $[-1, 1]$ for magnetic
fields and couplings respectively. For each instance, we computed low energy
spectrum of $k=100$ states with our algorithm. We used maximum chunk size of
$2^{29}$ for Titan V and CPU and chunk size of $2^{27}$ for GTX 1080.

As already mentioned, larger instances $(N > 32)$ were solved only using Titan
V GPU. For GTX 1080 and CPU implementation, the expected time to solve those
instances was estimated based on the timings for smaller $N$. The results of
our benchmarks are are presented in Fig. \ref{fig:benchmark_results}.

\begin{figure}
  \centering
  \includegraphics[width=\textwidth]{figures/resultsplot_reduced.pdf}
  \caption{Results of benchmarks of our algorithm. {\textbf{a.}} Time to solution vs.
    system size $N$. {\textbf{b.}} Speedup of multi-core/GPU implementation with
    respect to a single core one vs. system size $N$. The solid lines represent the
    numerical results and the dashed lines present estimates based on results
    obtained for smaller system sizes.} \label{fig:benchmark_results}
\end{figure}

\section{Improving the algorithm using Gray Code}

The algorithm presented in the previous chapter and its GPU--enabled
implementation are already highly performant. However, we can still improve
upon it by altering the order in which we enumerate the integral representation
of states used by our algorithm.

\subsection{Single bit--energy difference}
Suppose we are given a QUBO with $F(q_{1},\ldots,q_{N})$ as in the equation
\ref{eq:qubo}. Consider two states, say $\bq{1} = (\q{1}_{1},\ldots,\q{1}_{N})$
and $\bq{2}=(\q{2}_{1},\ldots,\q{2}_{N})$ such that they only differ in the
$k$-th bit, i.e. $\q{2}_{k}=1-\q{1}_{k}$ and $\q{2}_{i}=\q{1}_{i}$ for $i \ne
  k$. The energy difference $F(\bq{2})-F(\bq{1})$ can be easily computed and the
formula reads
\begin{align}
  \begin{split}
    \label{eq:energydiff1}
    F(\bq{2})-F(\bq{1}) &= b_{k}(\q{2}_{k}-\q{1}_{k}) + \sum_{i\ne k}a_{ik}\q{1}_{i}(\q{2}_{k} - \q{1}_{k}) \\
    &= (\q{2}_{k}-\q{1}_{k}) \left(b_{1} + \sum_{i \ne k}a_{ik}\q{1}_{i}\right) \\
    & = (1-2\q{1}_{k}) \left(b_{k} + \sum_{i \ne k}a_{ik}\q{1}_{i}\right).
  \end{split}
\end{align}
Interestingly, computing the difference in equation \eqref{eq:energydiff1}
requires only $N$ multiplications. But how can this be used to improve the
performance of the exhaustive search through QUBO state space?

Moving $F(\bq{1})$ to the right-hand side, we obtain a formula for $F(\bq{2})$,
which allows for computing it with only $N+1$ instead of maximum of $N(N+1)/2$
multiplications, provided that $F(\bq{1})$ is known. Remember that this is only
possible because $\bq{1}$ and $\bq{2}$ differ only by a single bit. If we could
enumerate states in such a fashion that every consecutive two states differ
only by a single bit, we could leverage the above formula instead of
recomputing energy for each state from scratch. Before we describe how the
procedure works and how to implement this on GPU, let us first introduce the
necessary notation. Given a state $\mathbf{q} = (q_{1},\ldots,q_{N})$, by
$\flip{\mathbf{q}}{k}$ we will denote a state resulting from flipping $k$-th
bit of $\mathbf{q}$, i.e.
\begin{equation}
  \flip{\mathbf{q}}{k} \coloneq (q_{1}, \ldots, q_{k-1}, 1-q_{k}, q_{k+1}, \ldots, q_{N})
\end{equation}
and by $\diff_{F}(\mathbf{q},k)$ we will denote the difference between the
energies of $\flip{\mathbf{q}}{k}$ and $\mathbf{q}$. Using the equation
\eqref{eq:energydiff1}, we see that the expression for
$\diff_{F}(\mathbf{q},k)$ is
\begin{align}
  \begin{split}
    \diff_{F}(\mathbf{q},k) = F(\flip{\mathbf{q}}{k}) - F(\mathbf{q}).
  \end{split}
\end{align}
The pseudocode for a serial algorithm for solving a QUBO problem using our
observations is outlined in listing \ref{lst:grayserial}. Before we can
implement it on GPU though, we need to answer the following questions:
\begin{lstlisting}[
  language=Python,
  caption={Pseudocode for algorithm solving the QUBO problem using energy differences and bit flips.},
  captionpos=b,
  label=lst:grayserial,
  float,
  floatplacement=H
]
def solve_qubo(F, q):
    q = [0] * N # Start with all bits set to 0
    best_state = current_state = q
    best_energy = current_energy = F(q)

    for i in range(2 ** N - 1):
        k = find_next_bit_to_flip(i)
        current_energy = current_energy + diff(q, k)
        current_state = flip(q, k)
        if current_energy < best_energy:
            best_energy = current_energy
            best_state = current_state
    return best_state, best_energy
\end{lstlisting}
\begin{enumerate}
  \item How to produce a sequence of $2^{N}-1$ bits such that executing them enumerates
   a rll possible states?
  \item How to divide work among CUDA threads?
\end{enumerate}
The answer to the first question is well--known and involves enumerating
integers using the Gray code, which we will describe now.

\subsection{Gray code}
When one talks about a binary encoding of integers, the first thing that comes
to mind is a usual positional base-2 system. This encoding certainly does not
fit our purpose. Indeed, suppose $N=3$ and we are currently processing state
corresponding to number $3$, whose representation in binary is $(011)_{2}$. The
next state, corresponding to $4$ is encoded by the string $(100)_{2}$, which
differs in all three bits.

Instead of using the positional system, we might utilize an encoding called
Gray code, or Reflected Binary Code (RBC), which is primarily used to improve
the robustness of electromechanical switches and in error correction protocols.
In this code, encoding of two successive integers always differs by at most one
bit, which makes it suitable for application in our algorithm.

The conceptual construction of Gray code is straightforward. For Gray Code of
length $1$ we have two binary strings: $0$ and $1$. To obtain all Gray codes
for a given length $N > 1$, we first construct an ordered list of codes of
length $N-1$ and call it $L$. Then, we reverse the list of codes, and call it
$H$, an operation called \emph{reflection}. Finally, we prepend $0$ to all
elements of $L$ and prepend $1$ to all elements of $H$. The concatenation of
$L$ and $H$ forms the $N$--bit Gray Code. The process is illustrated in Fig.
\ref{fig:gray} .One useful consequence of the construction is that the shorter
Gray Codes might be viewed as initial parts of the larger ones prepended with
enough zeros. Thus, statements as ``$n$-th Gray code'' make sense and are
unambiguous.

\begin{figure}
  \includegraphics[width=\textwidth]{figures/gray.pdf}
  \caption{Reflection--based construction of Gray Code. The length of the code is denoted
    by $N$. For $N=1$, the code comprises two binary strings, $0$ and $1$. To
    construct the code of length $N>1$, the code of length $N-1$ and its vertical
    reflection are stacked. Then, the first, unreflected half is prepended with $0$
    while the second, reflected half is prepended with $1$.} \label{fig:gray}
\end{figure}

An important thing to observe is that in our algorithm we need at most two Gray
Code--encoded numbers at the time to determine the bit to be flipped. The
reflection--based construction outlined so far would require precomputing a
large part (if not all) of the encodings at once. Considering the size of the
state space, this is clearly infeasible. However, there exists an explicit
formula for computing $n$-th Gray code, which reads:
\begin{equation}
  \mbox{gray}(n) = n \oplus (n >> 1),
\end{equation}
where $\oplus$ denotes the bitwise xor operation and $>>$ is right bitshift.

To compute which bit differs between consecutive Gray codes, we can xor them,
and then find the position of the only set bit in the resulting integer. One
can easily implement a function that finds the first set bit in a 64-bit
integer, or use one of the available library or compiler-builtin functions. For
instance, POSIX--compatible C standard libraries include \texttt{ffsll}
function. In CUDA, there is a \texttt{\_\_ffsll} function available. For both
of the above cases, the function counts bits from $1$. Using this convention,
we can write a pseudocode for a function \texttt{find\_bit\_to\_flip} from
listing \ref{lst:grayserial} like in the listing \ref{lst:findbit}

\begin{lstlisting}[
  language=Python,
  caption={Pseudocode for a function generating bit flips for Gray code construction},
  captionpos=b,
  label=lst:findbit,
  float,
  floatplacement=H
]
def find_bit_to_flip(i): # i starts from 0
    return ffs(gray(i) ^ gray(i+1))
\end{lstlisting}

Now that we know how to construct a correct sequence of bit flips, it is time
we design parallelization strategy, which is what we will do next.

\subsection{Parallelization using GPU}
The algorithm presented in \ref{lst:gray} is fully serial. Our task is now to
parallelize it so that it can be executed on GPU. Unsurprisingly, we will once
again employ the strategy of dividing each state into a suffix and a prefix
part. This time, however, it is the suffix that will stay fixed between
iterations. The prefix part will be updated in each iteration by flipping a
single bit in Gray code order. The process is illustrated in Fig.
\ref{fig:grayparallel}.

\begin{figure}
  \includegraphics[width=\textwidth]{figures/grayparallel}
  \caption{
    Parallel processing of $N$--variable QUBO configurations in Gray code order. In
    our example, suffix length $M=2$, and hence $2^{M}=4$ states are processed in
    each iteration. Consequently, there are $2^{N-M}=8$ iterations. For this
    example, we consider a kernel that computes two iterations per kernel
    invocation, resulting in $2^{N-M}/2 = 4$ kernel invocations total. }
  \label{fig:grayparallel}
\end{figure}

Throughout the execution of the algorithm, we maintain four arrays of size
$2^{M}$. In each array, the $i$-th item always corresponds to the $i$-th
suffix. The \texttt{best\_states} and \texttt{best\_energies} arrays store the
best states found so far amongst states with $i$-th suffix. The
\texttt{current\_states} and \texttt{current\_energies} store configuration and
corresponding energy of current state being processed for $i$-th suffix. Each
iteration starts by determining the index of the next bit to be flipped. This
value is the same for all suffixes. Next, the algorithm computes energy
difference using the equation \ref{eq:energydiff1} and updates the
corresponding energy accordingly. After all $2^{N-M}$ iterations, the
\texttt{best\_states} and \texttt{best\_energies} arrays are used to extract
the ground state.

It is crucial to note that since we are only interested in finding the ground
state, we can group several iterations in one kernel invocation. In fact, it is
entirely possible to implement a kernel that runs all the iterations, which
would avoid kernel launch overhead. Moreover, such kernel could use
thread--local variables to store current state and energy instead of using
global arrays, which would further increase performance. However, as we will
see further in this chapter, we will propose further optimizations that would
require us to split the algorithm into several kernel invocations.

\subsection{Further optimizing parallel execution}

There are two optimizations we can make to further reduce the number of
operations performed in each iteration. Let us first notice, that the only bit
flips that can happen, do so in the prefix part. Going back to the equation
\eqref{eq:energydiff1}, we can rewrite the expression for $F(\bq{2})-F(\bq{1})$
into a sum of two parts:
\begin{align}
  F(\bq{2})-F(\bq{1}) = & (1-2\q{1}_{k}) \left(b_{k} + \sum_{i=0,i\ne k}^{N-M-1}a_{ik}\q{1}_{i}\right) + \label{eq:energydiff2} \\
                        & (1-2\q{1}_{k}) \left(b_{k} + \sum_{i=N-M}^{N-1}a_{ik}\q{1}_{i}\right)\label{eq:energydiff3}
\end{align}
Since the first summand \eqref{eq:energydiff2} is independent from the suffix,
which means that for each of the considered suffixes in any given iteration, it
has the same value. Since the states in the iteration are processed in parallel
by GPU threads, we have to either redo the same computation multiple times, or
use some synchronization mechanism, e.g. compute the prefix in one thread in
each block and then propagate the result to the whole block through shared
memory. However, there is a third approach. For each iteration, we compute the
prefix part of the energy difference using CPU, and then use it as a kernel
parameter. More precisely, we compute $L$ values of the prefix part of the
energy difference and pass it to the kernel as an additional array. Since the
information about which bit to flip is also relevant, we pass the bit sequence
as another array as well.

As for the \eqref{eq:energydiff3} part, observe that for each given prefix
there are only $N-M$ possible values of $k$ (again, that's because the bit
flips happen only in the prefix part, and there are $N-M$ prefix bits).
However, not all values of $k$ are equally common. Examining the Gray code
construction (c.f. Fig. \ref{fig:gray}) reveals that the least significant bit
flips half of times and the second least significant bit flips a quarter of
times. Generally, for $k=0,\ldots,N-M-1$ the $k$-th bit flips constitutes
approximately \footnote{Approximately, because there is an odd number of
  $2^{N-M}$ -1 flips performed, because we do not perform last bit flip which
  would take us back to $(0,\ldots,0)_2$ prefix.} $1/2^{k+1}$ of times.
Therefore, we can cache the value of \eqref{eq:energydiff3} for the $K$ most
commonly--occuring bit flips, where $K$ is a user--controlled parameter.

\begin{figure}
  \includegraphics[width=\textwidth]{figures/bruteforcegray}
  \caption{
    Schematic representation of the GPU--enabled brute force algorithm for finding
    ground state of a QUBO problem using Gray codes.
  }
  \label{fig:bruteforcegray}
\end{figure}

\begin{figure}
  \includegraphics[width=\textwidth]{figures/bruteforcegray2}
  \caption{Implementation of the strided loop for kernel in fig \ref{fig:bruteforcegray}.}
\end{figure}

\subsection{Benchmarks}

\begin{figure}
  \includegraphics[width=\textwidth]{figures/bf_benchmarks_initial}
  \caption{
    Benchmarking results for Gray code--based brute force algorithm for finding a ground state
    of Ising model. The dashed lines between data points are provided for visual guidance. For each
    system size $N$, the solution times were averaged over 20 different instances with known ground
    states. Observe that for setups with multiple GPUs and small system sizes, the solution time
    remains virtually constant. This is because, for small system sizes, the execution time is
    dominated by tasks related to distributing work and gathering results. The parameters used for
    benchmarking in each setup are summarized in table \ref{tab:bruteforce-gs-only-table}.
  }
  \label{fig:bruteforce-gsonly-benchmarks}
\end{figure}

\begin{table}[!ht]
  \tiny
  \begin{tabular}{|l|c|c|c|c|c|}
    \hline
    \multirow{2}{*}{GPU(s)} & \multicolumn{2}{c|}{Kernel launch parameters} & \multicolumn{2}{c|}{Algorithm parameters} & \multirow{2}{*}{\makecell{\# Fixed \\ variables}}\\
    \cline{2-5}
     & Block size & Grid size & Suffix size & \makecell{\# Steps per \\ kernel launch} &\\
    \hline
    A10& 512 & 4096 & 27 & 4096 & N/A\\
    \hline
    A100& 512 & 4096 & 27 & 4096 & N/A\\
    \hline
    A6000 & 1024 & 4096 & 27 & 4096 & N/A\\
    \hline
    V100 x8 & 1024 & 4096 & 27 & 4096 & 3\\
    \hline
    DGX H100 x8 & 1024 & 8192 & 29 & 8192 & 3\\
    \hline
    RTX 4090 x8 & 512 & 4096 & 28 & 4096 & 1\\
    \hline
  \end{tabular}
  \caption{Parameters used for benchmarking}
  \label{tab:bruteforce-gs-only-table}
\end{table}
%%% Local Variables:
%%% mode: latex
%%% TeX-master: "../main"
%%% End:

\chapter{Application to railway dispatching and conflict management}

As a conclusion of the thesis, in this chapter we describe how the results presented so far can be applied in the field of operational research. Namely, we propose an approach to solving railway dispatching problem using quantum annealing. We benchmark implementation of our algorithm for simple problem instances on current generation of D-Wave annealers, using solutions obtained via tensor networks and exhaustive search as a baseline for comparison.


\section{Railway dispatching and conflict management}
In our work, we focused only on the delay management in single-track railways in the presence of railway traffic disturbance. Consider a part of 
\chapter*{Summary}

In this thesis, we focused on benchmarking quantum annealers and validating their
usefulness in practical settings. One of the most anticipated uses of quantum computers
is simulating physics, or, more precisely, simulating the dynamics of quantum systems.
Therefore, it seems there is no better benchmark for a quantum computer than to test
how far it is from achieving this long--awaited goal. To this end, we described in detail a
proof-of-concept algorithm for simulating the dynamics of a quantum (or in fact, any
dynamical) system using quantum annealers. Although the applicability of the algorithm
to current devices is limited by their small number of qubits and sparse connectivity,
our experiments indicated that already the present-day D-Wave annealers can capture
the dynamics of a very simple two-level system. We also contrasted the obtained results
with the ones produced by several classical solvers, concluding that they perform better
than the tested quantum devices. We also provided a possible explanation why the
particular optimization problems solved in our experiments are particularly hard for
D-Wave devices and checked our predictions with numerical experiments.

To assist in the process of benchmarking the annealers, we developed two distinctive
algorithms. The recent, tensor network-based algorithm allows one to solve Ising
spin--glass instances defined on Chimera graph and other similar layouts. The algorithm
is useful in itself as an optimization approach, but for other research conducted for
this thesis, provided a classical baseline for the results obtained by the D-Wave
annealer. The second of our algorithms, a massively parallelizable distributed brute-force
algorithm, allows for the exact computation of the low--energy spectrum of small,
but otherwise arbitrary, spin-glass instances. Importantly, this simple yet
efficient algorithm is exact and deterministic. We used the brute-force algorithm
to obtain low-energy spectra for some of the smaller instances used throughout
our experiments. This provided us not only with a means of assessing the quality of
solutions obtained from other solvers or annealers but also with valuable
insights into the structure of the solution space.

To benchmark another anticipated use of quantum annealers, i.e. solving hard
optimization problems stemming from real-life problems, we described an approach
for solving railway-dispatching problems by converting them to QUBO. We then
conducted experiments testing our approach on two generations of D-Wave quantum
annealers. Remarkably, for our tests, we used real railway timetables from two Polish
railway segments. In our experiments, D-Wave annealers were able to successfully
find an optimal solution to the small problem instances, although the performance
varied greatly depending on the parameters such as the annealing time and chain strength.

%%% Local Variables:
%%% mode: latex
%%% TeX-master: "../main"
%%% End:



\printbibliography

\backmatter
\appendix
\chapter{Asymptotic notation}

In order to characterize the complexity of algorithms, it is useful to use
asymptotic big-O notation. Consider two functions $f, g \colon \NN \to \RR$. We
say that $f$ is $O(g)$ if and only if there exists a constant $C > 0$ and a
natural number $n_{C}$ such that the inequality $0 \le f(n) \le C\cdot g(n)$
holds for all $n > n_{C}$ \cite{clrs}. It is common to write $f=O(g)$ instead
of ``$f$ is $O(g)$'', slightly abusing the mathematical notation \cite{clrs}.
One should notice that big-O notation does not provide a tight bound. For
instance, we have $n + 1 = O(n)$ (since $n + 1 \le 2 \cdot n$) but also $n+1 =
  O(n^{10})$.

In the context of computational complexity, big-O notation is most commonly
used for expressing upper bound on number of (dominating) operations performed
by an algorithm as a function of its input size $N$. Since the number of
performed operations is roughly proportional to the algorithm's execution time,
it follows that algorithms with better bound can be considered as more
performant. However, care must be taken when applying this reasoning to judge
practical performance. In particular, one should be mindful of the constant
factor $C$ in the definition above, as well as any bottlenecks stemming from
the working of the underlying hardware. As a concrete example, Strassen's
algorithm for multiplying two $N \times N$ matrices requires $O(N^{\alpha})$
multiplications, where $2 < \alpha < 3$, and yet may perform worse than naive
algorithm peforming $N^{3}$ multiplications, even for $N$ of order of several
hundreds \cite{dalberto}.

We conclude this section by mentioning that there exist several other
asymptotic notations. For instance, $\Omega$, describing the asymptotic lower bound
of a function, and $\Theta$ combining big-O and $\Theta$. For more details, we
refer the reader to \cite{clrs}.

\chapter{Conditional probability on square lattice}
\label{sec:probability}
Consider a square lattice, such as the one depicted in Fig. \ref{fig:lattice-and-border}.
Let denote by $H_X$ the usual Hamiltonian $H$ restricted to the graph
induced by vertices in $X$. Further, let $H_{X, \overline{X}} = H - H_X -
  H_{\overline{X}}$. Notice that $H_{X, \overline{X}}$ contains only quadratic
terms $J_{ij} s_i s_j$ such that $i \in X$ and $j \in \overline{X}$. Slightly
abusing the notation, one may thus write
\begin{equation}
  \small
  H(s_1, \ldots, s_N) = H_X(s_1, \ldots, s_k) + H_{\overline{X}}(s_{k+1}, \ldots, s_N) + H_{X, \overline{X}}(s_1, \ldots, s_N)
\end{equation}
Using definition of conditional probability applied to Boltzmann distribution,
one thus gets

\begin{align}
   & p(s_{k+1}|s_1, \ldots, s_k) = \frac{\sum\limits_{(z_{k+2}, \ldots, z_N)}e^{-\beta H(s_1, \ldots, s_{k+1}, z_{k+2},\ldots,z_N)}}{\sum\limits_{(z_{k+1}, \ldots, z_N)}e^{-\beta H(s_1, \ldots, s_k, z_{k+1},\ldots,z_N)}}                                                                                                                         \\
   & = \frac{\sum\limits_{(z_{k+2}, \ldots, z_N)}e^{-\beta (H_X(s_1, \ldots, s_k) + H_{\overline{X}}(s_{k+1}, z_{k+2},\ldots,z_N) + H_{X, \overline{X}}(s_1, \ldots, z_N))}}{\sum\limits_{(z_{k+1}, \ldots, z_N)}e^{-\beta (H_X(s_1, \ldots, s_k) + H_{\overline{X}}(z_{k+1}, \ldots,z_N) + H_{X, \overline{X}}(s_1, \ldots, z_N))}}                 \\
   & = \frac{e^{-\beta H_X(s_1, \ldots, s_k)}\sum\limits_{(z_{k+2}, \ldots, z_N)} e^{-\beta(H_{\overline{X}}(s_{k+1}, z_{k+2},\ldots,z_N) + H_{X, \overline{X}}(s_1, \ldots, z_N))}}{e^{-\beta H_X(s_1, \ldots, s_k)}\sum\limits_{(z_{k+1}, \ldots, z_N)}e^{ -\beta(H_{\overline{X}}(z_{k+1}, \ldots,z_N) + H_{X, \overline{X}}(s_1, \ldots, z_N))}} \\
   & = \frac{\sum\limits_{(z_{k+2}, \ldots, z_N)} e^{-\beta(H_{\overline{X}}(s_{k+1}, z_{k+2},\ldots,z_N) + H_{X, \overline{X}}(s_1, \ldots, z_N))}}{\sum\limits_{(z_{k+1}, \ldots, z_N)}e^{ -\beta(H_{\overline{X}}(z_{k+1}, \ldots,z_N) + H_{X, \overline{X}}(s_1, \ldots, z_N))}}
\end{align}
Note, in both numerator and denominator, spins with indices from $X$ appear
non-trivially only in $H_{X, \overline{X}}$ , i.e. the whole expression depends
only on those spins in $X$ that directly interact with spins in $\overline{X}$,
which was to be shown.

\chapter{Dispatching conditions}
\label{chapter:dispatching}
In the following appendix, we use the notation from chapter \ref{chapter:trains}.
\section{The minimum passing time condition.}
Any train $j$ cannot travel through a block $b \in \Bj$ faster than the corresponding minimum
passing time:
\begin{equation}
  \label{eq:dc1}
  \tout(j, b) \ge \tin(j, b) + \pmin(j, b).
\end{equation}
Using \eqref{eq:djs} and \eqref{eq:pt} one can easily verify that inequality
\eqref{eq:dc1} is equivalent to the following inequality for station blocks:
\begin{equation}
  \label{eq:passingtime}
  d(j, s_{j,k+1}) \ge d(j, s_{j,k}) - \sum_{b}\alpha(j, b),
\end{equation}
where the sum runs over all blocks starting form the one succeeding $s_{j,k}$
and ending in $s_{j,k+1}$.
In binary variables, it means that if, for a fixed $j,s,m$, the $x_{j,s,m}=1$,
then delays $d(j,s)$ smaller than $m-\sum_{b}\alpha(j, b)$ are prohibited
and thus the corresponding variables have to zero out. Hence, we arrive at the
following condition:
\begin{equation}
  \label{eq:qubo:passingtime}
  \forall_{j} \forall_{s \in S_j \setminus \{s_{{j,\eend}}\}}
  \sum_{m \in A_{j,s}}
  \left(
  \sum_{ m' \in D(m) \cap A_{j, s_{j,k+1}}} x_{j, s, m}
  x_{j, s_{j,k+1}, m'} \right) = 0,
\end{equation}
where $D(m) = \{0, 1, \ldots, m - \sum_{b}\alpha(j, b) -1\}$.
\section{The single block occupation condition.}
Two trains cannot occupy the same line block. Consider two
trains, $j, j' \in \JJ_0$ leaving the same station $s_{j,k} \in \Sj$ in the direction of the
next station block $s_{j,k+1}$. Suppose further that the train $j$ leaves
first. i.e. $\tout(j', s) > \tout(j, s)$. Since two trains cannot occupy the
same block, some headway time has to pass after $\tout(j, s)$ before the
train $j'$ can leave. This headway is dependent on both $j$ and a
sequence of blocks, and hence we denote it by $\tauu(j, s_{j,k})$. Thus,
the condition becomes:
\begin{equation}
  \label{eq:single-block}
  \tout(j', s_{j,k}) \ge \tout(j, s_{j,k}) + \tauu(j, s_{j,k}).
\end{equation}
Substituting for $\tout$ in \eqref{eq:single-block} yields the following
inequality for delays:
\begin{equation}
\begin{split}
  \label{eq:single-block-delays}
  d(j', s_{j,k}) &\ge d(j, s_{j,k}) + \ttout(j, s_{j,k}) - \ttout(j', s_{j,k}) + \\
  &+\tauu(j, s_{j,k})
\end{split}
\end{equation}
or,
\begin{equation}
  d(j', s_{j,k}) \ge d(j, s_{j,k}) + \Delta(j, j',s_{j,k}) + \tauu(j, s_{j,k})
\end{equation}
where
\begin{equation}
  \label{eq:delta}
  \Delta(j, j', s_{j,k}) = \ttout(j, s) - \ttout(j', s)
\end{equation}
The precise form of the headway $\tauu$ depends on the dispatching detail of the problem.
In our approach, we propose the following form:
\begin{equation}
  \tauu(j, s_{j,k}) = \max_{b}\{\pt(j,b)\}
\end{equation}
where the maximum is taken over all blocks between stations $s_{j,k}$ and $s_{j,k+1}$.
For our decision variables, we use a similar scheme as with the previous
constraint and the condition becomes:
\begin{equation}
  \label{eq:qubo:singleblock}
  \forall_{i=0,1} \forall_{j, j' \in \JJ^{i}} \forall_{s \in S^{*}_{j} \cap S^{*}_{j'}} \sum_{m \in A_{j, s}} \left(
  \sum_{m' \in B(m) \cap A_{j', s}} x_{j,s,m}x_{j',s,m'}
  \right) = 0,
\end{equation}
where, as previously, $\Sjs = \Sj \setminus \{s_{j,\eend}\}$, and $B(m) = \{m + \Delta(j, j', s), m + \Delta(j, j', s)+ 1,\ldots, m +
  \Delta(j, j', s) + \tau_{(1)}(j,s)-1 \}$
is a set of delays violating condition \eqref{eq:single-block-delays}.


\section{The deadlock condition}
The deadlock condition is analogous to the single block occupation condition
but for trains going in opposite directions. Suppose trains $j$ and $j'$ are
heading in opposite directions on a route determined by two consecutive
stations $s_{j,k}$ and $s_{j,k+1}$. Note that for $j'$ the order is reversed, i.e. it
starts at $s_{j,k+1}$ and travels in the direction of $s_{j,k}$. In this case, if $j$
is supposed to to leave $s_{j,k}$ before $j'$ leaves $s_{j,k+1}$, the following
condition has to be satisfied:
\begin{equation}
  \label{eq:deadlock}
  \tout(j', s_{j,k+1}) \ge \tout(j,s_{j,k}) + \tauuu(j, s),
\end{equation}
where $\tauuu(j, s_{j,k})$ is the minimum time required for train $j$ to
get from station block $s_{j,k}$ to $s_{j,k+1}$. Rewritten in terms of delays, the
inequality \eqref{eq:deadlock} reads:
\begin{equation}
  \label{eq:deadlock2}
  d(j',s_{j,k+1}) \ge d(j, s_{j,k}) + \Delta(j,j',s) + \tauuu(j, s).
\end{equation}
In decision variables, the deadlock condition in its basic form looks as
follows:
\begin{equation}
  \label{eq:qubo:deadlock}
  \forall_{s \in S^{*}_{j} \cap S^{*}_{j'}} \sum_{m \in A_{j, s}} \left(
  \sum_{m' \in C(m) \cap A_{j', s}} x_{j,s,m}x_{j',s,m'}
  \right) = 0,
\end{equation}
and has to be applied for a limited number of trains $j \in \JJ^{0} (\JJ^{1})$
and $j' \in \JJ^{1}(\JJ^{0})$. Here, $C(m)$ is, similarly to $B(m)$, the set of delays
violating the condition for the given pair.
\section{The rolling stock circulation condition}
Our model assumes that some trains might be assigned the same train set. Naturally,
there exists some necessary \emph{turnover time}, before a train set can be
reused. Formally, if trains $j$ and $j'$ going in opposite directions are
assigned the same train set, then the following inequality has to hold:
\begin{equation}
  \tout(j', s_{j',1}) > \tout(j, s_{j,\eend}) + \Delta(j, j')
\end{equation}
where $\Delta(j, j')$ is the minimum turnover time. In the delay
representation, the inequality becomes:
\begin{equation}
  \label{eq:rolling}
  \begin{split}
    d(j',s_{j',1}) + \ttout(j',s_{j',1}) > & \; d(j, s_{j,\eend-1}) + \ttout(j, s_{j,\eend-1}) + \\
    & \; \tauuu(j, s_{j,\eend-1}) + \Delta(j,j').
  \end{split}
\end{equation}
Inequality \eqref{eq:rolling} can be simplified to:
\begin{equation}
  d(j',s_{j',1}) > d(j, s_{j,\eend-1}) - R(j,j'),
\end{equation}
by setting:
\begin{equation}
  \label{eq:rolling2}
  \begin{split}
    R(j, j') \coloneq &\ttout(j',s_{j',1}) - \ttout(j, s_{j,\eend-1}) - \tauuu(j,s_{j,\eend-1}) - \\
                      & \Delta(j,j')
  \end{split}
\end{equation}
In decision variables, the rolling stock circulation condition for trains $j$
and $j'$ can be written as
\begin{equation}
  \label{eq:qubo:rollingstock}
  \sum_{m \in A_{j, s_{(j, \eend-1)}}} \sum_{m' \in E(d) \cap A_{j',s_{(j',1)}}} x_{j,s_{(j,\eend-1)},m}x_{j', s_{(j',1)},m'} = 0
\end{equation}
where $E(d) = \{0, 1, \ldots, m-R(j, j')\}$.

\section{The capacity condition}
Let $s$ be a station block with $b$ tracks and let $\{j_{1},\ldots,j_{b+1}\} \subset \JJ$ be any $b+1$-tuple of
trains. There should not exist time $t$ for which all the following conditions are simultaneously satisfied:
\begin{equation}
  \begin{split}
    \tin(j_{1}, s) \le &t \le \tout(j_{1}, s) \\
    &\ldots \\
    \tin(j_{b+1}, s_{j,k}) \le &t \le \tout(j_{b+1}, s)
  \end{split}
\end{equation}
In delay representation, the conditions read:
\begin{equation}
  \label{eq:buffer}
  \begin{split}
    d(j_{1}, s_{j_{1},k_{1}-1}) &+ \ttout(j_{1},s_{j_{1},k_{1}-1}) \le t \\
                            &\le d(j_{1},s_{j_{1},k_{1}}) + \ttout(j_{1},s_{j_{1},k_{1}})\\
    \ldots \\
    d(j_{b+1}, s_{j_{b+1},k_{b+1}-1}) &+ \ttout(j_{b+1},s_{j_{b+1},k_{b+1}-1}) \le t \\
                            &\le d(j_{b+1},s_{j_{b+1},k_{b+1}}) + \ttout(j_{b+1},s_{j_{b+1},k_{b+1}}),
  \end{split}
\end{equation}
where $k_{j_{i}}$ is the index of station $s$ in sequence $\Sj$.

The condition \eqref{eq:buffer} translated into binary variables can give a lot of additional terms.
In our problem instances, we ignore this condition, but verify the obtained solutions against it.
%%% Local Variables:
%%% mode: latex
%%% TeX-master: "../main"
%%% End


\end{document}
