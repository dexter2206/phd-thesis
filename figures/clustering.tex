\documentclass[dvipsnames]{standalone}
\usepackage{tikz}
\usetikzlibrary{fit,positioning,calc,arrows,arrows.meta}

\begin{document}
\tikzstyle{ghost}=[on grid,thick,circle]
\tikzstyle{graph_vertex}=[
    on grid,circle,thick,draw=blue!75,fill=blue!20
]
\tikzstyle{cluster_vertex}=[
    on grid,circle,thick,minimum size=0.5cm,draw=red,fill=red!20
    ]

\tikzstyle{intra}=[draw=red,thick]
\tikzstyle{inter}=[draw=ForestGreen,thick]
\tikzstyle{cedge}=[draw=ForestGreen,very thick]

\tikzstyle{cluster}=[draw=red,thick,dashed,rounded corners=0.3cm]

\newcommand{\rowheight}{0.75cm}
\newcommand{\colwidth}{1cm}
\newcommand{\cellhspacing}{3cm}
\newcommand{\cellvspacing}{-4cm}

\begin{tikzpicture}
    \begin{scope}
        \node [graph_vertex] (v1) {};
        \node [graph_vertex, right=\colwidth of v1] (v2) {};
        \node [graph_vertex, below=\rowheight of v1] (v3) {};
        \node [graph_vertex, below=\rowheight of v2] (v4) {};
        \node [graph_vertex, below=\rowheight of v3] (v5) {};
        \node [graph_vertex, below=\rowheight of v4] (v6) {};
        \node [graph_vertex, below=\rowheight of v5] (v7) {};
        \node [graph_vertex, below=\rowheight of v6] (v8) {};
        \node [fit=(v1)(v8), draw, cluster] {};
        
        \draw[intra] (v1) -- (v2);
        \draw[intra] (v3) -- (v4);
        \draw[intra] (v5) -- (v6);
        \draw[intra] (v7) -- (v8);
        \draw[intra] (v1) -- (v3);
        \draw[intra] (v2) -- (v4);
        \draw[intra] (v3) -- (v5);
        \draw[intra] (v4) -- (v6);
        \draw[intra] (v5) -- (v7);
        \draw[intra] (v6) -- (v8);
        
    \end{scope}
    
    \begin{scope}[xshift=\cellhspacing]
        \node [ghost] (w1) {};
        \node [ghost, below right=3*\rowheight and \colwidth of w1] (w2) {};
        \node [graph_vertex, below right=1.5 * \rowheight and 0.5 \colwidth of w1] (w3) {};
        \node [fit=(w1)(w2), draw, cluster] {};
    \end{scope}
    
    \begin{scope}[xshift=2*\cellhspacing]
        \node [graph_vertex] (u1) {};
        \node [graph_vertex, right=\colwidth of u1] (u2) {};
        \node [graph_vertex, below=\rowheight of u1] (u3) {};
        \node [graph_vertex, below=\rowheight of u2] (u4) {};
        \node [graph_vertex, below=\rowheight of u3] (u5) {};
        \node [graph_vertex, below=\rowheight of u4] (u6) {};
        \node [graph_vertex, below=\rowheight of u5] (u7) {};
        \node [graph_vertex, below=\rowheight of u6] (u8) {};
        \node [fit=(u1)(u8), draw, cluster] {};
        
        \draw[intra] (u1) -- (u2);
        \draw[intra] (u1) -- (u4);
        \draw[intra] (u1) -- (u6);
        \draw[intra] (u1) -- (u8);
        \draw[intra] (u3) -- (u2);
        \draw[intra] (u3) -- (u4);
        \draw[intra] (u3) -- (u6);
        \draw[intra] (u3) -- (u8);
        \draw[intra] (u5) -- (u2);
        \draw[intra] (u5) -- (u4);
        \draw[intra] (u5) -- (u6);
        \draw[intra] (u5) -- (u8);
        \draw[intra] (u7) -- (u2);
        \draw[intra] (u7) -- (u4);
        \draw[intra] (u7) -- (u6);
        \draw[intra] (u7) -- (u8);
    \end{scope}
    
    \begin{scope}[yshift=\cellvspacing]
        \node [graph_vertex] (s1) {};
        \node [graph_vertex, right=\colwidth of s1] (s2) {};
        \node [graph_vertex, below=\rowheight of s1] (s3) {};
        \node [graph_vertex, below=\rowheight of s2] (s4) {};
        \node [fit=(s1)(s4), draw, cluster] {};
        
        \draw[intra] (s1) -- (s2);
        \draw[intra] (s3) -- (s4);
        \draw[intra] (s1) -- (s3);
        \draw[intra] (s2) -- (s4);
    \end{scope}
    
    \begin{scope}[xshift=\cellhspacing,yshift=\cellvspacing]
        \node [ghost] (t1) {};
        \node [ghost, below right=\rowheight and \colwidth of t1] (t2) {};
        \node [graph_vertex, right=0.5*\colwidth of t1] (t3) {};
        \node [graph_vertex, below=\rowheight of t3] (t4) {};
        \node [fit=(t1)(t2), draw, cluster] {};
        
        \draw[intra] (t3) -- (t4);
    \end{scope}
    
    \begin{scope}[yshift=\cellvspacing,xshift=2*\cellhspacing]
        \node [graph_vertex] (z1) {};
        \node [graph_vertex, right=\colwidth of z1] (z2) {};
        \node [graph_vertex, below=\rowheight of z1] (z3) {};
        \node [graph_vertex, below=\rowheight of z2] (z4) {};
        \node [fit=(z1)(z4), draw, cluster] {};
        
        \draw[intra] (z1) -- (z4);
        \draw[intra] (z2) -- (z3);
    \end{scope}
    
    % Intercluster edges
    \draw[inter] (v3) to [bend right] (s1);
    \draw[inter] (v7) to [bend right] (s3);
    \draw[inter] (v2) to [bend right] (s2);
    \draw[inter] (v4) to [bend right] (s4);
    \draw[inter] (v2) to [bend left] (w3);
    \draw[inter] (v4) to [bend left] (w3);
    \draw[inter] (v6) to [bend left] (w3);
    \draw[inter] (v8) to [bend left] (w3);
    \draw[inter] (w3) to [bend left] (u3);
    \draw[inter] (w3) to [bend left] (u5);
    \draw[inter] (s2) to [bend left] (t3);
    \draw[inter] (s4) to [bend left] (t4);
    \draw[inter] (t3) to [bend left] (z1);
    \draw[inter] (t4) to [bend left] (z1);
    \draw[inter] (t4) to [bend left] (z3);
    
    \draw[inter] (u5) to [bend right] (z1);
    \draw[inter] (u7) to [bend right] (z2);
    \draw[inter] (u8) to [bend right] (z2);
    \draw[inter] (w3) to [bend right] (t3);
    \draw[inter] (w3) to [bend right] (t4);
    
    % Arrow between graphs
    \node [ghost, right=1cm of u8] (arrowstart) {};
    \node [ghost, right=2cm of arrowstart] (arrowstop) {};
    \draw [-{Triangle[width=12pt,length=15pt]}, line width=4pt] (arrowstart) -> (arrowstop);
    
    % Clustered graph
    
    \begin{scope}[xshift=3.5 * \cellhspacing, yshift=-1.1cm]
        \node [cluster_vertex] (c1) {};
        \node [cluster_vertex,right=2 *\colwidth of c1] (c2) {};
        \node [cluster_vertex,right=2 *\colwidth of c2] (c3) {};
        \node [cluster_vertex,below=3 * \rowheight of c1] (c4) {};
        \node [cluster_vertex,below=3 * \rowheight of c2] (c5) {};
        \node [cluster_vertex,below=3 * \rowheight of c3] (c6) {};
        
        \draw[cedge] (c1) -- (c2);
        \draw[cedge] (c2) -- (c3);
        \draw[cedge] (c4) -- (c5);
        \draw[cedge] (c5) -- (c6);
        \draw[cedge] (c1) -- (c4);
        \draw[cedge] (c2) -- (c5);
        \draw[cedge] (c3) -- (c6);
    \end{scope}
\end{tikzpicture}
\end{document}