\chapter[Railway conflict management]{Application to railway conflict management}

As the last point in the thesis, in this chapter we describe how the results
presented so far can be applied in the field of operational research. Namely,
we propose an approach to solving the railway dispatching problem using quantum
annealing. We benchmark the implementation of our algorithm on the current
generation of D-Wave annealers, using solutions obtained via tensor networks
and exhaustive search as a baseline for comparison.

\section{Overview of the problem}
We will consider a part of a railway network, which we will simply refer to as
a \emph{network}. The network is divided into \emph{block sections} or simply
\emph{blocks}. In our approach, we focused only on the single--track railways,
which means that the network can only comprise the following types of blocks:
\begin{itemize}
  \item \emph{Line blocks}, or \emph{single tracks}, sections that can be occupied by one train
    at a time.
  \item \emph{Sidings}, or \emph{parallel tracks} (occurring e.g. at stations). At the sidings,
    trains passing in the same direction can meet--and--overtake (M--O), and trains passing
    in the opposite directions can meet--and--pass (M--P). Each siding comprises two or more
    tracks, each of which can also be occupied by one train at a time. When appropriate, the
    sidings occurring at the stations will be called \emph{station blocks}.
\end{itemize}
Fig. \ref{fig:railway-network} shows an example network.

\begin{figure}[ht]
  \includegraphics[width=\textwidth]{figures/example_line}
  \caption{
    An example network. Sections $2, 3, 4$ and $6$ are \emph{line blocks}, while
    sections $1$ and $5$ are \emph{sidings} with respectively $4$ and $2$ tracks.
    Rectangles represent platforms. Circles represent points where a line block and
    a siding join (white) or where two line blocks join (blue). Superscripts denote
    tracks within a siding. } \label{fig:railway-network}
\end{figure}

The trains move through the network according to a \emph{timetable}. It is
assumed that this timetable is conflict-free, i.e. at any time no two trains
occupy the same track.

Now, suppose the network is affected by a disturbance, which has prevented some
trains from running according to their original timetable. Examples of possible
disturbances include, but are not limited to, a malfunction of one or more
trains or a malfunction of railway tracks. After the disturbance, some trains
occupy different parts of the networks than they are supposed to and resuming
operations according to the original timetable might not be possible. First of
all, some trains might already be delayed too much to be physically able to
follow the timetable. Secondly, trying to closely follow the timetable after
the disturbance might generate conflicts. Hence, a new, conflict--free
timetable has to be promptly created. Preferably, this new timetable should, in
some sense, minimize the resulting delays.

Let us observe that independently from the algorithm for constructing a new
timetable, some delays after the disturbance might be inevitable, e.g. due to
engineering or even physical limits. For instance, if a train has been broken
for some time, it can only follow the timetable if it is not already late and
it can make up for the time it already lost -- and this can only happen if it
can reach a sufficiently large speed. If this is not the case, the train will
be necessarily delayed. Moreover, by taking into account the maximum speed with
which the trains can move through each section, one can calculate the lower
bounds on how much the train will be delayed at each station. These lower
bounds are known as unavoidable, or \emph{primary}, delays.

In the ideal case, if all the trains could travel at their maximum speed, all
trains would be delayed only by their primary delays and there would be nothing
to optimize. However, it might not always be possible. Suppose for instance,
that two trains going in the same direction are already delayed at a station
neighboring a line block. From each train's perspective, the optimal solution
is to start its route immediately when possible. However, as only one of them
can do this because a line block can only be occupied by one train at a time.
Hence, at least one of these two trains will have a delay larger than the
primary one. What is important is that this additional delay is not a
consequence of some physical or engineering limitations, but rather a
consequence of the dispatcher's decision made to avoid a potential conflict.
All such delays are called the \emph{secondary} delays and, unlike the primary
ones, they are subject to optimization.

The distinction between primary and secondary delays might seem artificial at
first, but it has profound consequences. Namely, when constructing a function
to be minimized we only need to take into account the secondary delays. For
instance, we might want to minimize their total sum or their weighted sum, with
weights corresponding to the trains' priorities.

Our high--level description of the problem needs now a mathematical
formulation. We start by describing the details of the network model in the
next section.

\section{The delay representation}
Before we can formulate the optimization problem to be run on D-Wave, we need
first to formally describe the railway model. The first idea that comes to mind
is to define quantities corresponding to departure and arrival times of each
train and relevant station blocks, and express all other quantities in the
model in terms of their difference with respect to the times found in the
original timetable. However, as we will soon see, one can almost completely
forget about arrival and departure times, and instead express all qunatities in
the model using delays. Moreover, we will further simplify our model by
assuming all secondary delays are integers falling into some finite range.

The set of all trains will be denoted by $\JJ$. This set is naturally
partitioned into the set $\JJ_0$ of trains going into one direction and the set
$\JJ_1$ of trains going into the opposite direction. This is a proper
partition, i.e.:
\begin{equation}
  \JJ_0 \cup \JJ_1 = \JJ \quad \JJ_0 \cap \JJ_1 = \emptyset
\end{equation}
For any train $j \in \JJ$ its route is a sequence of blocks. Our model forbids
recirculation, i.e. each train passes every block in its route exactly once.
Furthermore, we assume that each train starts and ends its route at some
station, and its route is uniquely identified by a sequence of station blocks
$\left(s_{j,1}, s_{j, 2}, \ldots, s_{j, \text{end}_j}\right)$, i.e., there are
no alternative routes between any two stations. For convenience, we will denote
the station block preceding $s_{j,k}$ in a given train's route by
$\pi(s_{j,k})$ and the station block succeeding it by $\rho(s_{j,k})$:
\begin{align}
  \pi(s_{j,k})  & = s_{j,k-1} \quad \mbox{for } 2 \le k \le \mbox{end}_j      \\
  \rho(s_{j,k}) & = s_{j,k+1} \quad \mbox{for } 1 \le k \le \mbox{end}_j - 1.
\end{align}
We will denote the time at which the train $j$ should leave the block $s$
according to the original timetable by $\ttout(j, s)$. Similarly, the time at
which the train $j$ is supposed to leave block $s$ will be denoted by $\ttin(j,
  s)$. In our model, we assume that the time at which a train leaves one block is
precisely the same as the time it enters the next block, i.e.
\begin{equation}
  \ttout(j, s) = \ttin(j, \rho_j(s))
\end{equation}
It is clear that the original timetable determines how long it takes for a
train $j$ to travel through a given block $s$. We call this time the passage
time, denoted by $\pt(j, s)$
\begin{equation}
  \pt(j, s) = \ttout(j, s) - \ttin(j, s)
\end{equation}
An important observation is that passage times defined by the timetable may not
be the minimum physically achievable passing times $\pmin(j, s)$. In other
words, for each train $j$ and block $s$ there exists a time reserve
\begin{equation}
  \label{eq:pt}
  0 \le \alpha(j, s) = \pt(j, s) - \pmin(j, s)
\end{equation}
This time reserve will become important when discussing the propagation of the
primary delays.

\subsection{Delay representation}
Suppose the disturbance happened, resulting in some trains not being able to
meet the schedule. Hence, the actual leaving and arrival times (denoted by
$\tout$ and $\tin$) differ from the scheduled ones. The delay $d(j, s)$ of the
train $j$ at block $s$ is defined as the difference
\begin{equation}
  \label{eq:djs}
  d(j, s) \coloneqq \tout(j, s) - \ttout(j, s) %= \tin(j, \rho(s)) - \ttin(j, \rho(s)).
\end{equation}
As already mentioned, $d(j, s)$ can be expressed as a sum
\begin{equation}
  d(j, s) = d_U(j, s) + d_S(j, s)
\end{equation}
where $d_U$ denotes the primary (or unavoidable) delay, and $d_S$ denotes the
secondary delay. In the absence of time reserve, one would simply have $d_U(j,
  s) = d_U(j, s')$ for a given train $j$ and blocks $s$ and $s'$ on its route.
However, the time reserve allows to somewhat compensate the delays
\begin{equation}
  d_U(j, \rho(s)) = \max\{0, d_U(j, s) - \alpha(j, \rho(s))\}
\end{equation}

The secondary delays can be, in principle, arbitrary large. However, it is
convenient to assume that all secondary delays for the train $j$ are bound from
above by some constant $d_{\max}(j)$. One can find a reasonable upper bound by
running some fast heuristic or determine it manually (e.g. there might be an
\emph{a priori} established maximum allowable delay of the train). Henceforth,
we will consider $d_{\max}(j)$ to be a parameter of the model. With this
assumption, we have the following upper and lower bound on the overall delay
\begin{equation}
  d_U(j, s) \le d(j, s) \le d_U(j, s) + d_{\max}(j).
\end{equation}

\section{Discretizing delays}
Formulation of the problem presented so far can facilitate the construction of
a linear, constrained model of the dispatching problem. However, since the
secondary delay values are continuous variables, such a model would not be
compatible with the quantum annealer. We circumvent this issue by discretizing
the delays. One way to do it is to require that all secondary delays are natural
numbers, i.e.
\begin{equation}
  \forall_{j \in \JJ} \forall_{s \in S_{j}}\quad  d_{s}(j, s) \in \{0, 1, \ldots, d_{\max}(j)\}.
\end{equation}
As a consequence, the total delays get discretized as well. We will denote the
set of possible values for $d(j, s)$ by $A_{j, s}$, i.e.
\begin{equation}
  A_{j, s} \coloneq \{d_{U}(j, s), d_{U}(j, s) + 1, \ldots, d_{U}(j, s) + d_{\max}(j)\}.
\end{equation}
Notice that this discretization is not particularly restrictive, as timetables
typically have a finite resolution of minutes anyway.

We can now use one--hot encoding for $d(j, s)$ and introduce binary variables
$x_{s, j, m}$:

\begin{equation}
  \label{eq:onehotconstraint}
  \forall_{j \in \JJ}\forall_{s \in S_{j}} \forall_{m \in A_{j, s}} \quad x_{s,j,m} = \begin{cases}
    1, & d(j, s) = m      \\
    0, & \mbox{otherwise}
  \end{cases}.
\end{equation}
Naturally, possible values for $d(j, s)$ are mutually exclusive, which can be expressed as
the following constraint:
\begin{equation}
  \forall_{j \in \JJ}\forall_{s \in S_{j}} \sum_{m \in A_{j, s}} x_{s, j, m} = 1
\end{equation}
As for the cost function, we decided to use a simple weighted sum of the delays, i.e. the
cost function of the form
\begin{equation}
  \label{eq:qubo:cost}
  f(\mathbf{x}) = \sum_{j \in \mathcal{J}}\sum_{s \in S^{*}_{j}}\sum_{m \in A_{j,s}} w(s,j,m) \cdot x_{j,s,m},
\end{equation}
For instance, choosing $w(s, j, m)=m$ would result in an objective of minimizing the sum of all
delays. In general, however, one could take into account the relative importance of the
trains, as we will describe later when introducing the real railway sections considered in our research.

\section{Dispatching conditions}
The cost function \eqref{eq:cost} together with constraint \eqref{eq:onehotconstraint} is not enough to
construct a meaningful optimization problem. We also have to take into account other constraints stemming
from dispatching conditions. For instance, we cannot allow a schedule in which two trains occupy the
same track at the same time. We will address these additional dispatching conditions next.

\subsection{The minimum passing time condition.}
Train cannot travel through a block faster than the corresponding minimum
passing time
\begin{equation}
  \label{eq:dc1}
  \tout(j, s) \ge \tin(j, s) + \pmin(j, s).
\end{equation}
Using \eqref{eq:djs} and \eqref{eq:pt} one can easily verify that
inequality \eqref{eq:dc1} is equivalent to
\begin{equation}
  \label{eq:passingtime}
  d(j, \rho(s)) \ge d(j, s) - \alpha(j, s, \rho(s)).
\end{equation}
In binary variables, it means that if, for a fixed $j,s,m$, the $x_{j,s,m}=1$, then delays
$d(j,s)$ smaller than $m-\alpha(j, s, \rho_j(s))$ are prohibited and thus the corresponding
  variables have to zero out. Hence, we arrive at the following condition:
\begin{equation}
  \label{eq:qubo:passingtime}
  \forall_{j} \forall_{s \in S_j \setminus \{s_{{j,\text{end}}}\}}
  \sum_{d \in A_{j,s}}
  \left(
  \sum_{ d' \in D(d) \cap A_{j, \rho_j(s)}} x_{j, s, d}
  x_{j, \rho_j(s), d'} \right) = 0,
\end{equation}
where $D(d) = \{0, 1, \ldots, d - \alpha(j, s, \rho_j(s)) -1\}$.
\subsection{The single block occupation condition.}
Two trains cannot occupy the same part of a single
railway track. Consider two trains, $j, j' \in \JJ_0$ leaving the same station $s$ in the direction
of the next station block $\rho_j(s)$. Suppose further that the train $j$ leaves first. i.e.
$\tout(j', s) > \tout(j, s)$. Since two trains cannot occupy the same block, some amount of time has
to pass after $\tout(j, s)$ before the train $j'$ can leave. This amount of time is dependent on
both $j$ and a sequence of blocks, and hence we denote it by $\tauu(j, s, \rho_j(s))$. Thus, the
condition becomes
\begin{equation}
  \label{eq:single-block}
  \tout(j', s) \ge \tout(j, s) + \tauu(j, s, \rho_j(s)).
\end{equation}
Substituting for $\tout$ in \eqref{eq:single-block} yields the following
inequality for delays
\begin{equation}
  \label{eq:single-block-delays}
  d(j', s) \ge d(j, s) + \ttout(j, s) - \ttout(j', s) + \tauu(j, s, \rho_j(s))
\end{equation}
or,
\begin{equation}
  d(j', s) \ge d(j, s) + \Delta(j, s, j', s) + \tauu(j, s, \rho_j(s))
\end{equation}
where
\begin{equation}
  \label{eq:delta}
  \Delta(j, s, j', s) = \ttout(j, s) - \ttout(j', s)
\end{equation}
The precise form of $\tauu$ depends on the dispatching detail of the problem.
In our approach, we propose the following form:
\begin{equation}
  \tauu(j, s) = \max_{i \in \{k+1,\ldots,l-1\}}(\ttin(j, m_{i+1}) - \ttin(j, m_i))
\end{equation}

For our decision variables, we use a similar scheme as with the previous constraint, and the
condition becomes:
\begin{equation}
  \label{eq:qubo:singleblock}
  \forall_{i=0,1} \forall_{j, j' \in \JJ^{i}} \forall_{s \in S^{*}_{j} \cap S^{*}_{j'}} \sum_{m \in A_{j, s}} \left(
    \sum_{m' \in B(m) \cap A_{j', s}} x_{j,s,m}x_{j',s,m'}
  \right) = 0
\end{equation}
\section{The deadlock condition}
The deadlock condition is analogous to the single block occupation condition but for
trains going into opposite directions. Suppose trains $j$ and $j'$ are heading in the opposite
directions on a route determined by two consecutive stations
$s$ and $\rho_j(s)$. Note that for $j'$ the order is reversed, i.e. it starts at $\rho_j(s)$ and
travels in the direction of $s$. In this case, $j$ has to arrive at $\rho_j(s)$ before $j'$ can
leave $\rho_j(s)$. We formalize it as:
\begin{equation}
  \label{eq:deadlock}
  \tout(j', \rho_j(s)) \ge \tout(j,s) + \tauuu(j, s, \rho_j(s)),
\end{equation}
where $\tauuu(j, s, \rho_j(s))$ is the minimum time required for train $j$ to get from station
block $s$ to $\rho_{j}(s)$.
Rewritten in terms of delays, the inequality \eqref{eq:deadlock} reads:
\begin{equation}
  \label{eq:deadlock2}
  d(j',\rho_j(s)) \ge d(j, s) + \Delta(j,s,j',\rho_j(s)) + \tauuu(j, s, \rho_j(s)).
\end{equation}
This condition is to be applied if train $j$ is supposed to leave before train $j'$ leaves
$\rho_{j}(s)$, otherwise the order has to be appropriately reversed.

In decision variables, the deadlock condition in its basic form looks as follows
\begin{equation}
  \label{eq:qubo:deadlock}
  \forall_{s \in S^{*}_{j} \cap S^{*}_{j'}} \sum_{m \in A_{j, s}} \left(
    \sum_{m' \in B(m) \cap A_{j', s}} x_{j,s,m}x_{j',s,m'}
  \right) = 0,
\end{equation}
and has to be applied for limited number of trains $j \in \JJ^{0} (\JJ^{1})$ and $j' \in \JJ^{1}(\JJ^{0})$.
This limit is imposed indirectly by the upper bounds on the delays.
\section{The rolling stock circulation condition}
Our model assumes that some trains are assigned the same train set. Naturally, there exists some necessary \emph{turnover time}, before a train set can be used. Formally, if trains $j$ and $j'$ going in opposite
directions are assigned the same train set, then the following inequality has to hold:
\begin{equation}
  \tout(j', s_{j',1}) > \tout(j, s_{j,end}) + \Delta(j, j')
\end{equation}
where $\Delta(j, j')$ is the minimum turnover time. In the delay
representation, the inequality becomes:
\begin{equation}
  \label{eq:rolling}
  \begin{split}
    d(j',s_{j',1}) + \ttout(j',s_{j',1}) > & \; d(j, s_{j,end-1}) + \ttout(j, s_{j,end-1}) + \\
    & \; \tauuu(j, s_{j,end-1}) + \Delta(j,j').
  \end{split}
\end{equation}
Inequality \eqref{eq:rolling} can be simplified to
\begin{equation}
  d(j',s_{j',1}) > d(j, s_{j,end-1}) - R(j,j'),
\end{equation}
by setting
\begin{equation}
  \label{eq:rolling2}
  R(j, j') = \ttout(j',s_{j',1}) - \ttout(j, s_{j,end-1}) - \tauuu(j,s_{j,end-1})
\end{equation}
In decision variables, the rolling stock circulation condition for trains $j$ and $j'$ can
be written as
\begin{equation}
  \label{eq:qubo:rollingstock}
  \sum_{m \in A_{j, s_{(j, end-1)}}} \sum_{m' \in E(d) \cap A_{j',s_{(j',1)}}} x_{j,s_{(j,end-1)},m}x_{j', s_{(j',1)},m'} = 0
\end{equation}
where $E(d) = \{0, 1, \ldots, m-R(j, j')\}$.

\subsection{Penalties}
The conditions \eqref{eq:qubo:passingtime}, \eqref{eq:qubo:singleblock}, \eqref{eq:qubo:deadlock}, \eqref{eq:qubo:rollingstock} together with cost function \eqref{eq:qubo:cost} define a constrained $0-1$ problem.
However, in use a quantum annealer, we must convert it to QUBO, which means we have to incorporate those constraints
into the cost function.

One might observe that penalties defined by the dispatching conditions are of the form:
\begin{equation}
  \label{eq:quadraticpenalty}
  \sum_{(j, j') \in \mathcal{V}_{p}} x_{i}x_{j} = 0,
\end{equation}
for some set of pairs of indices $\mathcal{V}_{p}$. For every feasible solution (i.e. one meeting all the constraints) the sum in equation \eqref{eq:quadraticpenalty} is 0, whereas violation of the corresponding condition gives a strictly
positive value. Hance, one can add such a sum to the cost function, effectively penalizing the infeasible solutions.
More generally, one might multiply the sum by some constant $p_{pair} \ge 1$, to further increase the value of the
cost function for the infeasible solutions.

The same reasoning cannot be applied e.g. to the constraint \eqref{eq:onehot}, which comprises equations of the form
\begin{equation}
  \label{eq:linearpenalty}
  \sum_{i \in \mathcal{V}_{s}}x_{i} = 1.
\end{equation}
If one added sums from the equation \eqref{eq:linearpenalty} to the cost function, it would favor the infeasible solution
comprising of all 0s. Instead, one can consider the following quadratic form of the same penalty:
\begin{equation}
  \label{eq:linearpenalty2}
    \left(\sum_{i \in \mathcal{V}_{s}}x_{i} -1 \right)^{2} = 0
  \end{equation}
  In contrast to \eqref{eq:linearpenalty}, this time the left hand is equal to 0 for feasible solution, and a positive value for any solution violating the one--hot encoding constraint. As with previous, quadratic penalties, we might want to multiply such penalties by some constant $p_{sum} > 1$. An important thing to mention here is that the expansion of the left--hand side in \eqref{eq:linearpenalty2} gives a nonzero constant offset, which we will ignore.
  therefore, the final form of the penalty term reads:
  \begin{equation}
    \mathcal{P}_{sum}(\mathbf{x}) = \sum_{\mathcal{V}_{s}}p_{sum}\left(\sum_{i,j \in \mathcal{V}_{s}^{\times 2}, i\ne j} x_{i}x_{j}  - \sum_{i \in \mathcal{V}_{s}}x_{i}\right).
  \end{equation}
  Lastly, the total cost function for our QUBO reads:
  \begin{equation}
    f'(\mathbf{x}) = f(\mathbf{x}) + \mathcal{P}_{sum}(\mathbf{x}) + \mathcal{P}_{pair}(\mathbf{x}).
  \end{equation}
  \section{Results}

\subsection{Studied railway segments}
In our work, we considered two single-track railway lines managed by the polish
state--owned company PKP Polskie Linie Kolejowe:

\begin{itemize}
  \item Railway line No. 216 (Nidzica -- Olsztynek section)
  \item Railway line No. 191 (Goleszów -- Wisła Uzdrowisko section)
\end{itemize}

The segments are depicted in Fig. \ref{fig:linesmall:line} and Fig. \ref{fig:linelarge:line}.
For the railway line No. 216, we considered its official train schedule (as of
April 2020). The line No. 191 was undergoing a renovation at the time of
writing, and hence it had no available timetable. Based on the planned
parameters of the line, as described in the official documents \cite{PKPPLK},
we constructed a cyclic timetable. Initial, undisturbed timetables are depicted
in Fig. \ref{fig:linesmall:diagram} and Fig. \ref{fig:linelarge:diagram}.

\begin{figure}
  \begin{subfigure}{\textwidth}
    \caption{}\label{fig:linesmall:line}
    \includegraphics[width=\textwidth]{figures/line_small.pdf}
  \end{subfigure}
  \begin{subfigure}{\textwidth}
    \caption{}\label{fig:linesmall:diagram}
    \includegraphics[width=\textwidth]{figures/train_diagram_small}
  \end{subfigure}
  \caption{\subref{fig:linesmall:line} Nidzica -- Olsztynek segment of line No. 216. The segment comprises three
      station blocks (1 -- Nidzica, 3 -- Waplewo, 5 -- Olsztynek), and two line
      blocks (2, 4). We assume that passing through the station block takes the same
      amount of time independently of which track is used. \subref{fig:linesmall:diagram} Train diagram for the undisturbed timetable of the line in \subref{fig:linesmall:line}. The timetable features two
      \emph{Inter--City} trains IC3521 and IC3520 and one \emph{Regio} train R90602. The paths for the
      \emph{Inter-City} trains are marked with red and path of the \emph{Regio} train is marked with black.}
    \label{fig:linesmall}
\end{figure}

\begin{figure}
  \begin{subfigure}{\textwidth}
    \caption{}\label{fig:linelarge:line}
    \includegraphics[width=\textwidth]{figures/line.pdf}
  \end{subfigure}
  \begin{subfigure}{\textwidth}
    \caption{}\label{fig:linelarge:diagram}
    \includegraphics[width=\textwidth]{figures/train_diagram}
  \end{subfigure}
  \caption{\subref{fig:linelarge:line} Goleszów -- Wisła Uzdrowisko segment of line No. 191. The segment comprises 4
      station blocks (1 -- Goleszów, 3 -- Ustroń, 7 -- Ustroń Polana, 10 -- Wisła
      Uzdrowisko) and 6 line blocks (2, 4, 5, 6, 8, 9). Between line blocks there are
      additional passenger platforms at Ustroń Zdrój, Ustroń Poniwiec and Wisła
      Jawornik. \subref{fig:linelarge:diagram} Train diagram for the timetable of the line in \subref{fig:linelarge:line}.
      The timetable features two \emph{Inter--City} trains (IC1 and IC2) and four regional trains (Ks1--Ks4).
      The paths of the \emph{Inter--City} trains are marked with red and paths of the regional trains are
      marked with black.}
    \label{fig:linelarge}
\end{figure}

For both network segments, we assume that the minimum waiting times at all
considered stations are 1 minute. Also, we assume that the passing times
through all the line blocks were initially scheduled according to the maximum
permissible speeds.

For the Nidzica--Olsztynek railway segment, we considered two delays. The first
one is a 15 minutes delay of IC5320 from station block 5. The second one is
that of the IC3521 leaving station block 1 5 minutes late. Considering this and
our assumptions, this creates conflicts, where two \emph{Inter--City} trains,
as well as an \emph{Inter--City} train and the regio train, have conflicts at
block 4. The conflicted, infeasible train diagram for this situation is
depicted in Fig. \ref{fig:conflictsmall}.

\begin{figure}
  \begin{subfigure}[b]{0.5\columnwidth}
    \caption{}\label{conflict}
    \includegraphics[width=\textwidth]{figures/small_conflict}
  \end{subfigure}
  \begin{subfigure}[b]{0.5\columnwidth}
    \caption{}\label{resolution}
    \includegraphics[width=\textwidth]{figures/small_solution}
  \end{subfigure}
  \caption{
    \subref{conflict} Conflicted timetable for line No. 191. Compared to the original timetable
    from \ref{fig:linesmall:diagram}, two trains are delayed, resulting in a conflict where all three
    trains meet at block 4. \subref{resolution} Conflict resolution
    via AMCC heuristics. The same solution was obtained using FCFS and FLFS heuristics.
  }
  \label{fig:smallconflict}
\end{figure}

For the Goleszów -- Wisła Uzdrowisko line, we considered several scenarios
allowing us to analyze the behavior of our algorithm under different
circumstances. Those scenarios are:

\begin{enumerate}
  \item A moderate delay of the \emph{Inter--City} train starting from the station
    block 1.
  \item A moderate delay of all the trains starting from station block 1.
  \item A significant delay of some trains starting from station block 1.
  \item A significant delay of the \emph{Inter--City} train starting from the station
    block 1.
\end{enumerate}

The delays in all the aforementioned scenarios were chosen so that they indeed
result in a delay. The conflicted timetables are presented in Fig.
\ref{fig:conflictlarge}.

\begin{figure}
  \begin{subfigure}[b]{0.5\textwidth}
    \caption{}\label{c1}
    \includegraphics[width=\textwidth]{figures/case1_conflict}
  \end{subfigure}
  \begin{subfigure}[b]{0.5\textwidth}
    \caption{}\label{c2}
    \includegraphics[width=\textwidth]{figures/case2_conflict}
  \end{subfigure}

  \begin{subfigure}[b]{0.5\textwidth}
    \caption{} \label{c3}
    \includegraphics[width=\textwidth]{figures/case3_conflict}
  \end{subfigure}
  \begin{subfigure}[b]{0.5\textwidth}
    \caption{}\label{c4}
    \includegraphics[width=\textwidth]{figures/case4_conflict}
  \end{subfigure}
  \caption{Conflicted timetables for line No. 216. \subref{c1} Single conflict, observe that the additional delay of Ks$2$ will propagate to the delay of Ks$3$ which is operated with the same train set.
    \subref{c2} two conflicts, with no impact of Ks$2$ on Ks$3$. \subref{c3} Multiple conflicts. \subref{c4}
  One conflict, straightforward to resolve.}
  \label{fig:conflictlarge}
\end{figure}

\subsection{Solution using simple heuristics}
To establish a baseline for Quantum Annealing, we solved the problems described
in the previous section using simple heuristics common in the railways
practice. Those heuristics are:

\begin{itemize}
  \item FCFS (First Come First Served),
  \item FLFS (First Leave First Served),
  \item AMCC (Avoid Maximum Current $C_{\max}$.
\end{itemize}

In FCFS (resp. FLFS) the way is given to the train that first arrives (resp.
first leaves) the considered station block at which the conflict occurs. AMCC
is slightly more complex. In this heuristic, one tries to minimize the maximum
secondary delays of the trains. We want to stress that those heuristics
facilitate different objective functions, and hence it is not possible to directly
compare them -- nevertheless, it might be useful to discuss qualitative
differences between the solutions they produce.

For the case of Nidzica--Olsztynek line, all heuristics returned the same
solution, depicted in Fig. \ref{fig:smallheurresolution}. The conflict is
avoided by delaying IC3521 by another 3-minutes, and allowing R9062 to enter
the block not earlier than 14:25, i.e. 4 minutes later than in the conflicted
timetable. In this case, the additional 4 minutes constitute the maximum
secondary delay of the solution.

We also applied the aforementioned heuristics to all the considered
distrubances in the Goleszów -- Wisła Uzdrowisko segment. For brevity, we
refrain from presenting the detailed discussion of the solutions for all the
cases and limit ourselves to the summary of the maximum secondary delay, which
is presented in Table \ref{tab:simple}

\begin{table}[bh]
  \centering
  \begin{tabular}{ccccc}
    Heuristics & case $1$ & case $2$ & case $3$ & case $4$ \\
    \hline
    FLFS       & 6        & 13       & 4        & 2        \\
    \hline
    FCFS       & 5        & 5        & 5        & 2        \\
    \hline
    AMCC       & 5        & 5        & 4        & 2        \\
    \hline
  \end{tabular}
  \caption{The maximum secondary delays, in minutes, resulting from simple heuristics.
    Observe that for each case, there are solutions far below $d_{\text{max}} =
      10$.} \label{tab:simple}
\end{table}

%%% Local Variables:
%%% mode: pdflatex
%%% TeX-master: "../main"
%%% End:
