\chapter[Railway conflict management]{Application to railway conflict management}
\label{chapter:trains}
As the last point in the thesis, in this chapter we describe how the methods
presented so far can be applied in the field of operational research. Namely,
we propose an approach to solving the railway dispatching problem using quantum
annealing. We benchmark the implementation of our algorithm on the current
generation of D-Wave annealers, using solutions obtained via tensor networks
and exhaustive search as a baseline for comparison.

\section{Overview of the problem}
We will consider a part of a railway network, which we will simply refer to as
a \emph{network}. The network is divided into \emph{block sections} or simply
\emph{blocks}. In our approach, we focused only on the single--track lines,
which means that the network can only comprise the following types of blocks:
\begin{itemize}
  \item \emph{Line blocks}, or \emph{single track} sections, pieces of infrastructure that can be occupied by one train
    at a time.
  \item \emph{Sidings}, or \emph{parallel tracks} (occurring e.g. at stations). At the sidings,
    trains passing in the same direction can meet--and--overtake, and trains passing
    in the opposite directions can meet--and--pass. Each siding comprises two or more
    tracks, each of which can also be occupied by one train at a time. In our examples, the
    sidings will occur at the station, and hence we will also call them \emph{station blocks}.
\end{itemize}
Fig. \ref{fig:railway-network} shows an example network. -
\begin{figure}[ht]
  \includegraphics[width=\textwidth]{figures/example_line}
  \caption{
    An example network. Sections $2, 3, 4$ and $6$ are \emph{line blocks}, while
    sections $1$ and $5$ are \emph{sidings} with respectively $4$ and $2$ tracks.
    Rectangles represent platforms. Circles represent points where a line block and
    a siding join (white) or where two line blocks join (blue). Superscripts denote
    tracks within a siding. } \label{fig:railway-network}
\end{figure}

The trains move through the network according to a \emph{timetable}. It is
assumed that this timetable is conflict-free, i.e. at any time no two trains
occupy the same track.

Now, suppose the network is affected by a disturbance, which has prevented some
trains from running according to their original timetable. Examples of possible
disturbances include, but are not limited to, a malfunction of one or more
trains or a malfunction of railway tracks. After the disturbance, some trains
occupy different parts of the networks than they are supposed to and resuming
operations according to the original timetable might not be possible. The problem
might be viewed from
various perspectives, e.g. that of a passenger or transport operation company
\cite{tornquist,lamorgese,Jensen2016}. In this chapter, we look at the problem
from the perspective of the infrastructure manager whose task, in the presence
of a disturbance, is to create a new, conflict-free timetable. Naturally, in
most cases, there will be multiple possible solutions to the arising conflicts,
and hence one has to decide on what criteria make one timetable more appealing
than another. In our approach, we assume that the dispatcher aims to minimize
some function of the trains' delays, which we will describe later. There are
also other possible choices of the objective function \cite{8795577} such as
the total passenger delay or the total cost of operations.

Let us observe that independently from the algorithm for constructing a new
timetable, some delays after the disturbance might be inevitable, e.g. due to
engineering or even physical limits\footnote{For instance, if a train has been broken
for some time, it can only follow the timetable if it is not already late and
it can make up for the time it already lost -- and this can only happen if it
can reach a sufficiently large speed. If this is not the case, the train will
be necessarily delayed.}. Moreover, by taking into account the maximum speed with
which the trains can move through each section, one can calculate the lower
bounds on the delays. These lower
bounds are known as unavoidable, or \emph{primary}, delays \cite{dariano}.

In the ideal case, if all the trains could travel at their maximum speed, all
trains would be delayed only by their primary delays and there would be nothing
to optimize. However, it might not always be possible. Suppose for instance,
that two trains going in the same direction are already delayed at a station
neighboring a line block. From each train's perspective, the optimal solution
is to start its route immediately when possible. However, as only one of them
can do this because a line block can only be occupied by one train at a time.
Hence, at least one of these two trains will have a delay larger than the
primary one. What is important is that this additional delay is not a
consequence of some physical or engineering limitations, but rather a
consequence of the dispatcher's decision made to avoid a potential conflict.
All such delays are called the \emph{secondary} delays and, unlike the primary
ones, they are subject to optimization.

The distinction between primary and secondary delays might seem artificial at
first, but it has profound consequences. Namely, when constructing a function
to be minimized we only need to take into account the secondary delays. For
instance, we might want to minimize their total sum or their weighted sum, with
weights corresponding to the trains' priorities.

The above high-level description of the problem needs now a mathematical
formulation, which we present in the next section.

\section{The mathematical model}
Before we can formulate the optimization problem to be run on D-Wave, we need
first to formally describe the railway model. The first idea that comes to mind
is to define quantities corresponding to the departure and arrival times of each
train and relevant station blocks and express all other quantities in the
model in terms of their difference with respect to the times found in the
original timetable. However, as we will soon see, one can almost completely
forget about arrival and departure times, and instead express all quantities in
the model using delays. Moreover, we will further simplify our model by
assuming all secondary delays are integers falling into some finite range.

We assume that the analyzed network segment is a sequence $N$ blocks, with
first and the last blocks being station blocks. The set of all trains will be
denoted by $\JJ$. This set is naturally partitioned into the set $\JJ_0$ of
trains going into one direction and the set $\JJ_1$ of trains going into the
opposite direction. This is a proper partition, i.e.:
\begin{equation}
  \JJ_0 \cup \JJ_1 = \JJ \quad \JJ_0 \cap \JJ_1 = \emptyset
\end{equation}
We assume that each train travels through the whole analyzed network segment.
The route $\Bj$ of a train $j$ comprises sequence of blocks:
\begin{equation}
  \Bj \coloneq (b_{j,1}, b_{j,2}, \ldots, b_{j,N})
\end{equation}
Our model forbids recirculation, i.e. each train passes every block in its
route exactly once. Therefore, the route of each train is uniquely identified
by a sequence of station blocks $\Sj$:
\begin{equation}
  \Sj \coloneq  \left(s_{j,1}, s_{j, 2}, \ldots, s_{j, \eend}\right).
\end{equation}
We will denote the time at which the train $j$ should leave a block $b \in \Bj$ according to the
original timetable by $\ttout(j, b)$. Similarly, the time at which the train
$j$ is supposed to enter block $b$ will be denoted by $\ttin(j, b)$. In our
model, we assume that the time at which a train leaves one block is precisely
the same as the time it enters the next block, i.e.
\begin{equation}
  \ttout(j, b_{j,k}) = \ttin(j, b_{j,k+1}).
\end{equation}
It is clear that the original timetable determines how long it takes for a
train $j$ to travel through a given block $b \in \Bj$. We call this time the
\emph{passage time}, and denote it by $\pt(j, b)$:
\begin{equation}
  \pt(j, b) \coloneq \ttout(j, b) - \ttin(j, b).
\end{equation}
An important observation is that the passage times defined by the timetable may
not be the minimum physically achievable passing times $\pmin(j, b)$.
Therefore, one can define a time reserve $\alpha(j, b)$ which can be used by
train $j$ to compensate for the delay when traveling through block $b$:
\begin{equation}
  \label{eq:pt}
  0 \le \alpha(j, b) \coloneq \pt(j, b) - \pmin(j, b).
\end{equation}
The time reserve will become important when discussing the propagation of the
primary delays.

\subsection{Delay representation}
Suppose the disturbance happened, resulting in some trains not being able to
meet the schedule. Hence, the actual leaving and arrival times (denoted by
$\tout$ and $\tin$) differ from the scheduled ones. The delay $d(j, s)$ of the
train $j$ at station block $s \in \Sj$ is defined as the difference:
\begin{equation}
  \label{eq:djs}
  d(j, s) \coloneqq \tout(j, s) - \ttout(j, s) %= \tin(j, \rho(s)) - \ttin(j, \rho(s)).
\end{equation}
As already mentioned, $d(j, s)$ can be expressed as a sum:
\begin{equation}
  d(j, s) = d_U(j, s) + d_S(j, s)
\end{equation}
where $d_U$ denotes the primary (or unavoidable) delay, and $d_S$ denotes the
secondary delay \cite{dariano}. In the absence of time reserve, one would simply have $d_U(j,
  s) = d_U(j, s')$ for any given train $j$ and blocks $s,s' \in S_{j}$. However,
    the time reserve allows to somewhat compensate delays, and hence we have
    \begin{equation}
      d_U(j, s_{j,k+1}) = \max\left\{0, d_U(j, s_{j,k}) - \sum_{b}\alpha(j, b)\right\},
    \end{equation}
    where the sum runs over all blocks starting from the one following $s_{j,k}$
    and ending on $s_{j,k+1}$. The secondary delays can be, in principle,
    arbitrarily large. However, it is convenient to assume that all secondary
    delays for the train $j$ are bound from above by some constant $d_{\max}(j)$.
One can find a reasonable upper bound by running some fast heuristic, or
determine it manually (e.g. there might be an \emph{a priori} established
maximum allowable delay of the train). Henceforth, we will consider
$d_{\max}(j)$ to be parameters of the model. With this assumption, we have the
following bounds on the overall delays:
\begin{equation}
  d_U(j, s) \le d(j, s) \le d_U(j, s) + d_{\max}(j).
\end{equation}

\section{Discretizing delays}
Formulation of the problem presented so far can facilitate the construction of
a linear, constrained model of the dispatching problem. However, since the
secondary delay values are continuous variables, such a model would not be
compatible with the quantum annealer. We circumvent this issue by discretizing
the delays. One way to do it is to require all secondary delays to be natural
numbers, i.e.:
\begin{equation}
  \forall_{j \in \JJ} \forall_{s \in S_{j}}\quad  d_{S}(j, s) \in \{0, 1, \ldots, d_{\max}(j)\}.
\end{equation}
As a consequence, the total delays become discretized as well. We will denote
the set of possible values for $d(j, s)$ by $A_{j, s}$, i.e.:
    \begin{equation}
      \forall_{j}\forall_{s \in \Sj}  A_{j, s} \coloneq \{d_{U}(j, s), d_{U}(j, s) + 1, \ldots, d_{U}(j, s) + d_{\max}(j)\}.
    \end{equation}
    Notice that this discretization is not particularly restrictive, as timetables
    typically have a finite resolution of minutes anyway.

    We can now use one-hot encoding for $d(j, s)$ and introduce the decision binary variables
  $x_{s, j, m}$:

    \begin{equation}
      \forall_{j \in \JJ}\forall_{s \in S_{j}} \forall_{m \in A_{j, s}} \quad x_{s,j,m} = \begin{cases}
        1, & d(j, s) = m      \\
        0, & \mbox{otherwise}
      \end{cases}.
    \end{equation}
    Naturally, possible values for $d(j, s)$ are mutually exclusive, which can be
    expressed as the following constraint:
    \begin{equation}
      \label{eq:onehotconstraint}
      \forall_{j \in \JJ}\forall_{s \in \Sj} \sum_{m \in A_{j, s}} x_{s, j, m} = 1
    \end{equation}
    As for the cost function, we decided to use a simple weighted sum of the
    delays, i.e. the cost function of the form:
    \begin{equation}
      \label{eq:qubo:cost}
      f(\mathbf{x}) = \sum_{j \in \mathcal{J}}\sum_{s \in \Sjs} \sum_{m \in A_{j,s}} w(s,j,m) \cdot x_{j,s,m},
    \end{equation}
    where $\Sjs = \Sj \setminus \{s_{j,\eend}\}$. For instance, choosing $w(s, j,
  m)=m$ would result in an objective of minimizing the sum of all delays. In
general, however, one could take into account the relative importance of the
trains, as we will describe later when introducing the real railway sections
considered in our research.

\section{Dispatching conditions and the penalties}
The cost function \eqref{eq:qubo:cost} together with constraint
\eqref{eq:onehotconstraint} is not enough to construct a meaningful
optimization problem. We also have to take into account other constraints
stemming from dispatching conditions. For instance, we cannot allow a schedule
in which two trains occupy the same track at the same time. We describe the
precise forms of the constraints in detail in the
Appendix~\ref{chapter:dispatching}, and in this section, we will only provide
their brief overview. The dispatching conditions are:
\begin{enumerate}
  \item \textbf{The minimum passing time condition.} Train cannot travel through a block faster than the corresponding minimum passing time.
  \item \textbf{The single block occupation condition.} Two trains cannot occupy the same part of a single railway track.
  \item \textbf{The deadlock condition.} Suppose trains $j$ and $j'$ are
    heading in opposite directions on a route determined by two consecutive
    stations $s_{j,k}$ and $s_{j,k+1}$. In this case, $j$ has to arrive at $s_{j,k+1}$ before $j'$ can leave $s_{j,k+1}$, or vice versa.
  \item \textbf{The rolling stock circulation condition.} Our model assumes that some trains are assigned the same train set. The rolling stock circulation condition ensures there exists some minimum \emph{turnover time}, before a train set can be reused.
\end{enumerate}

The dispatching conditions together with the cost function \eqref{eq:qubo:cost}
and one-hot encoding constraint \eqref{eq:onehotconstraint} define a
constrained $0-1$ problem. However, in order to use a quantum annealer, we must
convert it to QUBO, which means we have to incorporate those constraints into
the objective function.

One might observe that penalties defined by the dispatching conditions (c.f.
Appendix \ref{chapter:dispatching}) are of the form:
\begin{equation}
  \label{eq:quadraticpenalty}
  \sum_{(l, l') \in \mathcal{V}_{p}} x_{l}x_{l'} = 0,
\end{equation}
for some set $\mathcal{V}_{p}$ of pairs of multiindices. For every feasible
solution (i.e. one meeting all the constraints) the sum in the equation
\eqref{eq:quadraticpenalty} is 0, whereas violation of the corresponding
condition gives a strictly positive value. Hence, one can add such a sum to the
cost function, effectively penalizing the infeasible solutions. More generally,
one might multiply the sum by some constant $\ppair > 0$, to further increase
the value of the cost function for the infeasible solutions. Finally, taking
into account all penalties from all dispatching conditions gives the following
term that can be added to the objective function:
\begin{equation}
  \Ppair(\mathbf{x}) = \ppair \sum_{\mathcal{V}_{p}}\sum_{(l, l') \in \mathcal{V}_{p}} x_{l}x_{l'}.
\end{equation}

The same reasoning cannot be applied e.g. to the constraint
\eqref{eq:onehotconstraint}, which comprises equations of the form:
\begin{equation}
  \label{eq:linearpenalty}
  \sum_{l \in \mathcal{V}_{s}}x_{l} = 1.
\end{equation}
If one added sums from the equation \eqref{eq:linearpenalty} to the cost
function, it would favor the infeasible solution comprising all 0s. Instead,
one can consider the following quadratic form of the same penalty:
\begin{equation}
  \label{eq:linearpenalty2}
  \left(\sum_{l \in \mathcal{V}_{s}}x_{l} -1 \right)^{2} = 0
\end{equation}
In contrast to \eqref{eq:linearpenalty}, this time the left-hand side is equal
to 0 for feasible solutions, and takes a positive value for any solution
violating the one-hot encoding constraint. As with previous, quadratic
penalties, we might want to multiply such penalties by some constant $\psum >
  0$. An important thing to mention here is that the expansion of the left-hand
side in \eqref{eq:linearpenalty2} gives a nonzero constant offset, which we
will ignore. Therefore, the final form of the penalty corresponding to the
one-hot encoding constraint is:
\begin{equation}
  \Psum(\mathbf{x}) = \psum \sum_{\mathcal{V}_{s}}\left(\sum_{(l,l') \in \mathcal{V}_{s}^{\times 2}, l\ne l'} x_{l}x_{l'}  - \sum_{l \in \mathcal{V}_{s}}x_{l}\right).
\end{equation}
Lastly, the total objective function for our QUBO reads:
\begin{equation}
  f'(\mathbf{x}) = f(\mathbf{x}) + \Psum(\mathbf{x}) + \Ppair(\mathbf{x}).
\end{equation}
\section{Results}

\subsection{Studied railway segments}
In our work, we considered two single-track railway lines managed by the Polish
state-owned company PKP Polskie Linie Kolejowe:

\begin{itemize}
  \item Railway line No. 216 (Nidzica -- Olsztynek section)
  \item Railway line No. 191 (Goleszów -- Wisła Uzdrowisko section)
\end{itemize}

The segments are depicted in Fig. \ref{fig:linesmall:line} and Fig.
\ref{fig:linelarge:line}. For the railway line No. 216, we considered its
official train schedule (as of April 2020). The line No. 191 was undergoing a
renovation at the time we were conducting our original experiments \cite{railwaydispatching}, and hence it had no available timetable.
Based on the planned parameters of the line, as described in the official
documents \cite{PKPPLK}, we constructed a cyclic timetable. Initial,
undisturbed timetables are depicted in Fig. \ref{fig:linesmall:diagram} and
Fig. \ref{fig:linelarge:diagram}.

\begin{figure}
  \begin{subfigure}{\textwidth}
    \caption{}\label{fig:linesmall:line}
    \includegraphics[width=\textwidth]{figures/line_small.pdf}
  \end{subfigure}
  \begin{subfigure}{\textwidth}
    \caption{}\label{fig:linesmall:diagram}
    \includegraphics[width=\textwidth]{figures/train_diagram_small}
  \end{subfigure}
  \caption{\subref{fig:linesmall:line} Nidzica -- Olsztynek segment of line No. 216. The segment comprises three
    station blocks (1 -- Nidzica, 3 -- Waplewo, 5 -- Olsztynek), and two line
    blocks (2, 4). We assume that passing through the station block takes the same
    amount of time independently of which track is used. \subref{fig:linesmall:diagram} Train diagram for the undisturbed timetable of the line in \subref{fig:linesmall:line}. The timetable features two
    \emph{Inter--City} trains IC3521 and IC3520 and one \emph{Regio} train R90602. The paths for the
    \emph{Inter-City} trains are marked with red and path of the \emph{Regio} train is marked with black.}
  \label{fig:linesmall}
\end{figure}

\begin{figure}
  \begin{subfigure}{\textwidth}
    \caption{}\label{fig:linelarge:line}
    \includegraphics[width=\textwidth]{figures/line.pdf}
  \end{subfigure}
  \begin{subfigure}{\textwidth}
    \caption{}\label{fig:linelarge:diagram}
    \includegraphics[width=\textwidth]{figures/train_diagram}
  \end{subfigure}
  \caption{\subref{fig:linelarge:line} Goleszów -- Wisła Uzdrowisko segment of line No. 191. The segment comprises 4
    station blocks (1 -- Goleszów, 3 -- Ustroń, 7 -- Ustroń Polana, 10 -- Wisła
    Uzdrowisko) and 6 line blocks (2, 4, 5, 6, 8, 9). Between line blocks there are
    additional passenger platforms at Ustroń Zdrój, Ustroń Poniwiec and Wisła
    Jawornik. \subref{fig:linelarge:diagram} Train diagram for the timetable of the line in \subref{fig:linelarge:line}.
    The timetable features two \emph{Inter--City} trains (IC1 and IC2) and four regional trains (Ks1--Ks4).
    The paths of the \emph{Inter--City} trains are marked with red and the paths of the regional trains are
    marked with black.}
  \label{fig:linelarge}
\end{figure}

Timetable for the network segment of line No. 216 includes two
\emph{Inter-City} trains, IC5320 and IC3521, and a regional \emph{Regio} train
R90602. For the line No. 191, the timetable includes two \emph{Inter-City}
trains IC1, IC2 and four regional trains Ks1--Ks4. We assume both
\emph{Inter-City} trains in line No. 191 are operated with the same train set,
with a minimum turnover time (see Appendix \ref{chapter:dispatching}) of
$R(j,j') = 20$ minutes.

For both network segments, we assume that the minimum waiting times at all
considered stations are 1 minute. Also, we assume that the passing times
through all the line blocks were initially scheduled according to the maximum
permissible speeds. As a result of those assumptions, the only possible nonzero
time reserve occurs at the station blocks.

\subsection{Disturbance scenarios}

For the Nidzica--Olsztynek railway segment, we considered a single scenario
with two delays. The purpose of this scenario is to illustrate our approach on
a simple and yet real-world example. The first one is a 15-minute delay of
IC5320 starting from station block 5. The second one is that of the IC3521
leaving first station block 5 minutes late. Considering this and our assumptions,
this creates conflicts where two \emph{Inter--City} trains, as well as an
\emph{Inter--City} train and the \emph{Regio} train, have conflicts at block 4.
The conflicted, infeasible train diagram for this situation is depicted in Fig.
\ref{fig:smallconflict}.

\begin{figure}
  \begin{subfigure}[b]{0.5\columnwidth}
    \caption{}\label{conflict}
    \includegraphics[width=\textwidth]{figures/small_conflict}
  \end{subfigure}
  \begin{subfigure}[b]{0.5\columnwidth}
    \caption{}\label{resolution}
    \includegraphics[width=\textwidth]{figures/small_solution}
  \end{subfigure}
  \caption{
    \subref{conflict} Conflicted timetable for railway segment of line No. 216. Compared to the original timetable
    (Fig. \ref{fig:linesmall:diagram}), two trains are delayed, resulting in two conflicts. The conflicts
    can be quickly identified visually as intersections of train paths at line blocks.
    \subref{resolution} Conflict resolution via AMCC heuristics. The same solution was obtained using FCFS and FLFS heuristics.
  }
  \label{fig:smallconflict}
\end{figure}

For the Goleszów -- Wisła Uzdrowisko line, we considered several different
scenarios, which were designed to illustrate our approach on a larger example:

\begin{enumerate}
  \item A moderate delay of the \emph{Inter--City} train starting from the station
    block 1. This results in a single conflict between IC1 and Ks2.
  \item A moderate delay of all the trains starting from station block 1, resulting in
    two conflicts.
  \item A significant delay of some trains starting from station block 1. Results in
    two conflicts.
  \item A significant delay of the \emph{Inter--City} train IC1 starting from the
    station block 1. Results in a single conflict, which is straightforward to
    resolve.
\end{enumerate}

The delays in all the aforementioned scenarios were chosen so that they indeed
result in conflicts. The conflicted timetables are presented in Fig.
\ref{fig:conflictlarge}.

\begin{figure}
  \begin{subfigure}[b]{0.5\textwidth}
    \caption{}\label{c1}
    \includegraphics[width=\textwidth]{figures/case1_conflict}
  \end{subfigure}
  \begin{subfigure}[b]{0.5\textwidth}
    \caption{}\label{c2}
    \includegraphics[width=\textwidth]{figures/case2_conflict}
  \end{subfigure}

  \begin{subfigure}[b]{0.5\textwidth}
    \caption{} \label{c3}
    \includegraphics[width=\textwidth]{figures/case3_conflict}
  \end{subfigure}
  \begin{subfigure}[b]{0.5\textwidth}
    \caption{}\label{c4}
    \includegraphics[width=\textwidth]{figures/case4_conflict}
  \end{subfigure}
  \caption{Conflicted timetables for line No. 216. \subref{c1} Single conflict, observe
    that the additional delay of Ks$2$ will propagate to the delay of Ks$3$.
    \subref{c2} Two conflicts, with no impact of Ks$2$ on Ks$3$. \subref{c3}
    Multiple conflicts. \subref{c4} One conflict, straightforward to resolve.}
  \label{fig:conflictlarge}
\end{figure}

\subsection{Solution using simple heuristics}
To establish a baseline for Quantum Annealing, we solved the problems described
in the previous section using simple heuristics common in the railways
practice. Those heuristics are:

\begin{itemize}
  \item FCFS (First Come First Served),
  \item FLFS (First Leave First Served),
  \item AMCC (Avoid Maximum Current $C_{\max}$).
\end{itemize}

In FCFS (resp. FLFS) the way is given to the train that first arrives (resp.
first leaves) the considered station block at which the conflict occurs. AMCC
\cite{mascis2002job} is slightly more complex. In this heuristic, one tries to
minimize the maximum secondary delays of the trains. We want to stress that
those heuristics facilitate different objective functions, and hence it is not
possible to directly compare them -- nevertheless, it might be useful to
discuss qualitative differences between the solutions they produce. The
solutions provided by the AMCC heuristic also provide a lower bound for the
values of maximum secondary delays $d_{\max}(j)$, which we will use when
constructing QUBO.

For the case of Nidzica--Olsztynek line, all heuristics returned the same
solution, depicted in Fig. \ref{resolution}. The conflict is avoided by
delaying IC3521 by another 3-minutes, and allowing R9062 to enter the block not
earlier than 14:25, i.e. 4 minutes later than in the conflicted timetable. In
this case, the additional 4 minutes constitute the maximum secondary delay of
the solution.

We also applied the aforementioned simple heuristics to all the considered
disturbances in the Goleszów -- Wisła Uzdrowisko segment. For brevity, we
refrain from presenting a detailed discussion of the solutions for all the
cases and limit ourselves to the summary of the maximum secondary delay, which
is presented in Table \ref{tab:simple}:

\begin{table}[bh]
  \centering
  \begin{tabular}{|c|c|c|c|c|}
    \hline
    \rowcolor{theader} Heuristics & case $1$ & case $2$ & case $3$ & case $4$ \\
    \hline
    FLFS                          & 6        & 13       & 4        & 2        \\
    \hline
    FCFS                          & 5        & 5        & 5        & 2        \\
    \hline
    AMCC                          & 5        & 5        & 4        & 2        \\
    \hline
  \end{tabular}
  \caption{The maximum secondary delays, in minutes, resulting from simple heuristics.
    Observe that for each case, there are solutions far below $d_{\text{max}} =
      10$.} \label{tab:simple}
\end{table}

\subsection{Details on QUBO construction}

To formulate our dispatching problems as QUBO and solve them on the D-Wave
annealer (or using any other method), we first need to decide on the values of
several parameters of the model, as well as the precise form of the cost
function. We start with the latter.

We decided on using the cost function proportional to the secondary delays of
all trains entering their last station block. Additionally, we weight the
contributions of each delay with a coefficient depending on the prioritization
of the corresponding train, resulting in the cost function of the form:
\begin{equation}
  f(\mathbf{x}) = \sum_{j \in J}\left(\sum_{m  \in A_{j,s_{\eend-1}}}w_{j} \frac{d(j,s^{*}) - d_{U}(j,s^{*})}{d_{\max}(j)}x_{j,s^{*},m}\right),
\end{equation}
where $s^{*} = s_{\eend-1}$. The priorities $w_{j}$ are chosen specifically
    for both networks. One can immediately observe that larger values of $w_{j}$
    increase contribution stemming from the delay of a given train, and hence the
    objective function favors solutions with smaller delays for the trains with
    larger priorities. For the segment of line No. 216, we assume $w_{j}= 1.5$ for
all \emph{Inter-City} trains, and $w_{j}=1.0$ for the regional train. This
    prioritization coincides with the usual prioritization of trains in Poland (and
    many countries). For the segment of line No. 191, we decided to adopt a
    slightly more complicated prioritization. For the trains heading toward block
    10, we set a lower priority of $w_{j}=0.9$. For the trains heading in the
    opposite direction, we set $w_{j}=1.5$ and $w_{j}=1.0$ for \emph{Inter-City}
and regional trains respectively. This is because the trains heading towards
block 1 (Goleszów) also head towards important junctions in the Polish railway
network (Katowice for regional trains, and the capital city of Warsaw for
\emph{Inter-city} trains). Our strategy therefore tries to avoid larger delays
in this direction to limit further disturbance to the rest of the network.

As for the maximum secondary delay $d_{\max}$, for simplicity, we assume it is
    the same for all trains. On the one hand, its value cannot be smaller than the
    one returned by the AMCC heuristics. On the other hand, setting this value too
    high increases the number of decision variables and complicates the objective
    function, which is especially undesirable because of the limited number of
    qubits on D-Wave annealers. For line No. 216, we set $d_{\max}=7$ and for line
    No. 191 we set $d_{\max}=10$. The total number of decision variables is given
    by
    \begin{equation}
      \label{eq:numvariables}
      \mbox{\#variables} = (\mbox{\#station blocks}-1) \cdot (\mbox{\#number of trains}) \cdot (d_{\max}+1)
    \end{equation}
    Using formula \eqref{eq:numvariables} we get $2\cdot 3 \cdot 8 = 48$ decision variables for line No. 216
    and $3 \cdot 6 \cdot 11 = 198$ variables for line No. 191. Importantly, the
moderately low number of variables for line no. 216 allows us to solve it using
the brute--force algorithm presented in Chapter \ref{chapter:bruteforce}.

Lastly, we need to choose values for $\ppair$ and $\psum$ penalty weights.
    This is a very subtle choice. On the one hand, setting it too low may cause
    some of the infeasible solutions to have the value of the objective function
    smaller than that of feasible solutions, which is undesirable. On the other
    hand, if penalty weights are too high, the actual cost function becomes merely
    a perturbation for the penalty terms, which is also undesirable. To illustrate
    the difference those weights make to the energy landscape, we computed the
    low-energy spectrum for the problem defined on line No. 216 for several
    different values of $\ppair$ and $\psum$. The obtained energy histograms
    are presented in Fig. \ref{fig:penaltyhistogram}.

\begin{figure}
  \includegraphics[width=\textwidth]{figures/railway_histograms_bf}
  \caption{Energy histogram for feasible (green) and infeasible (red) solutions of QUBO
    defined for line No. 216 with varying penalty weights. The figure takes into
    account the first 5000 low-energy states.} \label{fig:penaltyhistogram}
\end{figure}
In our experiments, we used several combinations of $p_{pair}$ and $p_{sum}$.
For D-Wave 2000Q series devices, which we used for the experiments reported
in~\cite{railwaydispatching}, we used $p_{\text{pair}}=2.7$,
  $p_{\text{sum}}=2.2$ and $p_{\text{pair}} = p_{\text{sum}} = 1.75$.
    Additionally, in this thesis, we extend these results by running
    experiments with $p_{\text{pair}}=p_{\text{sum}}=n$ for $n=2, 3, 4$ on
Advantage and Advantage2 prototype devices.

\subsubsection{Initial experiments on D-Wave annealers}
In our initial experiments, reported in \cite{railwaydispatching}, we used
mostly the D-Wave 2000Q device. We were able to successfully embed all the
problem instances, except case 3 for Line 191. As for the QUBO parameters, we
used $p_{\text{pair}}=2.2$, $p_{\text{sum}}=2.7$ and
  $p_{\text{pair}}=p_{\text{sum}}=1.75$. We used several values of the chain
    strengths, all of them being a multiplicity of $\max|J_{ij}|$ (computed
separately for each problem before the embedding). Following convention from
\cite{railwaydispatching}, we call the multiplier \emph{chain strength scale}
($css$), and in our experiments, it ranged from $1$ to $9$. The annealing time
varied between $5$--$2000\mu$s.

For QUBO defined for Line 216, the D-Wave annealers failed to reach the ground
state for all the tested parameters. However, the lowest energy--solution found
by the annealer was equivalent to the ground state from the dispatching
perspective\footnote{Here, equivalent from the dispatching perspective means
  that the order of trains leaving any given station is the same.).}. The Fig.
\ref{fig:dwtrainsold} shows the solutions obtained from the D-Wave annealer, as
well as the deviation from the ground state energy. Since the best solution was
obtained for $css=2$, we decided to use the same value for the consecutive
experiments.

\begin{figure}
  \begin{subfigure}[b]{0.5\textwidth}
    \caption{}\label{fig:dwtd1}
    \includegraphics[width=\textwidth]{figures/small_DWave22_27_2000_2}
  \end{subfigure}
  \begin{subfigure}[b]{0.5\textwidth}
    \caption{}\label{fig:dwtd2}
    \includegraphics[width=\textwidth]{figures/small_DWave175_175_2000_2}
  \end{subfigure}
  \begin{subfigure}[b]{0.5\textwidth}
    \caption{}\label{fig:dwen1}
    \includegraphics[width=\textwidth]{figures/energy_small_22_27}
  \end{subfigure}
  \begin{subfigure}[b]{0.5\textwidth}
    \caption{}\label{fig:dwen2}
    \includegraphics[width=\textwidth]{figures/energy_small_175_175}
  \end{subfigure}
  \caption{
    \textbf{a.} -- \textbf{b.} Best solutions obtained with D-Wave 2000Q annealer,
    optimized over all annealing times and chain strength scales.
    \textbf{c.} -- \textbf{d.} Energy of the best D-Wave solution as the function of
    $css$ scale. For panels \textbf{a.} and \textbf{c.} we used $\ppair=2.2$, $\psum=2.7$
    and for panels \textbf{b}, \textbf{d.} we used $\ppair=1.75$, $\psum=1.75$.
  }
  \label{fig:dwtrainsold}
\end{figure}

Finding a feasible solution for QUBOs defined for Line 191 proved to be much
more difficult for the D-Wave 2000Q annealer. Hence, we increased the total
number of obtained samples to $250$k. Still, even with the increased number of
samples we were unable to reach a feasible solution. The best solutions found
by the annealer for case 1 and case 2 violate a single constraint and can be
easily corrected to obtain a feasible (and in case 2, even optimal) solution,
see Fig. \ref{fig:dwtrainsoldlarge}.

\begin{figure}
  \begin{subfigure}[b]{0.5\textwidth}
    \caption{}\label{fig:dwtd1large}
    \includegraphics[width=\textwidth]{figures/sol_case1_DWave_1400_2_250k}
  \end{subfigure}
  \begin{subfigure}[b]{0.5\textwidth}
    \caption{}\label{fig:dwtd2large}
    \includegraphics[width=\textwidth]{figures/sol_case2_DWave_1200_2_250k}
  \end{subfigure}
  \caption{
    Lowest energy solutions obtained for QUBO problems defined for Line 191. In all
    panels $\ppair = 2.2, \psum = 2.7$ and $css=2.0$. \textbf{a.} Solution obtained
    for case 1 (with annealing time $\tau=1400$). The solution is infeasible
    because the train Ks3 stays at the station block 7 shorter than 1 minute. The
    solution can be turned into a feasible one by prolonging the stay of Ks3 at
    station block 7. \textbf{b.} The best solution obtained for case 2
    ($\tau=1200$). The solution is infeasible as Ks3 does not stop at station 7.
    This solution can be made into an optimal one by shortening the stays of Ks3 at
    station 3 and IC2 at station 7. } \label{fig:dwtrainsoldlarge}
\end{figure}

In comparison, the QUBOs for cases 1--4 turn out to not be that challenging for
the classical solvers. Both the tensor networks algorithm, described in Chapter
\ref{chapter:tn}, and the IBM CPLEX solver were able to find high-quality
solutions equivalent to the ground state from the dispatching point of view,
with CPLEX slightly outperforming the tensor network algorithm in cases 3 and
4. The values of the cost function obtained from these solvers are presented in
table \ref{tab:line191classical}. In the same table, we also present, for
reference, values of our objective function for solutions obtained with simple
heuristics.

At the time we were conducting experiments presented in
\cite{railwaydispatching}, the new Advantage System 1.1 device was entering the
market, and we were able to run a very limited set of experiments. We decided
to try a slightly larger problem, constructed by extending the timetable for
Line 191 with more trains. Although we were able to embed it on the device, our
attempts to find a feasible solution on this early Pegasus-based system were
futile. For the details of this part of the experiment, we refer the interested
reader to \cite{railwaydispatching}.

\begin{table}
  \centering
  \begin{tabular}{|c|c|c|c|c|c|}
    \hline
    \rowcolor{theader} \multicolumn{2}{|c|}{Method} & Case 1          & Case 2                      & Case 3                      & Case 4                                                    \\
    \hline
    \multirow{2}{*}{QUBO model}                     & CPLEX           & \textcolor{RoyalBlue}{0.54} & \textcolor{RoyalBlue}{1.40} & \textcolor{RoyalBlue}{0.73} & \textcolor{RoyalBlue}{0.20} \\
    \cline{2-6}
                                                    & Tensor Networks & \textcolor{RoyalBlue}{0.54} & \textcolor{RoyalBlue}{1.40} & \textcolor{RoyalBlue}{1.65} & \textcolor{RoyalBlue}{0.29} \\
    \hline
    \multirow{3}{*}{Simple heuristics}              & AMCC            & 0.77                        & \textcolor{RoyalBlue}{1.30} & \textcolor{RoyalBlue}{0.73} & \textcolor{RoyalBlue}{0.20} \\
    \cline{2-6}
                                                    & FLFS            & \textcolor{RoyalBlue}{0.54} & 1.71                        & \textcolor{RoyalBlue}{0.73} & \textcolor{RoyalBlue}{0.20} \\
    \cline{2-6}
                                                    & FCFS            & 0.77                        & \textcolor{RoyalBlue}{1.30} & 0.95                        & \textcolor{RoyalBlue}{0.20} \\
    \hline
  \end{tabular}
  \caption{
    Values of the cost functions obtained by the classical solvers for the QUBO
    problems defined for line 191. Values marked with \textcolor{RoyalBlue}{blue}
    represent solutions equivalent (from the dispatching perspective) to the ground
    state of the corresponding problem. Values for the solutions obtained with
    simple heuristics are provided for reference, but it should be noted that those
    methods use different objective functions and hence cannot be directly compared
    to CPLEX or tensor networks-based solver. } \label{tab:line191classical}
\end{table}

\subsubsection{Extended experiment on the Advantage System annealers}

Since the time of our experiments described in the previous section, new models
of the annealers from the Advantage System generation became available.
Furthermore, the first Advantage2 Prototype devices entered the market. We
decided to extend our experiment and run further tests to investigate the
performance of these devices for a broader range of parameters. To this end, we
decided to test how the newer Pegasus-based devices perform on the QUBO problem
defined on Line 216. We decided that due to the limitation of our resources, we
could not run comprehensive experiments with the problem cases defined on Line
191, and instead opted for a more comprehensive sweep through the parameter
space for the smaller problem.

In this new scenario, we decided to increase $\ppair$ and $\psum$ values, to
investigate if a wider energy separation between feasible and infeasible
solutions will be beneficial for the annealers' performance. As previously, the
annealing times varied from $\tau=5$ to $\tau=2000$. We used chain strengths
varying between $4$ and $12$. We would like to stress, that here we mean
absolute values of the chain strengths and not scales of chain strengths in
relation to the maximum absolute value of quadratic terms of the problem like
in the initial experiment.

All of the annealers were able to find a feasible solution to the problem for
at least some combination of parameters. However, their performance varied
highly depending on the parameter range. The frequency of finding a feasible
solution by the annealers is depicted in Fig. \ref{fig:dwline216freq}. As seen
there, the Advantage System6.3 and Advantage2 Prototype1.1 devices exhibited
much better performance than the older Advantage System4.1 device. As for the
quality of the solutions, all solvers managed to find an optimal solution,
although with different success rates. The summary of parameters for which a
ground state was obtained is presented in table \ref{tab:line216ground}.
Example ground states found are depicted in Fig. \ref{fig:dwline216grounds}.

\begin{figure}[t]
  \begin{subfigure}[b]{0.5\textwidth}
    \caption{}
    \includegraphics[width=\textwidth]{figures/dwave_line216_ground1}
  \end{subfigure}
  \begin{subfigure}[b]{0.5\textwidth}
    \caption{}
    \includegraphics[width=\textwidth]{figures/dwave_line216_ground1.pdf}
  \end{subfigure}
  \caption{Example ground state solutions for the conflicted timetable of Line 216. All
    other ground states are equivalent from the dispatching point of view.}
  \label{fig:dwline216grounds}
\end{figure}

\begin{table}
  \small
  \centering
  \begin{tabular}{|c|c|c|c|}
    \hline
    \rowcolor{theader} Solver & chain strength & annealing time & \# occurrences \\
    \hline
    Advantage System4.1       & 10             & 200            & 1              \\
    \hline
    Advantage System4.1       & 12             & 500            & 1              \\
    \hline
    \hline
    Advantage System6.3       & 10             & 500            & 1              \\
    \hline
    Advantage System6.3       & 12             & 100            & 1              \\
    \hline
    \hline
    Advantage2 Prototype1.1   & 12             & 5              & 1              \\
    \hline
    Advantage2 Prototype1.1   & 12             & 100            & 3              \\
    \hline
    Advantage2 Prototype1.1   & 12             & 1000           & 4              \\
    \hline
    Advantage2 Prototype1.1   & 12             & 2000           & 1              \\
    \hline
  \end{tabular}
  \caption{Parameters for which the D-Wave annealers managed to find the optimal solution to
    the problem defined on Line 216. All samples with ground states occurred at
    $p_{\text{pair}}=p_{\text{sum}}=4.0$. } \label{tab:line216ground}
\end{table}

\begin{figure}
  \includegraphics[width=\textwidth]{figures/dwave_line_216_result.pdf}
  \caption{
    Frequency of finding a feasible solution for the problem defined for Line 216.
    Rows in the grid correspond to different values of chain strength and the
    columns correspond to different values of penalty scalings. In each cell, the
    X-axis depicts the annealing time $\tau$, while the $Y$-axis depicts the
    obtained fraction of the feasible solution (out of 1000 samples) }
  \label{fig:dwline216freq}
\end{figure}

One of the interesting observations one could make about the results presented
in Fig. \ref{fig:dwline216freq} is the performance difference between different
models of the annealers, which seem to be highly dependent on the regime of
parameters. Determining the sources of these differences requires further
research, but it is possible that they can be partially explained by
differences in the available range of quadratic coefficients between the
devices (see D-Wave QPU datasheets \cite{dwavedocs}), which in turn might
affect the DAC quantization effect (see discussion of error sources in Section
\ref{sec:parallel-in-time}).

%%% Local Variables:
%%% mode: pdflatex
%%% TeX-master: "../main"
%%% End:
