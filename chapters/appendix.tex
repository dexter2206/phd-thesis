\chapter{Asymptotic notation}

In order to characterize the complexity of algorithms, it is useful to use
asymptotic big-O notation. Consider two functions $f, g \colon \NN \to \RR$. We
say that $f$ is $O(g)$ if and only if there exists a constant $C > 0$ and a
natural number $n_{C}$ such that the inequality $0 \le f(n) \le C\cdot g(n)$
    holds for all $n > n_{C}$ \cite{clrs}. It is common to write $f=O(g)$ instead
of ``$f$ is $O(g)$'', slightly abusing the mathematical notation \cite{clrs}.
One should notice that big-O notation does not provide a tight bound. For
instance, we have $n + 1 = O(n)$ (since $n + 1 \le 2 \cdot n$) but also $n+1 =
  O(n^{10})$.

    In the context of computational complexity, big-O notation is most commonly
    used for expressing upper bound on number of (dominating) operations performed
    by an algorithm as a function of its input size $N$. Since the number of
    performed operations is roughly proportional to the algorithm's execution time,
    it follows that algorithms with better bound can be considered as more
    performant. However, care must be taken when applying this reasoning to judge
    practical performance. In particular, one should be mindful of the constant
    factor $C$ in the definition above, as well as any bottlenecks stemming from
    the working of the underlying hardware. As a concrete example, Strassen's
    algorithm for multiplying two $N \times N$ matrices requires $O(N^{\alpha})$
    multiplications, where $2 < \alpha < 3$, and yet may perform worse than naive
    algorithm peforming $N^{3}$ multiplications, even for $N$ of order of several
hundreds \cite{dalberto}.

We conclude this section by mentioning that there exist several other
asymptotic notations. For instance, $\Omega$, describing asymptotic lower bound
of a function, and $\Theta$ combining big-O and $\Theta$. For more details, we
refer the reader to \cite{clrs}.

\chapter{Conditional probability on square lattice}
\label{sec:probability}
Consider a square lattice, such as the one depicted in Fig. \ref{fig:lattice-and-border}.
Let denote by $H_X$ the usual Hamiltonian $H$ restricted to the graph
induced by vertices in $X$. Further, let $H_{X, \overline{X}} = H - H_X -
  H_{\overline{X}}$. Notice that $H_{X, \overline{X}}$ contains only quadratic
    terms $J_{ij} s_i s_j$ such that $i \in X$ and $j \in \overline{X}$. Slightly
    abusing the notation, one may thus write
    \begin{equation}
      \small
      H(s_1, \ldots, s_N) = H_X(s_1, \ldots, s_k) + H_{\overline{X}}(s_{k+1}, \ldots, s_N) + H_{X, \overline{X}}(s_1, \ldots, s_N)
    \end{equation}
    Using definition of conditional probability applied to Boltzmann distribution,
    one thus gets

    \begin{align}
      &p(s_{k+1}|s_1, \ldots, s_k) = \frac{\sum\limits_{(z_{k+2}, \ldots, z_N)}e^{-\beta H(s_1, \ldots, s_{k+1}, z_{k+2},\ldots,z_N)}}{\sum\limits_{(z_{k+1}, \ldots, z_N)}e^{-\beta H(s_1, \ldots, s_k, z_{k+1},\ldots,z_N)}}                                                                                                                                                     \\
                                  & = \frac{\sum\limits_{(z_{k+2}, \ldots, z_N)}e^{-\beta (H_X(s_1, \ldots, s_k) + H_{\overline{X}}(s_{k+1}, z_{k+2},\ldots,z_N) + H_{X, \overline{X}}(s_1, \ldots, z_N))}}{\sum\limits_{(z_{k+1}, \ldots, z_N)}e^{-\beta (H_X(s_1, \ldots, s_k) + H_{\overline{X}}(z_{k+1}, \ldots,z_N) + H_{X, \overline{X}}(s_1, \ldots, z_N))}}                 \\
                                  & = \frac{e^{-\beta H_X(s_1, \ldots, s_k)}\sum\limits_{(z_{k+2}, \ldots, z_N)} e^{-\beta(H_{\overline{X}}(s_{k+1}, z_{k+2},\ldots,z_N) + H_{X, \overline{X}}(s_1, \ldots, z_N))}}{e^{-\beta H_X(s_1, \ldots, s_k)}\sum\limits_{(z_{k+1}, \ldots, z_N)}e^{ -\beta(H_{\overline{X}}(z_{k+1}, \ldots,z_N) + H_{X, \overline{X}}(s_1, \ldots, z_N))}} \\
                                  & = \frac{\sum\limits_{(z_{k+2}, \ldots, z_N)} e^{-\beta(H_{\overline{X}}(s_{k+1}, z_{k+2},\ldots,z_N) + H_{X, \overline{X}}(s_1, \ldots, z_N))}}{\sum\limits_{(z_{k+1}, \ldots, z_N)}e^{ -\beta(H_{\overline{X}}(z_{k+1}, \ldots,z_N) + H_{X, \overline{X}}(s_1, \ldots, z_N))}}
      \end{align}
    Note, in both numerator and denominator, spins with indices from $X$ appear
    non-trivially only in $H_{X, \overline{X}}$ , i.e. the whole expression depends
    only on those spins in $X$ that directly interact with spins in $\overline{X}$,
which was to be demonstrated.
%%% Local Variables:
%%% mode: latex
%%% TeX-master: "../main"
%%% End:
