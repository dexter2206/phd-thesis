\chapter{Introduction}
The previous century has witnessed what is now called the digital revolution.
The introduction of digital computers dramatically altered multiple aspects of
our lives. In particular, almost every area of science benefitted hugely from
the increasingly available computational power \cite{winsberg}. Physics was no
exception, and numerical simulations now commonly assist experiments.

Simulating quantum systems -- a holy grail of modern computational physics --
is a highly challenging task for classical computers \cite{feynman.82}. The
difficulties can be blamed on the enormous sizes of the state spaces of such
systems. Direct, naive simulations would require solving systems of
differential equations with the number of variables exponential in the number
of particles. But what about using more sophisticated algorithms? Surprisingly,
it is commonly believed that a sufficiently efficient classical algorithm for
simulating quantum systems does not exist \cite{feynman.82, poplavskii}.
Matters seem even worse when one considers that the increase in the classical
devices' computational power cannot accelerate infinitely. Moore's law
\cite{mack}, which so far well predicted this growth, is expected to slow down
in the years to come \cite{waldrop, kumar}.

If classical computers cannot simulate quantum physics efficiently, what device
can? In the 1980s, Richard Feynman and Paul Benioff put forward the idea that
quantum devices can be used to carry simulations of quantum systems
\cite{feynman.82,benioff.80}. This idea led to the development of several
quantum computation models. In 1985 David Deutsch described a universal,
gate-based quantum computer \cite{deutsch}, a device capable of simulating any
other quantum computer with at most polynomial slowdown. The 1990s and the
early 2000s saw the emergence of another model of quantum computation,
Adiabatic Quantum Computing (AQC) \cite{kadowaki,farhi}. Interestingly, AQC was
later proven to be equivalent to the standard gate-based model \cite{aharonov}.

Just like a classical computer, a quantum computer needs software to run, and
software is based on algorithms, describing how the computation should be
performed. It is not a surprise that quantum computers operate in a very
different way than classical ones, and so they require different, specialized
algorithms. What is surprising is that several notable quantum algorithms were
developed even before the first quantum computers were constructed. In 1994
Peter Shor published his, now famous, algorithm for integer factorization
\cite{shor}. Shor's algorithm demonstrated that quantum computers are (in
principle) capable of solving problems intractable by the classical ones
\cite{kleinjung}. It was also shown that quantum computers could offer a
significant performance boost for easier problems. For instance, in 1996 Grover
presented a quantum algorithm for unstructured database search \cite{grover},
offering quadratic speed-up over classical approaches.

The invention of specialized quantum algorithms further fuelled interest in the
field. In recent years, we observed the development of hardware that brings us
closer to the quantum revolution. Several implementations of gate--based
quantum computers \cite{ionq, bohnet} and quantum annealers \cite{johnson,
  dattani} were constructed and made publicly available. This allowed scientists
to benchmark them and further research their possible applications.

However promising, current quantum computers are far from perfect
\cite{pitfalls,preskill}. Can those noisy devices already be used to solve some
real-world problems? And how does one approach validating if this is the case?
In this thesis, we try to answer these questions, focusing solely on a specific
type of quantum computer -- namely, the D-Wave quantum annealers. We begin the
thesis with an introduction to Ising and QUBO models (collectively known as
Binary Quadratic Models) in Chapter \ref{chapter:ising}. This chapter's purpose
is to lay the necessary foundations for understanding optimization problems
that can be, at least in principle, solved using quantum annealers.

In Chapter \ref{chapter:near-term} we introduce technologies and devices used
for conducting research presented in this thesis. Quite naturally, the first of
those devices are quantum annealers. We briefly describe the principle of
operation of these devices and then move on to discuss currently available
models. We also describe NVIDIA CUDA, another technology that we used for
implementing the brute-force algorithm presented in Chapter
\ref{chapter:bruteforce}.

It is widely believed that a noiseless universal quantum computer would be
capable of simulating quantum systems. But what about the near-term quantum
devices? In Chapter \ref{chapter:simulating} we explore the idea of simulating
the evolution of dynamical (not necessarily quantum) systems using quantum
annealers. We describe how to represent the task of simulating the dynamics as
a static optimization problem and then present experimental results obtained
from the D-Wave annealer. We find out that for small systems, the annealer is
able to faithfully capture the dynamics. We also discuss possible sources of
errors for the problem instances that the annealers failed to solve. While our
algorithm is only a proof of concept, it exemplifies possible directions of
future research.

A key component in assessing the performance of current quantum annealers is
comparing them to the classical algorithms solving the same problems. While
there exists a plethora of general heuristic methods for finding a ground state
of Ising spin-glass, one can ask if it is possible to construct a better
algorithm tailored for problems defined on the same graph as the physical
device. In Chapter \ref{chapter:tn}, we present a recent, heuristic algorithm
for finding the low-energy spectrum of an Ising spin-glass based on tensor
networks, specifically suited for problems defined on Chimera-like graphs.

Chapter \ref{chapter:bruteforce} describes a fast, parallel approach for
exhaustively searching for a low-energy spectrum of Ising spin-glass problems.
Our method is suitable for solving small (less than 54 spins), but otherwise
arbitrary instances. The presented approach can be used for benchmarking other
algorithms that cannot certify their solution. Moreover, the possibility of
finding a low-energy spectrum (instead of a single solution) is extremely
useful for analyzing the structure of the energy landscape of the problem. We
exemplify the usage of our algorithm by conducting benchmarks of a recent
MPS-based algorithm on a set of random spin-glass problems. Compared to the
original algorithm presented in \cite{bruteforce}, the algorithm described in
Chapter \ref{chapter:bruteforce} contains several new, non-trivial
optimizations further increasing the problem sizes that it can tackle.

Lastly, in chapter \ref{chapter:trains}, we present the application of quantum
annealing to solving certain railway dispatching problems. We discuss how such
problems can be converted to QUBO problems suitable for running on the
annealer. We then report the performance of the current generation of quantum
annealers on a set of dispatching problems constructed for real Polish railway
networks. Presented benchmarks extend results presented in \cite{trains} to the
newer generation of quantum annealers. Compared to \cite{trains}, we also
include a more detailed discussion on the influence of penalty terms on the
quality of results.
%%% Local Variables:
%%% mode: latex
%%% TeX-master: "../main"
%%% End:
