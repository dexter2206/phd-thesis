\chapter*{Introduction}
The previous century has witnessed what is now called the digital revolution. Introduction of computers dramatically altered multiple aspects of our lives. In particular, almost every area of science benefitted tremendously from the increasingly available computing power. Physics was no exception and numerical simulations assisting experiments are now a commonplace.

However, simulating dynamics of quantum systems -- a holy grail for modern computational physics -- is still a highly nontrivial task. A natural question to ask is whether advances in technology can bring us closer to this goal.


Moore's law, that so far well predicted rate of growth of computational power of classical computers, is expected to slow down in the years to come. Even more importantly, using any Turing-machine compatible model of computations, simulating quantum systems must result in an exponential slowdown. For those two reasons alone one should expect that efficient simulations of many-body quantum systems, if at all possible, can be done only if one reaches beyond classical architectures of computing devices.

In 1980s Richard Feynmann suggested that quantum devices can be used to carry simulations of quantum systems. This realization opened new avenues of research and led to several different (albeit equivalent) computing paradigms, including quantum annealing and quantum gate computing. Even long before experimental devices implementing those paradigms were constructed, several notable algorithms that could utilize them were developed, which further fueled interest in quantum computing. In recent years we observed unprecedented development of hardware that brings us closer to the quantum revolution. In particular, several implementations of gate-based quantum computers and quantum annealers were developed and made publicly available, allowing scientist to benchmark them and further research their possible applications.

However promising, near-term quantum computers are far from perfect. One could ask whether already these noisy devices can be used for simulating dynamics of quantum systems. In this thesis, we are aiming to show that this is indeed the case, and present-day quantum annealers can be used in hybrid, parallel-in-time algorithms simulating simple quantum systems. While our algorithm is only a proof of concept, it exemplifies possible directions of future research.

Despite the fact that it is known that efficient simulations of sufficiently large quantum systems are outside the reach of the classical computers, it is still unclear where the boundary of quantum supremacy lies. Thus, it is still reasonable to seek new algorithms to push the limits of classical architectures further. This motivated us for developing exhaustive search (a.k.a. brute-force) solver for Ising spinglasses utilizing massively parallel Nvidia CUDA architecture. The solver is capable of finding low-energy spectrum of the spinglass and may be therefore utilized as a tool for benchmarking other heuristic solvers, especially ones that cannot certify their solution.

The structure of this work is as follows. In chapter \ref{chapter:near-term}, we provide a short yet self-contained introduction to the near term technologies used in the chapters that follow. At first, we begin with the description of Nvidia CUDA technology, an example of massively parallel architecture that gained popularity due to multiple applications in numerical computations and machine learning. The rest of the chapter is devoted to quantum annealing and the description of one of its implementations, namely the D-Wave quantum annealers.

The next chapter describes how CUDA can be harnessed for finding low energy spectrum of the Ising spin glass. Here we describe our novel algorithm for efficiently performing an exhaustive search (also known as a bruteforce search) over possible system's configuration and discuss results of extensive benchmarks we subjected our algorithm to. We also show an example application of our bruteforce solver to benchmarking the recently introduced algorithm based on MPS ansatz.

Parallel computations on classical computers are subjected to the Amdahl's law which states that the speedup resulting from scaling execution units horizontally is limited by the serial part of the algorithm. Quantum computers, on the other hand, operate inherently in parallel. While quantum supremacy is still to be demonstrated and currently available hardware is far from perfect, one observes steady improvement in the available technology. This motivated us to develop a proof of concept algorithm for simulating dynamics of a (in principle arbitrary) dynamical system using quantum annealer which we describe in chapter \ref{chapter:simulating}. Here we also discuss experimental results obtained from D-Wave annealers available at the moment of writing and show that even those near-term devices can capture the dynamics of a simple two-qubit system.