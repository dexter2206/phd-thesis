\chapter*{Summary}

In this thesis, we focused on benchmarking quantum annealers and validating their
usefulness in practical settings. One of the most anticipated uses of quantum computers
is simulating physics, or, more precisely, simulating the dynamics of quantum systems.
Therefore, it seems there is no better benchmark for a quantum computer than to test
how far it is from achieving this long--awaited goal. To this end, we described in detail a
proof-of-concept algorithm for simulating the dynamics of a quantum (or in fact, any
dynamical) system using quantum annealers. Although the applicability of the algorithm
to current devices is limited by their small number of qubits and sparse connectivity,
our experiments indicated that already the present-day D-Wave annealers can capture
the dynamics of a very simple two-level system. We also contrasted the obtained results
with the ones produced by several classical solvers, concluding that they perform better
than the tested quantum devices. We also provided a possible explanation why the
particular optimization problems solved in our experiments are particularly hard for
D-Wave devices and checked our predictions with numerical experiments.

To assist in the process of benchmarking the annealers, we developed two distinctive
algorithms. The recent, tensor network-based algorithm allows one to solve Ising
spin--glass instances defined on Chimera graph and other similar layouts. The algorithm
is useful in itself as an optimization approach, but for other research conducted for
this thesis, provided a classical baseline for the results obtained by the D-Wave
annealer. The second of our algorithms, a massively parallelizable distributed bruteforce
algorithm, allows for the exact computation of the low--energy spectrum of small,
but otherwise arbitrary, spin-glass instances. Importantly, this simple yet
efficient algorithm is exact and deterministic. We used the brute-force algorithm
to obtain low-energy spectra for some of the smaller instances used throughout
our experiments. This provided us not only with a means of assessing the quality of
solutions obtained from other solvers or annealers but also with valuable
insights into the structure of the solution space.

To benchmark another anticipated use of quantum annealers, i.e. solving hard
optimization problems stemming from real-life problems, we described an approach
for solving railway-dispatching problems by converting them to QUBO. We then
conducted experiments testing our approach on two generations of D-Wave quantum
annealers. Remarkably, for our tests, we used real railway timetables from two Polish
railway segments. In our experiments, D-Wave annealers were able to successfully
find an optimal solution to the small problem instances, although the performance
varied greatly depending on the parameters such as the annealing time and chain strength.

%%% Local Variables:
%%% mode: latex
%%% TeX-master: "../main"
%%% End:
