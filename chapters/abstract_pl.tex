\chapter{Streszczenie}

W niniejszej pracy skupiamy się na problemie walidowania i benchmarkowania wyżaraczy kwantowych w
praktycznym kontekście. W tym celu, przedstawiamy dwa algorytmy służące do rozwiązywania rzeczywistych
problemów, oraz sprawdzamy, jak dobrze sprawdzają się na obecnej generacji wyżaraczy kwantowych.
Pierwszy z algorytmów pozwala na rozwiązywanie dynamiki kwantowych układów (lub, w gruncie rzeczy,
dowolnych układów dynamicznych). Drugi z przedstawianych algorytmów może skolei zostać użyty do
rozwiązywania pewnego podzbioru problemów kolejowych\todo{Ask KD how to translate ``railway dispatching problem''}:
zarządania opóźnieniami i konfliktami w sieciach kolejowych o jednej linii.
Oceny działania obu w.w. algorytmów na bieżącej generacji wyżaraczy D-Wave dokonujemy z pomocą dwóch,
nowatorskich, klasycznych strategii rozwiązywania szkieł spinowych Isinga, które również prezentujemy
w pracy. Piewrszym z nich, jest opierający się na sieciach tensorowych heurystyczny algorytm
stworzony specjalnie do rozwiązywania szkieł spinowych zdefiniowanych na grafach przypominających
topologię Chimera, co sprawia, że idealnie nadaje się do wyznaczania referencyjnych rozwiązań,
do których można porównać wyniki z fizycznych wyżarzaczy. Drugim z prezentowanych podejść jest masywnie
równoległa implementacja wyczerpującego przeszukiwania całej przestrzeni rozwiązań, tzw. brute-force.
Mimo, że użycie algorytmu brute-force jest ograniczone do instancji o niewielkich rozmiarach,
posiada on tę zaletę, że może wyznaczać niskoenergetyczne spektrum, oraz certyfikować rozwiązaniea.
W związku z tym, algorytm ten może slużyć do uzyskania dodatkowego wglądu w strukturę przestrzeni rozwiązań.
Wyniki otrzymane w naszych eksperymentach sugerują, że już współczesne wyżarzacze są w stanie
uchwycić dynamikę prostych, dwupoziomowych układów kwantowych w specyficznym reżimie parametrów,
oraz mogą znaleźć dobrej jakości rozwiązania instancji kolejowych problemów. Wreszcie, nasze eksperymenty
pokazują jasno, że obecna generacja wyżaraczy D-Wave nie jest idealna. Wymieniamy instancje problemów,
dla których wyżarzanie nie potrafily znaleźść wysokojakościowych, lub nawet dopuszczalnych rozwiązań.
Tam gdzie to możliwe, omawiamy również możliwe wyjaśnienie dlaczego niektóre z prezentowanych instancji
mogą być dla wyżaraczy wymagające.

%%% Local Variables:
%%% mode: latex
%%% TeX-master: "../main"
%%% End:
