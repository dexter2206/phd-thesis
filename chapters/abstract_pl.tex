\chapter{Streszczenie}

\begin{otherlanguage}{polish}
  W niniejszej pracy skupiamy się na problemie walidowania i benchmarkowania
  wyżaraczy kwantowych w praktycznym kontekście. W tym celu, przedstawiamy dwa
  algorytmy służące do rozwiązywania rzeczywistych problemów, oraz sprawdzamy,
  jak dobrze sprawdzają się na obecnej generacji wyżaraczy kwantowych. Pierwszy z
  algorytmów pozwala na rozwiązywanie dynamiki kwantowych układów (lub, w gruncie
  rzeczy, dowolnych układów dynamicznych). Drugi z przedstawianych algorytmów
  może z kolei zostać użyty do rozwiązywania pewnego podzbioru kolejowych
  problemów dyspozytorski: zarządania opóźnieniami i konfliktami w sieciach
  kolejowych o jednej linii. Oceny działania obu w.w. algorytmów na bieżącej
  generacji wyżaraczy D-Wave dokonujemy z pomocą dwóch, nowatorskich, klasycznych
  strategii rozwiązywania szkieł spinowych Isinga, które również prezentujemy w
  niniejszej rozprawie. Pierwszym z nich jest opierający się na sieciach tensorowych
  heurystyczny algorytm stworzony specjalnie do rozwiązywania szkieł spinowych
  zdefiniowanych na grafach przypominających topologię Chimera, co sprawia, że
  idealnie nadaje się do wyznaczania referencyjnych rozwiązań, do których można
  porównać wyniki z fizycznych wyżarzaczy. Drugim z prezentowanych podejść jest
  masywnie równoległa implementacja wyczerpującego przeszukiwania całej
  przestrzeni rozwiązań, tzw. brute-force. Mimo, że użycie algorytmu brute-force
  jest ograniczone do instancji o niewielkich rozmiarach, posiada on tę zaletę,
  że może wyznaczać niskoenergetyczne spektrum, oraz certyfikować rozwiązania. W
  związku z~tym, algorytm przeszukiwania wyczerpującego może slużyć do uzyskania
  dodatkowego wglądu w strukturę przestrzeni rozwiązań. Wyniki otrzymane w
  naszych eksperymentach sugerują, że już współczesne wyżarzacze są w stanie
  uchwycić dynamikę prostych, dwupoziomowych układów kwantowych w specyficznym
  reżimie parametrów, oraz mogą znaleźć dobrej jakości rozwiązania instancji
  kolejowych problemów dyspozytorskich. Wreszcie, nasze eksperymenty pokazują jasno, że obecna
  generacja wyżaraczy D-Wave nie jest idealna. Wymieniamy instancje problemów,
  dla których wyżarzanie nie potrafily znaleźść wysokojakościowych, lub nawet
  dopuszczalnych rozwiązań. Tam gdzie to możliwe, omawiamy również możliwe
  wyjaśnienie dlaczego niektóre z prezentowanych instancji mogą być dla wyżaraczy
  wymagające.
\end{otherlanguage}

%%% Local Variables:
%%% mode: latex
%%% TeX-master: "../main"
%%% End:
